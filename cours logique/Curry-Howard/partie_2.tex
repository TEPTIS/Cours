\part{Isomorphisme de Curry-Howard}

\chapter{Lambda-calcul non typé}

Ce chapitre porte sur la version la plus basique du lambda-calcul, qui est le lambda-calcul non typé.

Le lambda-calcul a d'abord été développé par Church pour rendre compte de la notion de calculabilité, même si ce formalisme s'est porté bien plus fécond qu'attendu. Il est à la base du paradigme fonctionnel en programmation et les langages tels que Haskell et OCaml l'utilisent.

Nous commencerons par introduire la syntaxe du lambda-calcul, avec la forme des termes et les conversions $\alpha$ et $\beta$. Nous allons ensuite montrer que ce formalisme est aussi expressif que les machines de Turing en montrant son équivalence avec les fonctions récursives, puis clore le chapitre sur les théorèmes principaux du lambda-calcul.

\section{Types des structures habituelles}

Cette section sera séparée en plusieurs parties : nous allons chaque fois renforcer notre grammaire de typage et les termes que l'on peut construire, en partant de la notion la plus simple, celle-ci étant de simplement ajouter des types aux lambda-termes habituels.

\subsection{Types des fonctions}

Commençons par introduire la notion de types. On définit pour cela un ensemble de types de base $\iota\in \mathcal B$ qui pourra moralement représenter tous les types habituels concrets, leur donnée exacte n'étant pas pertinente dans l'étude théorique du lambda-calcul simplement typé. On ajoute le constructeur de type $\to$ qui dénote, étant donnés deux types $\sigma$ et $\tau$, le type des fonctions qui à un argument de type $\sigma$ associe une sortie de type $\tau$.

\begin{defi}[Grammaire des types]
    On définit inductivement l'ensemble $T_{\to}$ des types par la grammaire suivante :
    $$\sigma,\tau ::= \iota \mid \sigma\to \tau$$ où $\sigma,\tau\in T_{\to}$ et $\iota\in\mathcal B$.
\end{defi}

\begin{rmk}
    On utilisera une associativité à droite de la flèche : $\tau\to\tau'\to\sigma$ se lira $(\tau\to\tau')\to\sigma$. Cela se justifie de la même manière que notre convention d'associativité à gauche de l'application, car une fonction de la forme $f : (x,y) \mapsto f(x,y)$, une fois curryfiée, donnera une fonction de la forme $f : x\mapsto (y\mapsto f(x,y))$ qui, pour $x$ de type $\tau$, $y$ de type $\tau'$ et $f(x,y)$ de type $\sigma$, a le type $\tau\to(\tau'\to\sigma)$. Comme cette situation arrive très fréquemment, on simplifie l'écriture pour n'avoir qu'à écrire $\tau\to\tau'\to\sigma$.
\end{rmk}

Deux choix sont possibles pour définir un lambda-calcul typé : considérer les termes comme des lambda-termes classiques et y ajouter des annotations de type, ou bien construire des termes déjà typés en partie. Nous opterons pour le second, car il nous semble plus proche de la philosophie de la théorie des types, où les lambda-termes dépendent des types.

\begin{defi}[Lambda-terme typé]
    On définit l'ensemble des pré-lambda-termes typés $\Lambda_0^{\to}$ par la grammaire suivante :
    $$M,N ::= x^\tau\mid \lambda x^\tau.M\mid (M\;N)$$ où $x\in \VV,\tau\in T, M,N\in \Lambda_0^{\to}$.
\end{defi}

Le typage est alors l'action d'associer à un lambda-terme donné, un type. Remarquons que le processus se fait en deux temps : on commence par définir le lambda-terme, avant de définir son type. Cependant, comme nous le prouverons plus tard, la procédure qui à un lambda-terme associe son type est univoque. Nous utiliserons une notation déjà vue en logique, qui est celle des séquents. Pour pouvoir l'utiliser, nous allons définir la notion d'environnement de typage avant de définir les jugements de typage.

Dans $\Lambda_0^{\to}$, nous avons ajouté des annotations de types aux variables et aux abstractions, mais nous noterons rarement les annotations sur les variables. Pour les abstractions, on pourra aussi noter $\lambda (x : \tau).$ à la place de $\lambda x^\tau.$ mais le deuxième étant plus court, il sera privilégié ici.

\begin{defi}[Environnement de typage]
    Un environnement de typage est une liste $\Gamma := (\VV\times T)^*$. On notera en général la liste dans le sens inverse du sens habituel : l'élément de tête sera à droite. L'intuition d'un environnement $\Gamma$ est d'associer à un nombre fini de variables un certain type. Plutôt que $[(x_1,\sigma_1),\ldots,(x_n,\sigma_n)]$ nous écrirons $x_1 : \sigma_1,\ldots,x_n : \sigma : n$ pour écrire la liste, et $\Gamma,x : \sigma$ pour signifie la liste à laquelle on ajoute $(x,\sigma)$ en tête de liste.
\end{defi}

\begin{defi}[Jugement de typage]
    On définit par induction une relation $\vdash \subseteq (\VV\times T)^* \times (\Lambda_0^{\to} \times T)$, nommée relation de typage, qu'on notera $\Gamma\vdash M : \sigma$ pour $\vdash (\Gamma,(M,\sigma))$, par la relation suivante :
    \begin{center}
        \begin{prooftree}
            \infer0{\Gamma,x : \tau \vdash x^\tau : \tau}
        \end{prooftree}
        \qquad
        \begin{prooftree}
            \hypo{\Gamma\vdash M : \tau}
            \infer1[$x\notin \varlib M$]{\Gamma,x : \sigma \vdash M : \tau}
        \end{prooftree}
        
        \vspace{0.5cm}
        
        \begin{prooftree}
            \hypo{\Gamma,x : \sigma\vdash M :\tau}
            \infer1{\Gamma\vdash \lambda x^\sigma. M : \sigma \to \tau}
        \end{prooftree}
        \qquad
        \begin{prooftree}
            \hypo{\Gamma\vdash M : \sigma\to\tau}
            \hypo{\Gamma\vdash N : \sigma}
            \infer2{\Gamma\vdash (M\;N) : \tau}
        \end{prooftree}
    \end{center}

    On appelle jugement de typage une instance de cette relation. Si $\Gamma$ est la liste vide, on notera $\vdash M : \tau$ pour $\varnothing \vdash M : \tau$.
\end{defi}

\begin{expl}
    Nous allons montrer que l'on peut typer le terme $\lambda f^{\iota\to\iota}.\lambda x^{\iota}.f\;(f\;x)$ dans l'environnement vide :
    \begin{center}
        \begin{prooftree}
            \hypo{f : \iota\to\iota \vdash f : \iota\to\iota}
            \infer1{f : \iota\to\iota,x : \iota \vdash f : \iota\to\iota}
            \hypo{f : \iota\to\iota \vdash f : \iota\to\iota}
            \infer1{f : \iota\to\iota,x : \iota \vdash f : \iota\to\iota}
            \hypo{f : \iota\to\iota, x : \iota\vdash x^\iota : \iota}
            \infer2{f : \iota\to\iota, x : \iota \vdash f\; x^\iota : \iota}
            \infer2{f : \iota\to\iota, x :\iota \vdash f\;(f\;x^\iota) : \iota}
            \infer1{f : \iota\to\iota\vdash \lambda x^\iota.f\;(f\;x^\iota) : \iota\to\iota}
            \infer1{\vdash \lambda f^{\iota\to\iota}.\lambda x^\iota. f\;(f\;x^\iota) : (\iota\to\iota)\to\iota\to\iota}
        \end{prooftree}
    \end{center}
\end{expl}

Nous allons maintenant montrer des résultats de structure sur la relation de typage. Ceux-ci seront essentiels pour pouvoir montrer des résultats de façon rapide.

\begin{prop}[Unicité du typage]
    Soient $M\in\Lambda_0^{\to}$ et $\Gamma,\Gamma'$ deux environnements tels que $\Gamma\vdash M : \tau$ et $\Gamma'\vdash M : \sigma$, alors $\tau = \sigma$.
\end{prop}

\begin{proof}
    On procède par induction sur $\Gamma\vdash M : \tau$ :
    \begin{itemize}[label=$\bullet$]
        \item Si $\Gamma\vdash x^{\tau'} : \tau$ où $\Gamma = \Delta,x : \tau$, alors on raisonne par induction sur $\Gamma'\vdash x^{\tau'} : \sigma$ :
        \begin{itemize}[label=$\bullet$]
            \item Si $\Gamma'\vdash x^{\tau'} : \sigma$ où $\Gamma' = \Delta', x : \sigma$ alors par construction $\sigma = \tau'$ et de même $\tau = \tau'$ donc au total $\tau = \sigma$.
            \item Si $\Delta'\vdash x^{\tau'} : \sigma'$ où $\Delta = \Delta',y : \tau'',y\notin\varlib M$ et $\sigma' = \tau$ alors $\Delta \vdash x^{\tau'} : \sigma'$ et $\sigma = \sigma'$ par hypothèse d'induction donc $\sigma = \tau = \sigma'$.
            \item Dans les deux autres cas $M$ ne peut pas être mis sous la forme voulue, donc la prémisse est fausse, menant à une conclusion vraie peu importe la conclusion.
        \end{itemize}
        \item Si $\Delta\vdash M : \tau'$ où $\Gamma = \Delta,x : \tau'$ avec $x\notin\varlib M$ et $\tau' = \sigma$ alors $\Gamma\vdash M : \tau$ et $\tau = \tau' = \sigma$.
        \item Si $M = \lambda x^{\tau'}.N$ et que $\Gamma,x : \tau' \vdash N : \tau''$ où $\tau = \tau'\to\tau''$, alors on raisonne par induction sur $\Gamma'\vdash M$ en éliminant les deux cas non pertinents :
        \begin{itemize}[label=$\bullet$]
            \item Si $\Delta'\vdash M :\sigma$ avec $\Delta',y : \sigma' = \Delta, y \notin\varlib N$ et $\sigma = \tau$ alors on en déduit que $\tau = \sigma$ et $\Delta\vdash M : \sigma$.
            \item Si $\Gamma', x : \tau'\vdash N : \sigma'$ alors par hypothèse d'induction $\sigma' = \tau$ donc $\Gamma'\vdash \lambda x^{\tau'}. N :\tau'\to\sigma'$ et $\Gamma\vdash \lambda x^{\tau'}.N : \tau'\to\sigma'$ donc $\tau = \sigma$.
        \end{itemize}
        \item Si $M = P\;Q$ alors on raisonne de façon analogue au cas précédent pour montrer que $\tau = \sigma$.
    \end{itemize}
    Donc $\tau = \sigma$.
\end{proof}

\begin{exo}
    Montrer le lemme de structure suivant, pour $x\neq y$ : si $\Gamma,x : \tau,y : \tau',\Gamma'\vdash M : \sigma$ alors $\Gamma,y : \tau',x : \tau,\Gamma'\vdash M : \sigma$.
\end{exo}

\begin{exo}
    Montrer le lemme de structure suivant : si $x\notin\varlib M$ et que $\Gamma\vdash M : \tau$ alors $\Gamma'\vdash M : \tau$ où $\Gamma'$ est l'environnement $\Gamma$ où l'on a retiré la dernière occurrence de $x$.
\end{exo}

\begin{defi}[Terme typable]
    On dit qu'un terme $M$ est typable s'il existe un environnement $\Gamma$ et type $\tau$ tels que $\Gamma\vdash M : \tau$. On dit que $M$ est typable dans l'environnement $\Gamma$ s'il existe un type $\tau$ tel que $\Gamma\vdash M : \tau$. On dit que $\tau$ est habité s'il existe un lambda-terme $M$ tel que $\vdash M : \tau$.
\end{defi}

\begin{rmk}
    Si $M$ est typable alors le type associé est unique d'après une propriété précédente.
\end{rmk}

Nous voulons alors quotienter les lambda-termes par $\alpha$-équivalence pour définir $\Lambda^{\to} = \quot{\Lambda_0^{\to}}{=_\alpha}$, mais il faut pour cela définir la substitution sur les termes typés, qui prend en compte la cohérence des types.

\begin{prop}[Substitution typée]
    Soient $M,N\in\Lambda_0^{\to}$, $x\in\VV$ et $\Gamma$ un environnement tel que $$\Gamma\vdash M : \tau\qquad \Gamma\vdash N : \sigma\qquad \Gamma\vdash x : \sigma$$ alors $\Gamma\vdash M[N/x] : \tau$ où $M[N/x]$ est défini comme pour une substitution non typée.
\end{prop}

\begin{proof}
    On procède par induction sur $M[N/x]$ :
    \begin{itemize}[label=$\bullet$]
        \item Si $M = x$, alors $\Gamma\vdash x : \tau$ donc par l'exercice précédent $\tau = \sigma$, d'où $\Gamma\vdash N : \tau$.
        \item Si $M = y$, alors $\Gamma\vdash y : \tau$.
        \item Si $M = \lambda y^{\tau'}.M'$ pour $y\notin\varlib{N}$, et par hypothèse d'induction $\Gamma, y : \tau'\vdash M'[N/x] : \tau''$ et $\tau = \tau'\to\tau''$, alors en appliquant la règle de typage correspondante on en déduit que $\Gamma\vdash \lambda y.M' : \tau'\to\tau''$ d'où $\Gamma\vdash M[N/x] : \tau$.
        \item Si $M = (P\;Q)$, $\Gamma\vdash P[N/x] : \tau'\to\tau$ et $\Gamma\vdash Q[N/x] : \tau'$ alors on en déduit que $\Gamma\vdash (P[N/x]\;Q[N/x]) : \tau$ donc par définition de $[N/x]$ cela signifie $\Gamma\vdash M[N/x] : \tau$.
    \end{itemize}
    Donc par induction $\Gamma\vdash M[N/x] : \tau$.
\end{proof}

\begin{exo}
    Montrer que si $\Gamma\vdash (\lambda x.M) : \tau$ alors $\Gamma'\vdash (\lambda y.M[y/x]) : \tau$ où $y\notin\varlib M\cup\Gamma$ et $\Gamma'$ est l'environnement $\Gamma$ dans lequel on a remplacé chaque couple de la forme $(x,\sigma)$ par $(y,\sigma)$. En déduire que $\lambda x.M$ est typable si et seulement si $\lambda y.M[y/x]$ l'est aussi.
\end{exo}

\begin{defi}[$\alpha$-équivalence]
    On définit notre $\alpha$-équivalence $=_\alpha$ comme la plus petite congruence vérifiant la règle suivante :
    \begin{center}
        \begin{prooftree}
            \infer0[$y\notin\varlib M$]{\lambda x.M =_\alpha \lambda y.M[y/x]}
        \end{prooftree}
    \end{center}

    On pose alors $\Lambda^{\to} = \quot{\Lambda_0^{\to}}{=_\alpha}$ l'ensemble des termes typés.
\end{defi}

\begin{exo}
    Montrer que si $M =_\alpha N$ et $\Gamma\vdash M : \tau$ alors $\Gamma\vdash N : \tau$ et donc que le typage est bien défini sur $\Lambda^{\to}$.
\end{exo}

\begin{them}[Préservation du typage]
    En adaptant la réduction $\reduc$ aux lambda-termes typés de façon évidente, si $\Gamma\vdash M : \tau$ et $M\reduc N$ alors $\Gamma\vdash N : \tau$.
\end{them}

\begin{proof}
    On raisonne par induction sur $M\reduc N$ :
    \begin{itemize}[label=$\bullet$]
        \item Si $\Gamma\vdash M : \tau$ et que $M = (\lambda x^\sigma.P)Q$ avec $N = Q[P/x]$ alors on peut prendre $x\notin\varlib Q$ et faire une inversion sur le typage pour trouver que $\Gamma\vdash Q : \sigma$ et $\Gamma,x : \sigma\vdash P : \tau$, mais cela signifie aussi que $\Gamma,x : \sigma\vdash Q : \sigma$ et $\Gamma, x:\sigma\vdash x : \sigma$ donc $\Gamma,x : \sigma \vdash P[Q/x] : \tau$. De plus comme $x\notin\varlib P[Q/x]$, on en déduit que $\Gamma\vdash P[Q/x] : \tau$.
        \item Les autres cas se déroulent directement en appliquant les hypothèses d'induction.
    \end{itemize}
\end{proof}

On en déduit que $\Lambda^\to_\beta = \quot{\Lambda^{\to}}{\beteq}$ est compatible avec la relation de typage.

\begin{exo}
    Vérifier que le théorème de Church-Rosser s'applique encore pour $\Lambda^{\to}$.
\end{exo}

\begin{exo}
    Vérifier que l'ajout de la règle $\eta$ a les mêmes propriétés que pour $\Lambda$.
\end{exo}

\subsection{Type des paires}

On définit ici un nouveau lambda-calcul, nommé $\Lambda^{\to\times 1}$ qui permet de considérer des paires. En effet, contrairement au lambda-calcul non typé, il n'est pas possible en lambda-calcul de coder directement les paires et les projections. L'argument intuitif est que le codage $\langle M,N\rangle := \lambda p. p\;M\;N$ quantifie $p$ sur tous les types possibles et il faudrait donc pouvoir définir une fonction prenant un type quelconque. Nous verrons que cela est possible si on autorise des quantifications de second ordre, mais le typage simple est juste un typage qui n'utilise pas ces quantifications. On ajoute donc en parallèle les types produits, que l'on peut voir comme des analogues des produits cartésiens, le type unit qui est un type contenant un unique élément, et les fonctions permettant à ces types d'être construits et utilisés.

\begin{defi}[Types produit]
    On définit un nouvel ensemble de types, que l'on notera $T_{\to\times 1}$, par la grammaire suivante :
    $$\sigma,\tau ::= \iota\mid \unit\mid \sigma\to\tau\mid \sigma \times \tau$$ où $\iota\in\mathcal B$, $\unit$ est une constante de type, et $\sigma,\tau\in T_{\to\times 1}$.
\end{defi}

\begin{rmk}
    On note $\times$ prioritaire sur $\to$, donc $\sigma\times\tau\to\kappa$ se lit $(\sigma\times\tau)\to\kappa$.
\end{rmk}

\begin{defi}[$\Lambda_0^{\to\times 1}$]
    On définit l'ensemble $\Lambda_0^{\to\times 1}$ par la grammaire enrichie sur celle de $\Lambda^{\to}$ suivante :
    $$M,N ::= \ldots \mid \langle M,N\rangle \mid \pi_1\;M\mid \pi_2\;M\mid \langle\rangle$$ On définit $=_\alpha$ de la même façon que pour $\Lambda_0^{\to}$ et on définit alors $\Lambda^{\to\times 1} = \quot{\Lambda_0^{\to\times 1}}{=_\alpha}$.
\end{defi}

\begin{defi}[Typage dans $\Lambda^{\to\times 1}$]
    On définit la relation de typage $\vdash$ en ajoutant des règles à la relation de typage $\vdash$ dans $\Lambda^{\to}$ :
    \begin{center}
        \begin{prooftree}
            \hypo{\Gamma\vdash M : \tau}
            \hypo{\Gamma\vdash N : \sigma}
            \infer2{\Gamma\vdash \langle M,N\rangle : \tau\times \sigma}
        \end{prooftree}
        \qquad
        \begin{prooftree}
            \hypo{\Gamma\vdash M : \tau\times\sigma}
            \infer1{\Gamma\vdash \pi_1\;M : \tau}
        \end{prooftree}
        \qquad
        \begin{prooftree}
            \hypo{\Gamma\vdash M : \tau\times\sigma}
            \infer1{\Gamma\vdash \pi_2\;M : \sigma}
        \end{prooftree}

        \vspace{0.5cm}

        \begin{prooftree}
            \infer0{\Gamma\vdash \langle\rangle : \unit}
        \end{prooftree}
    \end{center}
\end{defi}

\begin{exo}
    Montrer que la relation de typage est bien définie sur $\Lambda^{\to\times 1}$ i.e. que si $M=_\alpha N$ et $\Gamma\vdash M : \tau$ alors $\Gamma\vdash N : \tau$. \textit{Indication : on généralisera d'abord les propriétés de structure nécessaires de $\Lambda_0^{\to}$.}
\end{exo}

\begin{defi}[$\beta$-réduction]
    On définit aussi la relation $\reduc\subseteq\Lambda^{\to\times 1}\times\Lambda^{\to\times 1}$ comme la plus petite relation compatible contenant les règles suivantes :
    \begin{center}
        \begin{prooftree}
            \infer0{(\lambda x^{\tau}.M)\;N\reduc M[N/x]}
        \end{prooftree}
        \qquad
        \begin{prooftree}
            \infer0{\pi_1\;\langle M,N\rangle\reduc M}
        \end{prooftree}
        \qquad
        \begin{prooftree}
            \infer0{\pi_2\;\langle M,N\rangle\reduc N}
        \end{prooftree}
    \end{center}

    Et $\beteq$ comme la congruence associée.
\end{defi}

\begin{rmk}
    Comme nous avons augmenté le nombre de constructeurs, les relations que vérifie une congruence ou une relation compatible sont aussi plus nombreuses : 
    \begin{center}
        \begin{prooftree}
            \hypo{M\;\RR\;M'}
            \infer1{\langle M,N\rangle\;\RR\;\langle M',N\rangle}
        \end{prooftree}
        \qquad
        \begin{prooftree}
            \hypo{N\;\RR\;N'}
            \infer1{\langle M,N\rangle\;\RR\;\langle M,N'\rangle}
        \end{prooftree}

        \vspace{0.5cm}

        \begin{prooftree}
            \hypo{M\;\RR\;M'}
            \infer1{\pi_1\;M\;\RR\;\pi_1\;M'}
        \end{prooftree}
        \qquad
        \begin{prooftree}
            \hypo{M\;\RR\;M'}
            \infer1{\pi_2\;M\;\RR\;\pi_2\;M'}
        \end{prooftree}
    \end{center}
\end{rmk}

\begin{exo}
    Vérifier que la préservation du typage est encore vraie.
\end{exo}

\begin{exo}
    Vérifier que $(\Lambda^{\to\times 1},\reduc)$ vérifie bien la propriété de Church-Rosser.
\end{exo}

\begin{defi}[$\eta$-réduction]
    On définit la relation $\reduc_{\beta\eta}\subseteq\Lambda^{\to\times 1}\times\Lambda^{\to\times 1}$ comme la plus petite relation compatible contenant $\reduc$ et les règles suivantes :
    \begin{center}
        \begin{prooftree}
            \infer0[$x\notin\varlib M$]{\lambda x^\tau.M\;x\reduc_{\beta\eta} M}
        \end{prooftree}
        \qquad
        \begin{prooftree}
            \infer0{\langle \pi_1\;M,\pi_2\;M\rangle \reduc_{\beta\eta} M}
        \end{prooftree}
        \qquad
        \begin{prooftree}
            \infer0[$M : \unit$]{M\reduc_{\beta\eta} \langle\rangle}
        \end{prooftree}
    \end{center}

    Et $=_{\beta\eta}$ comme la congruence associée.
\end{defi}

\begin{rmk}
    La propriété de Church-Rosser ne tient pas pour $(\Lambda^{\to\times 1},\reduc_{\beta\eta})$ : en effet, si on prend $x : \tau\times\unit$ et qu'on considère $M := \lambda \pi_1\;x,\pi_2\;x\rangle$ alors on peut au choix réduire $M$ en $x$ ou bien en $\langle \pi_1\;x,\langle\rangle\rangle$ qui sont deux formes normales.
\end{rmk}

\begin{exo}
    Définir la relation $\cong$ sur $\Lambda^{\to\times 1}$ et montrer qu'elle est égale à $=_{\beta\eta}$.
\end{exo}

\subsection{Types des unions disjointes}

Nous allons enfin ajouter un constructeur de type analogue aux sommes d'ensembles, c'est-à-dire aux unions disjointes. En théorie des ensembles, $E + F$, que l'on note aussi $E\sqcup F$, est défini par $E + F := \{(x,0)\mid x\in E\}\cup \{(y,1)\mid y\in F\}$, et on possède alors les fonctions $$\fonction{\kappa_1}{E}{E+F}{x}{(x,0)}\qquad\fonction{\kappa_2}{F}{E+F}{y}{(y,1)}$$ qui sont universelle en ce sens que s'il existe deux fonctions $f : E \to G$ et $g : F\to G$ pour un ensemble $G$ quelconque, alors il existe une unique fonction $[f,g] : E+F\to G$ telle que $[f,g](x,0) = f(x)$ et $[f,g](y,1) = g(y)$ pour $x\in E, y\in F$. Cet ensemble de fonctions et cette propriété universelle caractérisent presque l'ensemble $E+F$ (en fait à bijection près) mais cela suffit largement pour décrire le comportement de l'objet $E+F$. D'un point de vue du lambda-calcul, cette définition utilisant principalement des fonctions est privilégiée à l'approche ensembliste de base car elle permet de définir facilement les constructeurs dont nous avons besoin : un constructeur qui se comporte comme $\kappa_1$, un autre comme $\kappa_2$ et un constructeur de la forme $[-,-]$ qui permet de construire des fonctions $E+F\to G$ à partir d'une fonction $E\to G$ et une fonction $F\to G$. On ajoute enfin un type vide, correspondant à l'ensemble $\varnothing$. En théorie des ensembles on possède une fonction $\varnothing \to E$ pour tout ensemble $E$, donc il faut introduire aussi une telle fonction.

\begin{defi}[Types somme]
    On définit $T_{\to\times 1+0}$ par la grammaire suivante : $$\sigma,\tau ::= \iota\mid\unit\mid\voidt\mid \sigma\to\tau\mid\sigma\times\tau\mid\sigma+\tau$$ où $\iota\in\mathcal B$, $\unit$ et $\voidt$ sont des constantes de types et $\sigma,\tau\in T_{\to\times 1+0}$.
\end{defi}

\begin{rmk}
    On donne une priorité à $+$ intermédiaire entre $\times$ et $\to$ : $\tau\times\sigma+\kappa$ se lit $(\tau\times\sigma)+\kappa$ et $\tau+\sigma\to\kappa$ se lit $(\tau+\sigma)\to\kappa$.
\end{rmk}

\begin{defi}[$\Lambda^{\to\times 1+0}$]
    En quotientant par $\alpha$-équivalence, on définit $\Lambda^{\to\times 1+0}$ en enrichissant la grammaire de $\Lambda^{\to\times 1}$ :
    $$M,N,P::=\ldots\Big|\; \kappa_1\;M\;\Big|\kappa_2\;N\;\Big|\;\delta\;(x\mapsto M\mid y\mapsto N)\;P\;\Big|\; \delta_\bot\;M$$
\end{defi}

\begin{rmk}
    La gestion des variables libre doit être actualisée par rapport au fait que $x$ est liée dans $\delta\;(x\mapsto M\mid y\mapsto N)\;P$, mais cette adaptation est un processus administratif. De plus le constructeur peut prendre des variables différentes, par exemple $\delta\;(a\mapsto M\mid b\mapsto N)\;P$ est aussi valide, et l'identification dans $=_\alpha$ a donc une règle en plus.
\end{rmk}

\begin{defi}[Typage dans $\Lambda^{\to\times 1+0}$]
    On définit $\vdash$ sur $\Lambda^{\to\times 1+0}$ en enrichissant les règles précédentes :
    \begin{center}
        \begin{prooftree}
            \hypo{\Gamma\vdash M : \tau}
            \infer1{\Gamma\vdash \kappa_1\;M : \tau+\sigma}
        \end{prooftree}
        \qquad
        \begin{prooftree}
            \hypo{\Gamma\vdash M : \sigma}
            \infer1{\Gamma\vdash \kappa_1\;M : \tau+\sigma}
        \end{prooftree}
        \qquad
        \begin{prooftree}
            \hypo{\Gamma\vdash P : \sigma+\tau}
            \hypo{\Gamma,x : \sigma\vdash M : \kappa}
            \hypo{\Gamma,y : \tau\vdash N : \kappa}
            \infer3{\Gamma\vdash \delta\;(x\mapsto M\mid y\mapsto N)\;P : \kappa}
        \end{prooftree}

        \vspace{0.5cm}

        \begin{prooftree}
            \hypo{\Gamma\vdash M : \voidt}
            \infer1{\Gamma\vdash \delta_\bot\;M : \tau}
        \end{prooftree}
    \end{center}
\end{defi}

\begin{rmk}
    Il n'y a pas de constructeur de type $\voidt$, ce qui est normal pour un type vide. De plus au lieu de considérer une fonction $M : \sigma\to\kappa$ on considère un terme $M : \kappa$ en ajoutant $x : \sigma$ à l'environnement ; s'il est évident que les deux situations sont équivalentes, la deuxième nous permettra une meilleure visualisation de l'isomorphisme de Curry-Howard développé dans un chapitre ultérieur.
\end{rmk}

\begin{rmk}
    La règle de typage pour $\delta_\bot$ fait qu'il n'y a plus unicité du type d'une expression. On peut contourner ce problème en définissant plutôt $\delta_\bot^\tau$ où on indique le type d'arrivée de $\delta_\bot$, et en considérant qu'on ne l'écrira jamais, de la même façon qu'on note $x$ une variable et non $x^\tau$.
\end{rmk}

On ajoute ensuite les réductions.

\begin{defi}[$\beta$-relations]
    On définit la relation $\reduc$ comme la plus petite relation compatible contenant les règles suivantes :
    \begin{center}
        \begin{prooftree}
            \infer0{(\lambda x^{\tau}.M)\;N\reduc M[N/x]}
        \end{prooftree}
        \qquad
        \begin{prooftree}
            \infer0{\pi_1\;\langle M,N\rangle\reduc M}
        \end{prooftree}
        \qquad
        \begin{prooftree}
            \infer0{\pi_2\;\langle M,N\rangle\reduc N}
        \end{prooftree}

        \vspace{0.5cm}
        
        \begin{prooftree}
            \infer0{\delta\;(x\mapsto M\mid y\mapsto N)\;(\kappa_1\;P)\reduc M[P/x]}
        \end{prooftree}
        \qquad
        \begin{prooftree}
            \infer0{\delta\;(x\mapsto M\mid y\mapsto N)\;(\kappa_2\;P)\reduc N[P/y]}
        \end{prooftree}
    \end{center}

    Et $\beteq$ est alors la congruence associée.
\end{defi}

\begin{rmk}
    On ajoute encore des relations pour la compatibilité :
    \begin{center}
        \begin{prooftree}
            \hypo{M\;\RR\;M'}
            \infer1{\kappa_1\;M\;\RR\;\kappa_1\;M'}
        \end{prooftree}
        \qquad
        \begin{prooftree}
            \hypo{M\;\RR\;M'}
            \infer1{\kappa_2\;M\;\RR\;\kappa_2\;M'}
        \end{prooftree}
        \qquad
        \begin{prooftree}
            \hypo{M\;\RR\;M'}
            \infer1{\delta\;(x\mapsto M\mid y\mapsto N)\;P\;\RR\;\delta\;(x\mapsto M'\mid y\mapsto N)\;P}
        \end{prooftree}

        \vspace{0.5cm}
        
        \begin{prooftree}
            \hypo{N\;\RR\;N'}
            \infer1{\delta\;(x\mapsto M\mid y\mapsto N)\;P\;\RR\;\delta\;(x\mapsto M\mid y\mapsto N')\;P}
        \end{prooftree}
        \qquad
        \begin{prooftree}
            \hypo{P\;\RR\;P'}
            \infer1{\delta\;(x\mapsto M\mid y\mapsto N)\;P\;\RR\;\delta\;(x\mapsto M\mid y\mapsto N)\;P'}
        \end{prooftree}

        \vspace{0.5cm}
        
        \begin{prooftree}
            \hypo{M\;\RR\;M'}
            \infer1{\delta_\bot\;M\;\RR\;\delta_\bot\;M'}
        \end{prooftree}
    \end{center}
\end{rmk}

On notera aussi $\reduc_0$ la réduction en surface, qui est la version de $\reduc$ sans la compatibilité, c'est-à-dire n'effectuant une réduction que si celle-ci apparait directement dans le terme entier.

\begin{defi}[$\beta$-$\eta$-relations]
    On définit la relation $\reduc_{\beta\eta}$ comme la plus petite relation compatible contenant $\reduc$ et les règles suivantes :
    \begin{center}
        \begin{prooftree}
            \infer0[$x\notin\varlib M$]{\lambda x^\tau.M\;x\reduc_{\beta\eta} M}
        \end{prooftree}
        \qquad
        \begin{prooftree}
            \infer0{\langle \pi_1\;M,\pi_2\;M\rangle \reduc_{\beta\eta} M}
        \end{prooftree}
        \qquad
        \begin{prooftree}
            \infer0[$M : \unit$]{M\reduc_{\beta\eta} \langle\rangle}
        \end{prooftree}

        \vspace{0.5cm}

        \begin{prooftree}
            \infer0{\delta\;(x\mapsto M\;(\kappa_1\;x)\mid y\mapsto M\;(\kappa_2\;y))\;P\reduc_{\beta\eta} M\;P}
        \end{prooftree}
    \end{center}

    Et $=_{\beta\eta}$ comme la congruence associée.
\end{defi}

\begin{exo}[Booléens]
    On définit le type $\boolt := \unit + \unit$ et $\top_\boolt := \kappa_1 \;\langle\rangle, \bot_\boolt := \kappa_2\;\langle\rangle$. Définir les opérations booléennes usuelles $\lnot,\land,\lor$ et vérifier qu'elles se comportent comme attendu. Soit $\tau \in T_{\to\times1+0}$, définir une fonction $\ifthenelsee{-}{-}{-} : \boolt\to\tau\to\tau\to\tau$ telle que \begin{align*}
        \ifthenelsee{\top_\boolt}{M}{N} &\beteq M\\
        \ifthenelsee{\bot_\boolt}{M}{N} &\beteq N
    \end{align*}
\end{exo}


\section{Le lambda-calcul est Turing-complet}

Le but de cette section est de montrer que l'on peut simuler les fonctions récursives avec des lambda-termes. Nous commencerons par montrer comment coder différentes structures de données élémentaires, puis nous donnerons le théorème de Turing-complétude du $\lambda$-calcul.

\subsection{Entiers de Church}

La première étape est de définir un codage pour les entiers, c'est-à-dire que l'on va associer à chaque entier $n\in\nat$ un lambda-terme $\underline n$ correspondant. Ces lambda-termes s'appellent les entiers de Church.

\begin{defi}[Entiers de Church]
    Soit $n\in\nat$, on définit le codage $\underline n$ de $n$ par $$\underline n := \lambda f.\lambda x. f^n\;x$$ où $f^0\; x = x$ et $f^{n+1}\;x = f\;(f^n\;x)$.
\end{defi}

On sait que les données importantes sont en fait les constructeurs $0$ et $S$, pour définir $\nat$, et nous allons donc donner une présentation des entiers de Church n'utilisant que ces deux constructeurs.

\begin{prop}
    Soient $\underline 0 := \lambda f.\lambda x.x$ et $\underline S := \lambda n.\lambda f.\lambda x.f\;(n\;f\;x)$ alors $\forall n\in\nat, \underline n \beteq \underline S^n(\underline 0)$
\end{prop}

\begin{proof}
    On raisonne par récurrence :
    \begin{itemize}[label=$\bullet$]
        \item $\underline 0 = \underline S^0 (\underline 0)$ par définition de $\underline S^0$.
        \item On suppose que $\underline n =_\beta \underline S^n(\underline 0)$, alors :
        \begin{align*}
            \underline{n+1} &\beteq \lambda f.\lambda x.f^{n+1}\;x\\
            &\beteq \lambda f.\lambda x.f\;(f^n\;x)\\
            \underline S^{n+1}(\underline 0) &= \underline S(\underline S^{n}(\underline 0))\\
            &\beteq (\lambda n.\lambda f.\lambda x.f\;(n\;f\;x))(\underline n)\\
            &\reduc \lambda f.\lambda x.f\;(\underline n\;f\;x)\\
            &\reduc \lambda f.\lambda x.f\;(f^n\;x)
        \end{align*}
    \end{itemize}

    D'où le résultat par récurrence.
\end{proof}

On en déduit le résultat suivant :

\begin{cor}
    Soit $P$ un prédicat sur $\Lambda$ stable par $\beteq$, c'est-à-dire tel que $\forall M\in\Lambda, (P(M)\land (M\beteq N))\implies P(N)$. Si $P(\underline 0)$ et $\forall n\in\nat, P(\underline n)\implies P(\underline S\;\underline n)$ alors $\forall n\in\nat, P(\underline n)$.
\end{cor}

\begin{proof}
    On montre que l'ensemble $\{M\mid P(M)\}$ contient $\underline\nat = \{\underline n\mid n\in\nat\}$. Supposons qu'il existe un $n\in\nat$ tel que $\lnot P(\underline n)$, alors on peut considérer un $n$ minimal respectant cette condition : comme $n\neq 0$ on en déduit que $P(\underline{n-1})$ donc $P(\underline S\;\underline{n-1})$ mais $\underline S\;\underline{n-1} \beteq \underline S^{1+(n-1)}\;\underline 0 \beteq \underline n$ donc $P(\underline n)$, ce qui est absurde.
\end{proof}

\begin{rmk}
    On ne vérifiera pas, en général, que $P$ est stable par $\beteq$ car ce sera évident (les propositions dans cette partie utiliseront comme relation atomique $\beteq$). Dans ce document, on fera référence à ce raisonnement comme la $\lambda$-récurrence.
\end{rmk}

L'intuition derrière le codage de $n$ est de considérer la fonction d'ordre supérieure qui, étant donnée une fonction $f$, retourne la fonction qui itère $n$ fois $f$. Cette intuition va nous guider pour définir facilement les opérations $+$ et $\times$.

\begin{defi}[Addition]
    On définit le terme $\mathrm{add}$ par $$\mathrm{add} := \lambda n.\lambda m. n\;\underline S\;m$$ et ce terme vérifie $$\forall n\in\nat,\forall m\in\nat, \mathrm{add}\;\underline n\;\underline m \beteq \underline{n+m}$$
\end{defi}

\begin{proof}
    Il nous suffit de prouver cette propriété par $\lambda$-récurrence sur $n$ avec $m$ quelconque fixé avant :
    \begin{itemize}[label=$\bullet$]
        \item Tout d'abord : 
        \begin{align*}
            \mathrm{add}\;\underline 0\;\underline n &\beteq \underline 0\;\underline S\;\underline n\\
            &\beteq \underline S^0\;\underline n\\
            &\beteq \underline n = \underline{0+n}
        \end{align*}
        \item Si l'on suppose que $\mathrm{add}\;\underline n\;\underline m \beteq \underline{n+m}$ alors :
        \begin{align*}
            \mathrm{add}\;\underline {n+1}\;\underline m &\beteq \underline{n+1} \;\underline S\;\underline m\\
            &\beteq \underline S\;(\underline S^n\;(\underline m)\\
            &\beteq \underline S\;(\underline n\;\underline S\;\underline m)\\
            &\beteq \underline S\;(\mathrm{add}\;\underline n\;\underline m)\\
            &\beteq \underline S(\underline{n+m})\\
            &\beteq \underline{n+m+1}\\
            &\beteq \underline{(n+1)+m}
        \end{align*}
    \end{itemize}

    D'où le résultat par récurrence.
\end{proof}

\begin{exo}
    Montrer que le terme $\mathrm{add} := \lambda n.\lambda m. \lambda f.\lambda x. n\;f\;(m\;f\;x)$ vérifie la même propriété et qu'il est donc un choix possible pour l'addition.
\end{exo}

Pour alléger les notations, on notera maintenant $n+m$ pour $\mathrm{add}\;n\;m$. De façon similaire on peut définir la multiplication.

\begin{defi}[Multiplication]
    On définit le terme $\mathrm{mult}$ par $$\mathrm{mult} := \lambda n.\lambda m. n\;(\mathrm{add}\;m)\;\underline 0$$ et ce terme vérifie $$\forall n\in\nat,\forall m\in\nat, \mathrm{mult}\;\underline n\;\underline m \beteq \underline{n\times m}$$
\end{defi}

\begin{proof}
    Par $\lambda$-récurrence sur $n$ :
    \begin{itemize}[label=$\bullet$]
        \item Si $n = 0$ :
        \begin{align*}
            \mathrm{mult} \;\underline 0\;\underline m &\beteq \underline 0 \;(\mathrm{add}\;\underline m)\;\underline 0\\
            &\beteq \underline 0 = \underline{0\times m}
        \end{align*}
        \item Si l'on suppose que $\mathrm{mult}\;\underline n\;\underline m \beteq \underline{n\times m}$ alors :
        \begin{align*}
            \mathrm{mult}\;\underline{n+1}\;\underline m &\beteq \underline{n+1} \;(\mathrm{add}\;\underline m)\;\underline 0\\
            &\beteq \mathrm{add}\;\underline m\;(\underline n\;(\mathrm{add}\;\underline m)\;\underline 0\\
            &\beteq \mathrm{add}\;\underline m\;(\mathrm{mult}\;\underline n\;\underline m)\\
            &\beteq \mathrm{add}\;\underline m\;\underline{n\times m}\\
            &\beteq \underline{n\times m+m}\\
            &\beteq \underline{(n+1)\times m}
        \end{align*}
    \end{itemize}

    D'où le résultat par récurrence.
\end{proof}

On notera $n\times m$ pour $\mathrm{mult}\;n\;m$.

\begin{exo}
    Montrer que $$\forall n\in\nat,\forall m\in\nat,\underline n\;\underline m \beteq \underline{m^n}$$ et en déduire une traduction de la fonction $\fonction{\mathrm{exp}}{\nat\times\nat}{\nat}{(n,m)}{n^m}$ en lambda-terme.
\end{exo}

\subsection{Structure de données}

Nous allons nous intéresser maintenant au codage des tuples. Remarquons d'abord qu'il est suffisant de coder des paires : si l'on sait coder $\langle x,y\rangle$ alors on peut coder $\langle x_1,\ldots,x_n\rangle$ par $\langle\cdots\langle x_1,x_2\rangle,x_3\rangle,\ldots,x_n\rangle$ voire par $\langle\cdot\langle \bullet,x_1\rangle,x_2\rangle,\ldots,x_n\rangle$, où $\bullet$ est un lambda-terme quelconque, qui donne une présentation plus systématique des projections. Commençons donc par définir le codage des paires et des projections.

\begin{defi}[Paires]
    Soient $M,N\in\Lambda$, on définit le lambda-terme $$\langle M,N\rangle := \lambda p.p\;M\;N$$ et les lambda-termes $$\pi_1 := \lambda p.(p\;(\lambda x.\lambda y.x))\qquad\pi_2 := \lambda p.(p\;(\lambda x.\lambda y.y))$$
\end{defi}

\begin{prop}
    Pour tous termes $M,N\in\Lambda$, on a $\pi_1\;\langle M,N\rangle\beteq M$ et $\pi_2\;\langle M,N\rangle \beteq N$
\end{prop}

\begin{proof}
    Prouvons l'identité pour $\pi_1$, l'autre cas étant identique :
    \begin{align*}
        \pi_1\;\langle M,N\rangle &\beteq (\lambda p.p\;(\lambda x.\lambda y.x))(\lambda p.p\;M\;N)\\
        &\beteq (\lambda p.p\;M\;N)(\lambda x.\lambda y.x)\\
        &\beteq (\lambda x.\lambda y.x)\;M\;N\\
        &\beteq M
    \end{align*}
\end{proof}

On peut alors définir des tuples, et leurs projections associées.

\begin{defi}[Tuple]
    Soient $M_1,\ldots,M_n$ des lambda-termes, on définit $(M_1,\ldots,M_n)$ par induction sur $n\in\nat^*$ :
    \begin{itemize}[label=$\bullet$]
        \item $(M_1) := \langle \lambda x.x,M_1\rangle$
        \item $(M_1,\ldots,M_{n+1}) := \langle (M_1,\ldots,M_n),M_{n+1}\rangle$
    \end{itemize}

    On définit de plus les projections $\pi_{n,i}, i\leq n$ par induction sur $n-i$ :
    \begin{itemize}[label=$\bullet$]
        \item $\pi_{n,n} := \pi_2$
        \item $\pi_{n,i} := \pi_{n-1,i}\circ \pi_1$
    \end{itemize}
\end{defi}

\begin{prop}
    Soient $M_1,\ldots,M_n$ des lambda-termes, alors $$\forall i\in \{1,\ldots,n\}, \pi_{n,i}\;(M_1,\ldots,M_n) \beteq M_i$$
\end{prop}

\begin{proof}
    On prouve ce résultat pour tout $i\leq n$ et $M_1,\ldots,M_n\in\Lambda$ par récurrence sur $n\in\nat^*$ :
    \begin{itemize}[label=$\bullet$]
        \item Si $n = 1$ alors $(M_1) = \langle I,M_1\rangle$ et $\pi_{1,1} = \pi_2$, ce qui avec la proposition précédente nous donne bien le résultat.
        \item Si pour tout $i\leq n$, l'équation est vraie, alors soit $M_1,\ldots,M_{n+1}\in\Lambda$. Soit $i\leq n+1$. Si $i = n+1$ alors $$\pi_{n+1,i}\;(M_1,\ldots,M_{n+1}) = \pi_2\;\langle (M_1,\ldots,M_n),M_{n+1}\rangle \beteq M_{n+1}$$ Si $i \leq n$ alors 
        \begin{align*}
            \pi_{n+1,i}\;(M_1,\ldots,M_{n+1}) &= (\pi_{n,i}\circ \pi_1)\;(M_1,\ldots,M_{n+1})\\
            &\beteq \pi_{n,i}\;(\pi_1\;\langle (M_1,\ldots,M_n),M_{n+1}\rangle)\\
            &\beteq \pi_{n,i}\;(M_1,\ldots,M_n)\\
            &\beteq M_i
        \end{align*}
    \end{itemize}

    D'où le résultat.
\end{proof}

La structure de tuple sera utile pour le résultat théorique d'équivalence du lambda-calcul avec les fonctions récursives, mais elle donne aussi une idée simple pour définir les booléens et les conditions : dire \og si $b$ alors $M$ sinon $N$\fg{} revient à considérer $\langle M,N\rangle$ et identifier les projections avec les valeurs de vérité.

\begin{defi}[Booléen]
    On définit deux lambda-termes $$\top := \lambda x.\lambda y.x\qquad \bot := \lambda x.\lambda y.y$$
\end{defi}

\begin{exo}
    Montrer que la lambda-terme $\lnot := \lambda b. b\;\bot\;\top$ vérifie $\lnot\;\top \beteq \bot$ et $\lnot\;\bot \beteq \top$.
\end{exo}

\begin{exo}
    Construire deux lambda-termes, respectivement $\lor$ et $\land$ représentant les opérations booléennes de disjonction et de conjonction. Montrer qu'elles respectent les tables de vérité attendues.
\end{exo}

\begin{defi}[Condition]
    Soient $M,N,b\in\Lambda$, on définit $$\ifthenelsee{b}{M}{N} := b\;M\;N$$
\end{defi}

On peut vérifier de façon évidente que les conditions se comportent comme attendu : on a bien $\ifthenelsee{\top}{M}{N} \beteq M$ et $\ifthenelsee{\bot}{M}{N} \beteq N$. Enfin, pour que les booléens puissent exprimer une condition pertinente, il faut pouvoir trouver au moins une fonction qui associe un booléen à un entier.

\begin{defi}[\'Egalité à $0$]
    On définit le lambda-terme suivant : $$\eqz := \lambda n. n\;(\lambda x.\bot)\;\top$$ Ce terme vérifie $\eqz\;\underline 0 \beteq \top$ et $\eqz\;(\underline S\;\underline n) \beteq \bot$.
\end{defi}

\begin{proof}
    Il suffit de faire le calcul :
    \begin{align*}
        \eqz\;\underline 0 &\beteq \underline 0 \;(\lambda x.\bot)\;\top\\
        &\beteq \top\\
        \eqz\;(\underline S\;\underline n) &\beteq (\underline S\;\underline n)\;(\lambda x.\bot)\;\top\\
        &\beteq (\lambda x.\bot)(\underline n\;(\lambda x.\bot)\;\top)\\
        &\beteq \bot
    \end{align*}
\end{proof}

\begin{exo}[L'opération prédécesseur]
    L'objectif de cet exercice est de trouver un terme $\mathrm{pred}$ tel que $\pred(\underline 0)\beteq \underline 0$ et $\pred(\underline S\;\underline n) \beteq \underline n$.
    \begin{enumerate}
        \item Construire un lambda-terme $f$ tel que $f\;\langle M,\underline n\rangle \beteq \langle \underline n,\underline{n+1}\rangle$
        \item Montrer que $\underline n \;f\;\langle \underline 0,\underline 0\rangle \beteq \langle \underline{n-1},\underline{n}\rangle$
        \item En déduire un lambda-terme $\pred$ qui vérifie les conditions de l'énoncé et montrer que ces conditions sont bien vérifies.
    \end{enumerate}
\end{exo}

\begin{defi}[Soustraction]
    On définit le lambda-terme $$\min := \lambda n.\lambda m. m\;\pred\;n$$ et ce lambda-terme vérifie, pour tous $n,m\in\nat$, si $n\leq m$ alors $\min\;\underline n\;\underline m \beteq \underline 0$ et sinon $\min\;\underline n\;\underline m = \underline{n-m}$.
\end{defi}

\begin{proof}
    Montrons d'abord, par $\lambda$-récurrence, que $\forall n,m\in\nat,\min\;\underline {m+n}\;\underline n \beteq \underline m$ :
    \begin{itemize}[label=$\bullet$]
        \item Par définition, $\underline 0 \;\pred \beteq I$ donc $\min\; \underline {m+0}\;\underline 0 \beteq \underline m$.
        \item Si $\min\;\underline {m+n}\;\underline n \beteq \underline m$ alors
        \begin{align*}
            \min\;\underline{m+n+1}\;\underline{n+1} &\beteq \min\;(\underline S\;\underline n)\;(\underline S^{m+n+1}\;\underline 0)\\
            &\beteq (\underline S^{m+n+1}\;\pred)\;(\underline S\;\underline n)\\
            &\beteq (\underline S^{m+n}\;\pred)\;(\pred\;(\underline S\;\underline n))\\
            &\beteq \underline n\;\pred\;\underline {m+n}\\
            &\beteq \min\;\underline {m+n}\;\underline n\\
            &\beteq \underline m
        \end{align*}
    \end{itemize}

    De plus, montrons que $\min\;\underline n\;\underline{m+p} \beteq \min (\min\;\underline n\;\underline m)\; \underline p$ par récurrence sur $p$ :
    \begin{itemize}[label=$\bullet$]
        \item D'abord $\underline{m+0}=\underline m$ et pour tout $M\in\Lambda, \min\;M\;\underline 0 \beteq M$ donc le résultat est vrai pour $p = 0$.
        \item Si pour tout $n,m\in\nat, \min\;\underline n\;\underline{m+p} \beteq \min (\min\;\underline n\;\underline m)\; \underline p$ alors pour $p+1$ on a :
        \begin{align*}
            \min\;\underline n\;\underline{m+p+1} &\beteq \pred^{m+p+1}\;\underline n\\
            &\beteq \pred\;(\min\;(\min\;\underline n\;\underline m)\;\underline p)\\
            &\beteq \min\;(\min\;\underline n\;\underline m)\;\underline{p+1}
        \end{align*}
    \end{itemize}

    En combinant les deux résultats, si $n \leq m$, en écrivant $m = n+(m-n)$, on en déduit que $$\min\;\underline n\;\underline m \beteq \min\;(\min\;\underline n\;\underline n)\;\underline{m-n} \beteq \min \;\underline 0\; \underline{m-n}$$ et $\min \;\underline 0\;\underline n \beteq \underline 0$ pour tout $n\in\nat$ puisque c'est un point fixe de la fonction $\pred$.

    Enfin, si $m\leq n$ alors on peut écrire $n = m+(n-m)$ donc $$\min\;\underline n\;\underline m = \min \underline{(n-m)+m}\;\underline m \beteq \underline{n-m}$$
\end{proof}

On notera par souci de lisibilité $n-m$ pour le terme $\min\;n\;m$ mais il faudra faire attention à bien considérer que cette soustraction est dans $\nat$. De plus on notera $n-\underline 1$ plutôt que $\pred\;n$ en général.

\begin{exo}[Inégalité]
    Construire un lambda-terme $\mathrm{ineq}$ tel que pour tous $n,m\in\nat,\mathrm{ineq}\;\underline n\;\underline m \beteq \top$ si $n\leq m$ et $\mathrm{ineq}\;\underline n\;\underline m \beteq \bot$ sinon.
\end{exo}

\begin{exo}[\'Egalité]
    Constuire un lambda-terme $\mathrm{eq}$ tel que $\mathrm{eq}\;\underline n\;\underline n \beteq \top$ et $\mathrm{eq}\;\underline n\;\underline m\beteq \bot$ pour $n\neq m$ deux entiers quelconques, et montrer que ce lambda-terme convient bien.
\end{exo}

On notera $n==m$ pour $\mathrm{eq}\;n\;m$.

\subsection{Point fixe et récursion}

L'élément qui nous manque maintenant est de pouvoir faire des appels récursifs. Nous avons vu pour les fonctions récursives la récursion bornée, qui s'obtient en appliquant une fonction un certain nombre de fois sur un argument donné, moralement en construisant une fonction de la forme $n\mapsto f^n(x)$ pour une certaine fonction $f$ et un certain $x$. En lambda-calcul, la récursion sera non bornée : on définit simplement une fonction $f$ dans laquelle on peut appliquer $f$ directement. Utilisons un exemple classique de fonction récursive avec la fonction factorielle : $$\mathrm{fact} := \lambda n.\ifthenelsee{n == \underline 0}{\underline 1}{n\times \mathrm{fact}\;(n-\underline 1)}$$

Cette définition n'est valide qu'en considérant que fact est une variable libre dans ce terme, ce qui n'est pas notre objectif. Pour régler le souci et permettre de faire une définition inductive, il va nous falloir monter d'un cran en abstraction, en considérant au lieu de la fonction factorielle, une fonction qui va \og factorialiser\fg{} une fonction quelconque : prenant une fonction $f$, elle définira une fonction qui fera une étape de calcul de fact avant d'appliquer potentiellement $f$. On va la définir par $$g := \lambda f.\lambda n.\ifthenelsee{n == \underline 0}{\underline 1}{n\times f\;(n-\underline 1)}$$

Le point essentiel est alors le suivant : la fonction fact que l'on veut est exactement un point fixe de $g$. En effet, c'est une fonction telle que la factorialiser ne changera pas son comportement, car elle est déjà la factorielle. Imaginons qu'on trouve $f$ telle que $g\;f \beteq f$, alors $f$ se comportera bien comme la factorielle :
\begin{align*}
    f\;\underline n&\beteq g\;f\;\underline n\\
    &\beteq \ifthenelsee{\underline n == \underline 0}{\underline 1}{\underline n \times f\;(\underline n - \underline 1)}
\end{align*}

Ce qui nous donne bien le comportement attendu. Justement, il se trouve que tout lambda-terme possède un point fixe, et c'est ce que nous allons prouver.

\begin{defi}[Combinateur de point fixe]
    On définit le lambda-terme $$Y := \lambda f.(\lambda x.f\;(x\;x))(\lambda x.f\;(x\;x))$$ et le lambda-terme $$\Theta := (\lambda y.\lambda x.y\;(x\;x\;y))(\lambda y.\lambda x.y\;(x\;x\;y))$$

    Ces termes sont des combinateurs de points fixes, ce qui signifie que pour tout lambda-terme $M$, on a $Y\;M\beteq M\;(Y\;M)$ et $\Theta\;M \beteq M\;(\Theta\;M)$
\end{defi}

\begin{proof}
    Il nous suffit de faire le calcul :
    \begin{align*}
        Y\;M &\reduc (\lambda x.M\;(x\;x))(\lambda x.M\;(x\;x))\\
        &\reduc M\;((\lambda x.M\;(x\;x))(\lambda x.M\;(x\;x)))\\
        &\beteq M\;(Y\;M)\\
        \Theta\;M &\reduc (\lambda x.M\;(x\;x\;M))(\lambda y.\lambda x.y\;(x\;x\;y))\\
        &\reduc M\;((\lambda y.\lambda x.y\;(x\;x\;y))\;(\lambda y.\lambda x.y\;(x\;x\;y))\;M)\\
        &= M\;(\Theta\;M)
    \end{align*}
\end{proof}

\begin{rmk}
    Le combinateur $\Theta$ donne un résultat plus fort : le point fixe $\Theta\;M$ l'est pour la réduction elle-même et non simplement la $\beta$-équivalence.
\end{rmk}

On peut donc définir notre fonction fact : $$\mathrm{fact} := Y(\lambda f.\lambda n.\ifthenelsee{n==\underline 0}{\underline 1}{n\times f\;(n-\underline 1)})$$

\begin{exo}
    Montrer que $$\forall n\in\nat, \mathrm{fact}\;\underline n \beteq \underline{n!}$$
\end{exo}

\subsection{Fonction récursive en lambda-terme}

Nous pouvons maintenant nous atteler à montrer que toute fonction récursive peut être simulée par un lambda-terme.

\begin{defi}[Simulation]
    Soit $f : \nat^k\to\nat$, on dit que $f$ est simulée par le lambda-terme $\underline f$ si la propriété suivante est vérifiée : $$\forall n_1,\ldots,n_k\in\nat, \underline f\;(\underline{n_1},\ldots,\underline{n_k}) \beteq \underline{f(n_1,\ldots,n_k)}$$
\end{defi}

Pour montrer que les fonctions récursives sont simulables, il suffit de montrer les résultats de stabilité qu'on a déjà pu voir par exemple dans le chapitre sur l'arithmétique de Peano.

\begin{prop}
    Les fonctions constantes, les projections et la fonction successeur sont simulables, respectivement par $\lambda x.\underline n$ pour la fonction constante $n$, $\pi_{k,i}$ pour la $i$ème projection de $\nat^k\to\nat$, et par $\underline S$ pour la fonction successeur.
\end{prop}

\begin{proof}
    On a déjà prouvé ces propriétés en amont.
\end{proof}

\begin{prop}
    Si $f_1,\ldots,f_k : \nat^m\to\nat$ sont des fonctions simulées respectivement par $\underline{f_1},\ldots,\underline{f_k}$ et $h : \nat^k\to\nat$ est simulée par $\underline h$, alors la fonction $\underline k := \lambda x.\underline h\;(\underline{f_1}\;x,\ldots,\underline{f_k}\;x)$ simule la fonction $$\fonction{k}{\nat^m}{\nat}{n_1,\ldots,n_m}{ h(f_1(n_1,\ldots,n_m),\ldots,f_k(n_1,\ldots,n_m))}$$
\end{prop}

\begin{proof}
    Faisons le calcul :
    \begin{align*}
        \underline k\;(\underline{n_1},\ldots,\underline{n_m}) &\beteq \underline h\;(\underline{f_1}\;(\underline{n_1},\ldots,\underline{n_m}),\ldots,\underline{f_k}\;(\underline{n_1},\ldots,\underline{n_m}))\\
        &\beteq \underline h\;(\underline{f_1(n_1,\ldots,n_m)},\ldots,\underline{f_k(n_1,\ldots,n_m)})\\
        &\beteq \underline{h(f_1(n_1,\ldots,n_m),\ldots,f_k(n_1,\ldots,n_m))}\\
        &\beteq \underline{k(n_1,\ldots,n_m)}
    \end{align*}

    Respectivement car $\underline{f_i}$ simule $f_i$ et car $\underline h$ simule $h$.
\end{proof}

\begin{prop}[Récursion primitive]
    Si $f : \nat^k\to\nat$ est simulée par $\underline f$, $g : \nat^{k+2}\to\nat$ est simulée par $\underline g$ alors la fonction $\rec(f,g) : \nat^{k+1}\to\nat$ telle que définie dans le chapitre des fonctions récursives est simulée par la fonction $$\underline{\rec(f,g)} := \Theta(\lambda h.\lambda x.\ifthenelsee{\!\pi_{k+1,k+1} == \underline 0}{\!\underline f\;x}{\!\underline g\;\langle \langle\pi_1\;x,\pred\;(\pi_2\;x)\rangle,h\;\langle \pi_1\;x,\pred\;(\pi_2\;x)\rangle\rangle}) $$
\end{prop}

\begin{proof}
    On vérifie que la fonction ainsi définie représente bien $\rec(f,g)$ par $\lambda$-récurrence sur son dernier argument :
    \begin{align*}
        \underline{\rec(f,g)}(\underline{n_1},\ldots,\underline{n_k},\underline 0) &= \ifthenelsee{\pi_{k+1,k+1}\;(\underline{n_1},\ldots,\underline{n_k},\underline 0) == \underline 0}{\underline f\;(\underline{n_1},\ldots,\underline{n_k})}{[\ldots]}\\
        &\beteq \ifthenelsee{\underline 0 == \underline 0}{\underline f\;(\underline{n_1},\ldots,\underline{n_k})}{[\ldots]}\\
        &\beteq \underline f\;(\underline{n_1},\ldots,\underline{n_k})\\
        &\beteq \underline{f(n_1,\ldots,n_k)}\\
        \underline{\rec(f,g)}(\underline{n_1},\ldots,\underline{n_k},\underline S\;\underline n) &\beteq \underline g\;\langle \langle(\underline{n_1},\ldots,\underline{n_k}),\pred\;(\underline S\;\underline n)\rangle,\underline{\rec(f,g)}\;\langle(\underline{n_1},\ldots,\underline{n_k}),\pred\;(\underline S\;\underline n)\rangle\\
        &\beteq \underline g(\underline{n_1},\ldots,\underline{n_k},\underline n,\underline{\rec(f,g)}\;(\underline{n_1},\ldots,\underline{n_k},\underline n))\\
        &\beteq \underline g(\underline{n_1},\ldots,\underline{n_k},\underline n,\underline{\rec(f,g)\;(n_1,\ldots,n_k,n)})\\
        &\beteq \underline {g(n_1,\ldots,n_k,n,\rec(f,g)(n_1,\ldots,n_k,n))}
    \end{align*}
\end{proof}

\begin{prop}[Schéma $\mu$]
    Soit $f : \nat^{k+1}\to\nat$ simulée par $\underline f$, alors la fonction $g : x\mapsto \mu(x,f)$ associant à un tuple $x$ le plus petit $y$ tel que $f(x,y) = 0$ s'il existe, est simulée par la fonction $\underline g := \lambda x.\Theta(\lambda h.\lambda n.\ifthenelsee{f\;\langle x,n \rangle == \underline 0}{n}{h\;\langle x,\underline S\;\underline n \rangle})\;\underline 0$
\end{prop}

\begin{proof}
    Soient $n_1,\ldots,n_k\in\nat$, on suppose qu'il existe $y$ (que l'on prend minimal) tel que $f(n_1,\ldots,n_k,y) = 0$. On remarque que pour tout $n < y$, on a pour une fonction $h$ quelconque $$(\lambda n.\ifthenelsee{\underline f\;\langle (\underline{n_1},\ldots,\underline{n_k}),n \rangle == \underline 0}{n}{h\;\langle (\underline{n_1},\ldots,\underline{n_k}),\underline S\;\underline n \rangle})\;\underline n \beteq h\;(\underline{n_1},\ldots,\underline{n_k},\underline{n+1})$$ car par hypothèse de minimalité de $y, f(n_1,\ldots,n_k,n) \neq 0$ et $\underline f$ simule $f$, forçant la réduction à entrer dans le cas \textit{else}. De plus, $\underline y$ est un point fixe de cette fonction. Notons $$ h := \Theta(\lambda h.\lambda n.\ifthenelsee{f\;\langle x,n \rangle == \underline 0}{n}{h\;\langle x,\underline S\;n \rangle})$$ pour $x = (\underline{n_1},\ldots,\underline{n_k})$. On montre par récurrence finie sur $y-n$ que $h\;\underline i \beteq \underline y$ :
    \begin{itemize}[label=$\bullet$]
        \item Par l'argument précédent, $h\;y\beteq y$
        \item Supposons que $h\;i\beteq y$, alors 
        \begin{align*}
            h\;(\underline{i-1}) &\beteq \ifthenelsee{f\;\langle x,\underline{i-1} \rangle == \underline 0}{n}{h\;\langle x,\underline S\;\underline {i-1} \rangle}\\
            &\beteq h\;\langle x,\underline S\;\underline{i-1}\rangle\\
            &\beteq h\;(x,\underline i)\\
            &\beteq \underline y
        \end{align*}
    \end{itemize}

    Donc pour $i = 0$ cela nous donne $h\;\underline 0 \beteq \underline y$, soit $$\underline g\;(\underline{n_1},\ldots,\underline{n_k}) \beteq \underline{g(n_1,\ldots,n_k)}$$
\end{proof}

\begin{them}[Simulation]
    Pour toute fonction récursive $f : \nat^k\to\nat$ il existe un lambda-terme $\underline f$ tel que $$\forall n_1\in\nat,\ldots,\forall n_k\in\nat, \underline f\;(\underline{n_1},\ldots,\underline{n_k})\beteq \underline{f(n_1,\ldots,n_k)}$$
\end{them}

\begin{proof}
    En utilisant les propositions précédentes, on sait que les fonctions récursives atomiques sont simulables, et que la récursion, la composée et le schéma $\mu$ de fonctions simulables sont simulables, donc toutes les fonctions récursives sont simulables.
\end{proof}

Pour le sens réciproque, comme nous avons montré que l'on peut simuler les machines de Turing par des fonctions récursives, il suffit de montrer que l'on peut simuler le lambda-calcul par une machine de Turing. Nous ne le ferons pas, mais les constructions sur les indices de De Bruijn permettent par exemple de définir de façon purement algorithmique un interpréteur de lambda-calcul. Nous proposons un codage possible $\overline M$ du lambda-terme $M\in\Lambda_B$ sur l'alphabet $\Gamma := \{0,1,2,3\}$ :
\begin{itemize}[label=$\bullet$]
    \item $\overline n = 3^{n+1}$
    \item $\overline{\lambda M} = 0\cdot \overline M$
    \item $\overline{(M\;N)} = 1\cdot\overline M\cdot2\cdot\overline N$
\end{itemize}

\section{Propriétés du lambda-calcul}

Dans cette section, nous allons nous concentrer sur les propriétés du lambda-calcul en tant que système de réécriture. Nous chercherons à décrire les classes d'égalité $\Lambda_\beta := \quot{\Lambda}{\beteq}$ dans un premier temps, en parlant du théorème de Church-Rosser, puis nous parlerons de la règle $\eta$ et de l'extensionalité.

\subsection{Propriété de Chuch-Rosser}

Le résultat essentiel pour décrire $\Lambda_\beta$ est le théorème de Church-Rosser, mais nous allons commencer par donner le formalisme nécessaire à bien appréhender cette notion.

\begin{defi}[Système de réécriture]
    On appelle ici système de réécriture un couple $(E,\to)$ où $\to\subseteq E\times E$.
\end{defi}

Le but ici sera évidemment de considérer le système de réécriture $(\Lambda,\reduc)$. Plus précisément, on veut montrer la propriété de confluence de ce système de réécriture :

\begin{defi}[Confluence, confluence locale]
    On dit qu'un système de réécriture $(E,\to)$ est confluent si pour tous $x,y,z\in E$, tels que $y\;^*\!\!\leftarrow x \rightarrow^* z$ il existe $a\in E$ tel que $y\to^* a$ et $z\to^* a$.

    On dit qu'il est localement si pour tous $x,y,z\in E$ tels que $y\leftarrow x \rightarrow z$, il existe $a\in E$ tel que $y\to^* a$ et $z\to^* a$.

    On dit qu'il a la propriété du diamant si pour tous $x,y,z\in E$ tels que $y\leftarrow x \rightarrow z$ il existe $a\in E$ tel que $y\rightarrow a \leftarrow z$.
\end{defi}

\includefig{Curry-Howard/lambda_calc_nt/confluence.tex}{Illustration des différentes propriétés}

Une propriété équivalente à la confluence, dont la formulation nous intéresse plus, est celle de Church-Rosser :

\begin{defi}[Church-Rosser]
    On dit qu'un système de réécriture $(E,\to)$ a la propriété de Church-Rosser si pour tous $x,y\in E$ tels que $x\leftrightarrow^* y$ il existe $z\in E$ tel que $x\to^* z$ et $y\to^* z$ où $\leftrightarrow^*$ est la plus petite relation d'équivalence contenant $\to$.
\end{defi}

\begin{prop}
    $(E,\to)$ a la propriété de Church-Rosser si et seulement si c'est un système confluent.
\end{prop}

\begin{proof}
    Si $(E,\to)$ a la propriété de Church-Rosser, alors pour $y\;^*\!\!\leftarrow x \rightarrow^* z$ on peut directement déduire que $y\leftrightarrow^* z$ donc on trouve $a\in E$ tel que $y\to^* a$ et $z\to^* a$.

    Réciproquement, si $(E,\to)$ est confluent, soient $x,y\in E$ tels que $x\leftrightarrow^* y$. Par définition de $\leftrightarrow^*$, on trouve une suite $(x_i)_{i=1,\ldots,n}$ finie telle que $x_0 = x, x_n = y$ et pour tout $i = 1,\ldots, n-1$, $x_i\to^*x_{i+1}$ ou $x_{i+1}\to^* x_i$. On va montrer par récurrence sur ce $n$ (car toute telle suite est finie) qu'il existe $a\in E$ tel que $x\to^* a$ et $y\to^* a$ :
    \begin{itemize}[label=$\bullet$]
        \item si la suite est $(x)$, alors on a $x\;^*\!\!\leftarrow x \rightarrow^* x$ d'où l'existence de $a$ par confluence.
        \item supposons que pour toute suite de taille $n$ telle que décrite plus haut, il existe un élément $a$ tel que $x\to^* a$ et $y\to^* a$ pour $x$ et $y$ les deux extrémités de la suite. Soit $z$ tel que $y\to^* z$, alors cela signifie que $a\;^*\!\!\leftarrow y \rightarrow^* z$ donc on trouve par confluence $b\in E$ tel que $a\to^* b$ et $z\to^* b$, et comme $x\to^* a$ la transitivité nous donne $x\to^*b$. Soit $z$ tel que $z\to^* y$, on peut donc écrire $a\;^*\!\!\leftarrow z \rightarrow^* z$ par transitivite de $z\to^* y$ et $y\to^* a$, donc on trouve par confluence $b$ tel que $z\to^* b$ et $a\to^* b$, ce qui comme précédemment permet de conclure. Donc pour toute suite $(x_i)_{i=1,\ldots,n+1}$ il existe un terme $a\in E$ tel que $x_1\to^* a$ et $x_{n+1}\to^* a$.
    \end{itemize}
    On a donc trouvé pour $x\leftrightarrow^* y$ un $a\in E$ tel que $x\to^* a$ et $y\to^* a$.
\end{proof}

Enfin, la propriété du diamant est plus forte que la confluence :

\begin{prop}
    Si $(E,\to)$ a la propriété du diamant, alors $(E,\to)$ est confluente.
\end{prop}

\begin{proof}
    \'Etant donnés $x,y,z\in E$ tels que $x\to^* y$ et $x\to^* z$, on peut trouver deux chemins $(y_i)_{i=0,\ldots,n}$ et $(z_i)_{i=0,\ldots,m}$ respectivement de $x$ à $y$ et de $x$ à $z$ avec $y_i\to y_{i+1}$ et $z_i\to z_{i+1}$. On montre par récurrence sur $\max(n,m)$ qu'il existe $a\in E$ tel que $y\to^m a$ et $z\to^n a$ :
    \begin{itemize}[label=$\bullet$]
        \item Si $\max(n,m) = 0$ alors $y=x=z$. En prenant $a = x$ on a le résultat.
        \item Si $n = m = 1$ alors la propriété du diamant nous donne le résultat.
        \item Si $n = 0$ ou $m = 0$, en prenant respectivement $a = z$ et $a = y$ la propriété est vérifiée.
        \item Supposons que pour tous $x,y,z$ tels que $x\to^n y$ et $x\to^m z$, si $\max(n,m) \leq k$ alors il existe $a\in E$ tel que $y\to^m a$ et $z\to^n a$. Soient $x,y,z$ tels que $x\to^n y$ et $x\to^m z$ avec $\max(n,m) = k+1$ et $\min(n,m) \geq 1$. Soient $y' = y_1$ et $z'=z_1$, par propriété du diamant on trouve $b\in E$ tel que $y'\to b$ et $z'\to b$. On sait donc que $y'\to^{m-1} b$ et $z'\to^{n-1} b$, donc on peut appliquer l'hypothèse d'induction à $(y',b,y)$ et $(z',b,z)$ pour trouver respectivement $c\in E$ et $d\in E$ tels que $b\to^{n-1} c,y\to c$ et $b\to^{m-1} d, z\to d$. On applique encore une fois l'hypothèse d'induction mais à $(b,c,d)$ pour trouver $a\in E$ tel que $c\to^{m-1} a, d\to^{n-1} a$. On a alors 
        $$y\to c\to^{m-1} a\qquad z\to d\to^{n-1} a$$ d'où $y\to^m a$ et $z\to^n a$
    \end{itemize}
    Ainsi par récurrence on a trouvé un résultat qui donne l'existence de $a\in E$ tel que $y\to^* a,z\to^*a$.
\end{proof}

\begin{prop}
    $(E,\to)$ est confluent si et seulement si $(E,\to^*)$ a la propriété du diamant.
\end{prop}

\begin{proof}
    Dire que $(E,\to^*)$ a la propriété du diamant revient à dire que pour $x\to^* y$ et $x\to^* z$, il existe $c$ tel que $y\to^* c$ et $z\to^* c$, ce qui est l'énoncé de la confluence de $\to$.
\end{proof}

\begin{prop}
    Si $(E,\to^*)$ est confluent, alors $(E,\to)$ est confluent aussi.
\end{prop}

\begin{proof}
    Comme $\to^*$ est la plus petite relation réflexive et transitive contenant $\to$, la plus petite relation réflexive et transitive la contenant est elle-même, donc $\to^*$. Comme $\to^*=(\to^*)^*$ est confluent, on en déduit que $\to^*$ a la propriété du diamant, donc que $\to$ est confluent.
\end{proof}

Pour montrer que $(\Lambda,\reduc)$ est confluent, on va d'abord définir une relation intermédiaire entre $\reduc$ et $\reduc^*$ qui va, pour plusieurs réductions parallèles possibles, effectuer un nombre arbitraire de ces réductions.

\begin{defi}[Réduction parallèle]
    On définit la relation $\reduc_\|$ par les règles suivantes :
    \begin{center}
        \begin{prooftree}
            \infer0[$x\in\VV$]{x\reduc_\| x}
        \end{prooftree}
        \qquad
        \begin{prooftree}
            \hypo{M\reduc_\| M'}
            \hypo{N\reduc_\| N'}
            \infer2{M\;N\reduc_\| M'\;N'}
        \end{prooftree}
        \qquad
        \begin{prooftree}
            \hypo{M\reduc_\| M'}
            \infer1{\lambda x.M\reduc_\| \lambda x.M'}
        \end{prooftree}

        \vspace{0.5cm}

        \begin{prooftree}
            \hypo{M\reduc_\| M'}
            \hypo{N\reduc_\| N'}
            \infer2{(\lambda x.M)\;N\reduc_\| M'[N'/x]}
        \end{prooftree}
    \end{center}
\end{defi}

Donnons les propriétés essentielles de $\reduc_\|$ :

\begin{prop}
    La relation $\reduc_\|$ vérifie les propriétés suivantes :
    \begin{itemize}[label=$\bullet$]
        \item $\reduc_\|$ est réflexive.
        \item Si $M\reduc N$ alors $M\reduc_\| N$.
        \item Si $M\reduc_\| N$ alors $M\reduc^* N$.
        \item $\reduc_\|^* = \reduc^*$.
        \item Si $M\reduc_\| M'$ et $N\reduc_\| N'$ alors $M[N/x]\reduc_\| M'[N'/x]$.
    \end{itemize}
\end{prop}

\begin{proof}
    On montre ces propriétés :
    \begin{itemize}[label=$\bullet$]
        \item La relation est réflexive par induction sur un terme $M$ :
        \begin{itemize}[label=$\bullet$]
            \item $x\reduc_\| x$
            \item si $M\reduc_\| M$ et $M\reduc_\|N$ alors $M\;N\reduc_\|M\;N$
            \item si $M\reduc_\| M$ alors $\lambda x.M\reduc_\| \lambda x.M$
        \end{itemize}
        Donc par induction $M\reduc_\| M$ pour tout $M\in\Lambda$.
        \item Si $M\reduc N$ alors $M\reduc_\| N$. En effet, comme $M\reduc_\| M$ et $N\reduc_\| N$ on en déduit que $(\lambda x.M)\;N\reduc_\| M[N/x]$, et les autres règles définissant $\reduc$ sont aussi respectées.
        \item Si $M\reduc_\| N$ alors $M\reduc^* N$. On procède par induction sur $\reduc_\|$ :
        \begin{itemize}[label=$\bullet$]
            \item comme $\reduc^*$ est transitive, la première règle fonctionne.
            \item si $M\reduc^* M'$ et $N\reduc^* N'$, alors dans un premier temps $M\;N\reduc^*M\; N'$ puis $M\;N'\reduc^* M'\;N'$ donc $M\;N\reduc^* M'\;N'$.
            \item si $M\reduc^* M'$ alors $\lambda x.M\reduc^* \lambda x.M'$
            \item si $M\reduc^* M'$ et $N\reduc^* N'$ alors $(\lambda x.M)\;N\reduc^* (\lambda x.M')N$ puis $(\lambda x.M')\;N\reduc^* (\lambda x.M')\;N'$ et enfin $(\lambda x.M')\;N'\reduc M'[N'/x]$, d'où $(\lambda x.M)\;N\reduc^* M'[N'/x]$
        \end{itemize}
        Donc par induction si $M\reduc_\| N$ alors $M\reduc^* N$.
        \item $\reduc_\|^* = \reduc^*$ puisque $\reduc\subseteq \reduc_\|\subseteq \reduc^*$.
        \item Si $M\reduc_\| M'$ et $N\reduc_\| N'$ alors $M[N/x]\reduc_\| M'[N'/x]$. On raisonne par induction sur $M\reduc_\| M'$ :
        \begin{itemize}[label=$\bullet$]
            \item si $M=M'=x$ alors $M[N/x] = N\reduc_\| N' = M'[N'/x]$. Si $M=M'=y$ où $y\neq x$ alors $M[N/x] = y \reduc_\| y = M'[N/x]$.
            \item si $M = P\;Q$ et $M = P'\;Q'$ avec $P[N/x]\reduc_\| P'[N'/x],Q[N/x]\reduc_\| Q'[N'/x]$ alors $M[N/x] = P[N/x]\;Q[N/x] \reduc_\| P'[N'/x]\;Q'[N'/x] = M'[N'/x]$.
            \item si $M = \lambda y.P$ et $M' = \lambda y.P'$ avec $P[N/x]\reduc_\| P'[N'/x]$, alors $M[N/x] = \lambda y.P[N/x] \reduc_\| \lambda y.P'[N'/x] = M'[N'/x]$.
            \item si $M = (\lambda y.P)\;Q$ et $M' = P'[Q'/x]$ où $P[N/x]\reduc_\| P'[N'/x]$ et $Q[N/x]\reduc_\| Q'[N'/x]$, alors $M[N/x] = (\lambda y.P[N/x])\;Q[N/x]\reduc_\| P'[N'/x][Q'[N'/x]/y] = P'[Q'/y][N'/x] = M'[N'/x]$.
        \end{itemize}
        Donc par induction, $M[N/x]\reduc_\|M'[N'/x]$.
    \end{itemize}
\end{proof}

Définissons enfin la réduction maximale en un pas, pour un terme donné, qui nous donnera un terme vers lequel réduire deux termes issus d'un même terme.

\begin{defi}[Réduction maximal en un pas]
    On définit $M^\bullet$ par induction :
    \begin{itemize}[label=$\bullet$]
        \item si $M = x$ alors $M^\bullet = x$
        \item si $M = \lambda x.M'$ alors $M^\bullet = \lambda x.M'^\bullet$
        \item si $M = (\lambda x.M')\;N'$ alors $M^\bullet = M'^\bullet[N'^\bullet/x]$
        \item si $M = M'\;N'$ où $M'$ n'est pas une $\lambda$-abstraction, alors $M^\bullet = M'^\bullet\;N'^\bullet$.
    \end{itemize}
\end{defi}

On peut voir $M^\bullet$ comme le terme $M$ auquel on a appliqué le plus de réductions parallèles possibles. Il est donc naturel de s'attendre à ce que si $M\reduc_\| N$, donc en n'effectuant qu'une partie de ces réductions, on puisse poursuivre la réduction parallèle et avoir donc $N\reduc_\|M^\bullet$.

\begin{prop}
    Si $M\reduc_\| M'$ alors $M'\reduc_\| M^\bullet$.
\end{prop}

\begin{proof}
    On va prouver ce résultat par induction sur $M\reduc_\| M'$ :
    \begin{itemize}[label=$\bullet$]
        \item Si $M = M' = x$ alors $M^\bullet = x$ et $M'\reduc_\| M^\bullet$.
        \item Si $M = N\;P$ et $M' = N'\;P'$ où $N'\reduc_\| N^\bullet$ et $P'\reduc_\| P^\bullet$, alors dans le cas où $N$ n'est pas une $\lambda$-abstraction, cela nous donne directement $N'\;P'\reduc_\| (N\;P)^\bullet$. Dans le cas où $N$ est une $\lambda$-abstraction, on pose $N = \lambda x.Q$, et $(N\;P)^\bullet = Q^\bullet[P^\bullet/x]$, et comme $N = \lambda x.Q$, par inversion (en considérant toutes les règles d'induction on ne trouve qu'une règle permettant $N\reduc_\| N'$ dans ces conditions) on déduit que $Q'\reduc_\| Q^\bullet$ pour $N' = \lambda x.Q'$. Cela signifie donc $Q'\reduc_\| Q^\bullet$ et $P'\reduc_\| P^\bullet$, donc on en déduit que $(\lambda x.Q')\;P'\reduc_\| Q^\bullet[P^\bullet/x]$, c'est-à-dire $N'\;P'\reduc_\| (N\;P)^\bullet$.
        \item si $M = \lambda x.N$ et $M' = \lambda x.N'$ avec $N'\reduc_\| N^\bullet$, alors $M^\bullet = \lambda x.N^\bullet$ donc $M'\reduc_\| M^\bullet$.
        \item si $M = (\lambda x.N)\;P$ et $M' = N'[P'/x]$ avec $N'\reduc_\| N^\bullet$ et $P'\reduc_\| P^\bullet$, alors $M^\bullet = N^\bullet[P^\bullet/x]$ et grâce à une propriété précédemment montrée, on en déduit que $N'[P'/x]\reduc_\| N^\bullet[P^\bullet/x]$ donc que $M'\reduc_\| M$.
    \end{itemize}
    Le résultat découle donc par induction.
\end{proof}

On peut alors prouver que $\reduc_\|$ a la propriété du diamant :

\begin{prop}
    $\reduc_\|$ a la propriété du diamant.
\end{prop}

\begin{proof}
    Si l'on considère $M\reduc_\| N$ et $M\reduc_\| P$ alors par la proposition précédente, $N\reduc_\| M^\bullet$ et $P\reduc_\| M^\bullet$.
\end{proof}

On peut enfin montrer le résultat essentiel :

\begin{them}[Church-Rosser]
    Si $M\beteq N$ alors il existe $P$ tel que $M\reduc^* P$ et $N\reduc^* P$.
\end{them}

\begin{proof}
    Il nous suffit de montrer que $\reduc$ est confluent. Pour cela, on sait que $\reduc_\|$ a la propriété du diamant, donc $\reduc_\|^* = \reduc^*$ est confluent : on en déduit donc que $\reduc$ est confluent.
\end{proof}

Les conséquences principales de ce résultat sont les suivantes :

\begin{prop}
    Soit $M\in\Lambda$. Si $M$ est faiblement normalisable, alors il ne possède qu'une forme normale.
\end{prop}

\begin{proof}
    Soient $M',M''$ deux formes normales de $M$. Par définition, $M\reduc^* M'$ et $M\reduc^*M''$ donc par confluence $M'\reduc^* N$ et $M''\reduc^* N$ pour un certain $N$. De plus comme $M'$ et $M''$ sont des formes normales, $M'$ et $M''$ ne peuvent se réduire que vers elles-mêmes, donc $M'=N=M''$.
\end{proof}

\begin{prop}
    Soient $M,N$ deux lambda-termes (faiblement) normalisables. Alors $M\beteq N$ si et seulement si $M$ et $N$ ont la même forme normale.
\end{prop}

\begin{proof}
    Soit $M'$ la forme normale de $M$ et $N'$ celle de $N$. Par la propriété de Church-Rosser, on trouve $P$ tel que $M\reduc^* P$ et $N\reduc^* P$, par confluence on trouve $Q_1,Q_2$ tels que $P\reduc^* Q_1,P\reduc^* Q_2,M'\reduc^*Q_1,N'\reduc^* Q_2$ et par une dernière confluence, on trouve $Q$ tel que $Q_1\reduc^* Q$ et $Q_2\reduc^* Q$, donc $M'$ et $N'$ se réduisent vers le même terme, mais cela signifie que $Q=M'=N'$, donc $M$ et $N$ ont la même forme normale.

    Réciproquement, si $M\reduc^* P$ et $N\reduc^* P$ alors par définition $M \beteq N$.
\end{proof}

\begin{prop}
    Si $M \beteq N$ alors $M$ est faiblement normalisable si et seulement si $N$ l'est.
\end{prop}

\begin{proof}
    Si $M$ est normalisable alors on pose $M'$ sa forme normale. Comme $M\beteq M'$ et $M\beteq N$ on peut appliquer la propriété de Church-Rosser à $M'$ et $N$ : on trouve $P$ tel que $M'\reduc^* P$ et $N\reduc^* P$, mais comme $M'$ et une forme normale, $P = M'$ donc $N\reduc^* M'$ : $N$ est faiblement normalisable. La réciproque se traite de la même manière.
\end{proof}

On peut donc décrire $\Lambda_\beta$ pour les termes faiblement normalisables : les formes normales forment un système de représentants de $\mathcal N_\beta = \quot{\mathcal N}{=\beta}$ où $\mathcal N$ est l'ensemble des lambda-termes faiblement normalisables.

\subsection{Eta-réduction}

Enfin, nous allons considérer l'ajout d'une réduction appelée l'eta-réduction, et sa conséquence principale : l'exentionsalité. L'idée de l'extensionalité est que si nous interprétons un lambda-terme comme une fonction, alors deux lambda-termes qui associent des valeurs identiques devraient pouvoir être identifiés. L'eta-réduction est la version syntaxique de cette définition plus sémantique.

\begin{expl}
    Soit $M := \lambda x.y\;x$, on remarque que si l'on prend $N$ quelconque, $M\;N \beteq y\;N$. Ainsi $M$ et $y$ représentent la même fonction, mais pourtant $M$ est une forme normale, et $y$ aussi, donc $M$ et $y$ sont des éléments différents de $\Lambda_\beta$. La règle $\eta$ va permettre de définir une relation rendant égaux les termes $M$ et $y$.
\end{expl}

\begin{defi}[Eta-réduction]
    On définit l'eta-réduction par la règle suivante :
    \begin{center}
        \begin{prooftree}
            \infer0[$x\notin \varlib M$]{\lambda x.M\;x\reduc_\eta M}
        \end{prooftree}
    \end{center}
\end{defi}

Nous allons introduire un peu de vocabulaire pour décrire les relations sur les lamda-termes.

\begin{defi}[Relation compatible, congruence]
    On dit qu'une relation $\RR\subseteq \Lambda\times \Lambda$ est compatible si elle vérifie les règles suivantes :
    \begin{center}
        \begin{prooftree}
            \hypo{M\RR\;M'}
            \infer1{M\;N\RR\;M'\;N}
        \end{prooftree}
        \qquad
        \begin{prooftree}
            \hypo{N\RR\;N'}
            \infer1{M\;N\RR\;M\;N'}
        \end{prooftree}
        \qquad
        \begin{prooftree}
            \hypo{M\RR\;M'}
            \infer1{\lambda x.M\RR\;\lambda x.M'}
        \end{prooftree}
    \end{center}

    On dit que $\RR$ est une congruence si c'est une relation d'équivalence compatible.
\end{defi}

\begin{expl}
    Les relations $=$, $\beteq$, $\reduc$, $\reduc_\|$ et $\reduc^*$ sont compatibles. La relation $\beteq $ et la relation $=$ sont des congruences.
\end{expl}

On définit alors la $\beta-\eta$-réduction de la façon suivante :

\begin{defi}[$\beta\eta$-réduction]
    La relation $\reduc_{\beta\eta}$ est la plus petite relation compatible qui contient $\reduc$ et $\reduc_\eta$.
\end{defi}

\begin{defi}[$\beta\eta$-équivalence]
    La relation $=_{\beta\eta}$ est la plus petite congruence contenant $\reduc_{\beta\eta}$.
\end{defi}

En parallèle, donnons la notion d'avoir le même comportement :

\begin{defi}
    On définit la relation $\cong$, qu'on appellera isomorphisme, comme la plus petite relation d'équivalence vérifiant les règles suivantes :
    \begin{center}
        \begin{prooftree}
            \hypo{M\beteq N}
            \infer1{M\cong N}
        \end{prooftree}
        \qquad
        \begin{prooftree}
            \hypo{M\;x\cong N\;x}
            \infer1[$x\notin\varlib{M,N}$]{M\cong N}
        \end{prooftree}
        \qquad
        \begin{prooftree}
            \hypo{M\cong M'}
            \hypo{N\cong N'}
            \infer2{M\;N\cong M'\;N'}
        \end{prooftree}
    \end{center}
\end{defi}

\begin{exo}
    Montrer que si $M\cong N$ alors $\lambda x.M\cong \lambda x.N$. Montrer que $\cong$ est une congruence.
\end{exo}

Donnons un résultat important pour comprendre la motivation de la règle avec $M\;x\cong N\;x$.

\begin{prop}
    $\forall P\in\Lambda, M\;P\beteq N\;P$ si et seulement si $M\;x \beteq N\;x$ pour $x\notin \varlib{M,N}$, si et seulement si $\forall x \notin \varlib{M,N}, M\;x\beteq N\;x$.
\end{prop}

\begin{proof}
    Si $\forall P\in\Lambda,M\beteq N$ alors en considérant simplement $P = x$ on en déduit le résultat pour tout $x\notin\varlib M$ et donc en particulier pour un $x\notin\varlib M$.

    Supposons que $\exists x\notin\varlib{M,N},M\;x\beteq N\;x$. Soit $P\in\Lambda$ et $x\notin\varlib{M,N}$ comme défini plus tôt. Par hypothèse $M\;x\beteq N\;x$ donc par compatibilité de $\beteq$, $(\lambda x.M\;x)\;P\beteq (\lambda x.N\;x)\;P$ et comme $(\lambda x.M\;x)\;P\reduc M\;P$ et $(\lambda x.N\;x)\;P\reduc N\;P$ on en déduit que $M\;P\beteq N\;P$.
\end{proof}

Enfin, le résultat essentiel donnant l'intuition de l'extensionalité est le suivant :

\begin{them}
    Les relations $\cong$ et $=_{\beta\eta}$ sont égales.
\end{them}

\begin{proof}
    Montrons par induction sur $\cong$ que $\cong\subseteq =_{\beta\eta}$ :
    \begin{itemize}[label=$\bullet$]
        \item Si $M\beteq N$ alors $M=_{\beta\eta} N$.
        \item Si $M\;x=_{\beta\eta} N\;x$ pour $x\notin\varlib{M,N}$, alors par compatibilité $\lambda x.M\;x=_{\beta\eta}\lambda x.N\;x$, or $\lambda x.M\;x\reduc_\eta M$ et $\lambda x.N\;x\reduc_\eta N$, donc $M=_{\beta\eta} N$.
        \item Si $M=_{\beta\eta} M'$ et $N=_{\beta\eta} N'$ alors par compatibilité $M\;N=_{\beta\eta} M\;N'$ puis $M\;N'=_{\beta\eta} M'\;N'$ donc $M\;N=_{\beta\eta} M'\;N'$
    \end{itemize}
    Donc $\cong\subseteq =_{\beta\eta}$.

    Réciproquement, en faisant une induction sur $M=_{\beta\eta} N$ :
    \begin{itemize}[label=$\bullet$]
        \item Si $M\reduc N$ alors $M\cong N$.
        \item Si $M\reduc_\eta N$ alors on a $M = \lambda x.N\;x$ pour $x\notin\varlib{M,N}$ et on remarque que pour $y\notin\varlib{M,N}$ on a bien $(\lambda x.N\;x)\;y\beteq N\;y$ donc $(\lambda x.N\;x)\;y\cong N\;y$.
        \item Par définition $\cong$ est une relation d'équivalence.
        \item On a montré dans un exercice précédent que $\cong$ est une relation d'équivalence.
    \end{itemize}
    Donc $=_{\beta\eta}\subseteq \cong$.
\end{proof}

\begin{exo}
    Adapter la preuve du théorème de Church-Rosser au cas de $=_{\beta\eta}$, en adaptant $\reduc_\|$ et en montrant que cette relation vérifie les mêmes propriétés mais vis à vis de la $\beta-\eta$ réduction plutôt que la $\beta$-réduction.
\end{exo}

\chapter{Lambda-calcul simplement typé}

Maintenant que nous avons la théorie du lambda-calcul, nous allons introduire la notion de typage. Ce chapitre s'attachera au typage en lui-même et donnera des résultats théoriques motivant le typage (principalement le théorème de normalisation forte) ainsi que les résultats pour utiliser un système de type.

Motivons d'abord l'introduction des types. Nous avons parlé plus tôt de la normalisation, et avons vu qu'un terme tel que $\Omega := (\lambda x.x\;x)(\lambda x.x\;x)$ n'est pas normalisable. Cependant, on peut se demander si un tel terme correspond vraiment à une fonction comme on l'imagine. En effet, le terme $\lambda x.x\;x$ prend un terme et l'applique à lui-même : dans la conception habituelle de fonction, c'est un processus impossible puisque $E \neq F^E$ pour quelque $F$ que ce soit, donc si $f : E \to F$ alors on ne peut pas donner $f$ en argument à $f$. Il y a donc naturellement une notion d'homogénéité dans notre appréciation des fonctions, et cette notion que vont traduire les types. De plus, une conséquence particulièrement appréciable du fait de forcer l'homogénéité est que tous les termes homogènes en ce sens, sont fortement normalisables.

La théorie des types a d'ailleurs été initialement introduite par Russell pour éviter le paradoxe d'avoir un ensemble de tous les ensembles, qui d'une certaine façon peut se rapprocher de l'élimination des lambda-termes s'appliquant à eux-mêmes. Un type sera donc une étiquette ajoutée à un terme pour indiquer sa nature. On pourra par exemple trouver le type $\intt$ qui étiquette les termes qui représenteront les entiers.

\section{Types des structures habituelles}

Cette section sera séparée en plusieurs parties : nous allons chaque fois renforcer notre grammaire de typage et les termes que l'on peut construire, en partant de la notion la plus simple, celle-ci étant de simplement ajouter des types aux lambda-termes habituels.

\subsection{Types des fonctions}

Commençons par introduire la notion de types. On définit pour cela un ensemble de types de base $\iota\in \mathcal B$ qui pourra moralement représenter tous les types habituels concrets, leur donnée exacte n'étant pas pertinente dans l'étude théorique du lambda-calcul simplement typé. On ajoute le constructeur de type $\to$ qui dénote, étant donnés deux types $\sigma$ et $\tau$, le type des fonctions qui à un argument de type $\sigma$ associe une sortie de type $\tau$.

\begin{defi}[Grammaire des types]
    On définit inductivement l'ensemble $T_{\to}$ des types par la grammaire suivante :
    $$\sigma,\tau ::= \iota \mid \sigma\to \tau$$ où $\sigma,\tau\in T_{\to}$ et $\iota\in\mathcal B$.
\end{defi}

\begin{rmk}
    On utilisera une associativité à droite de la flèche : $\tau\to\tau'\to\sigma$ se lira $(\tau\to\tau')\to\sigma$. Cela se justifie de la même manière que notre convention d'associativité à gauche de l'application, car une fonction de la forme $f : (x,y) \mapsto f(x,y)$, une fois curryfiée, donnera une fonction de la forme $f : x\mapsto (y\mapsto f(x,y))$ qui, pour $x$ de type $\tau$, $y$ de type $\tau'$ et $f(x,y)$ de type $\sigma$, a le type $\tau\to(\tau'\to\sigma)$. Comme cette situation arrive très fréquemment, on simplifie l'écriture pour n'avoir qu'à écrire $\tau\to\tau'\to\sigma$.
\end{rmk}

Deux choix sont possibles pour définir un lambda-calcul typé : considérer les termes comme des lambda-termes classiques et y ajouter des annotations de type, ou bien construire des termes déjà typés en partie. Nous opterons pour le second, car il nous semble plus proche de la philosophie de la théorie des types, où les lambda-termes dépendent des types.

\begin{defi}[Lambda-terme typé]
    On définit l'ensemble des pré-lambda-termes typés $\Lambda_0^{\to}$ par la grammaire suivante :
    $$M,N ::= x^\tau\mid \lambda x^\tau.M\mid (M\;N)$$ où $x\in \VV,\tau\in T, M,N\in \Lambda_0^{\to}$.
\end{defi}

Le typage est alors l'action d'associer à un lambda-terme donné, un type. Remarquons que le processus se fait en deux temps : on commence par définir le lambda-terme, avant de définir son type. Cependant, comme nous le prouverons plus tard, la procédure qui à un lambda-terme associe son type est univoque. Nous utiliserons une notation déjà vue en logique, qui est celle des séquents. Pour pouvoir l'utiliser, nous allons définir la notion d'environnement de typage avant de définir les jugements de typage.

Dans $\Lambda_0^{\to}$, nous avons ajouté des annotations de types aux variables et aux abstractions, mais nous noterons rarement les annotations sur les variables. Pour les abstractions, on pourra aussi noter $\lambda (x : \tau).$ à la place de $\lambda x^\tau.$ mais le deuxième étant plus court, il sera privilégié ici.

\begin{defi}[Environnement de typage]
    Un environnement de typage est une liste $\Gamma := (\VV\times T)^*$. On notera en général la liste dans le sens inverse du sens habituel : l'élément de tête sera à droite. L'intuition d'un environnement $\Gamma$ est d'associer à un nombre fini de variables un certain type. Plutôt que $[(x_1,\sigma_1),\ldots,(x_n,\sigma_n)]$ nous écrirons $x_1 : \sigma_1,\ldots,x_n : \sigma : n$ pour écrire la liste, et $\Gamma,x : \sigma$ pour signifie la liste à laquelle on ajoute $(x,\sigma)$ en tête de liste.
\end{defi}

\begin{defi}[Jugement de typage]
    On définit par induction une relation $\vdash \subseteq (\VV\times T)^* \times (\Lambda_0^{\to} \times T)$, nommée relation de typage, qu'on notera $\Gamma\vdash M : \sigma$ pour $\vdash (\Gamma,(M,\sigma))$, par la relation suivante :
    \begin{center}
        \begin{prooftree}
            \infer0{\Gamma,x : \tau \vdash x^\tau : \tau}
        \end{prooftree}
        \qquad
        \begin{prooftree}
            \hypo{\Gamma\vdash M : \tau}
            \infer1[$x\notin \varlib M$]{\Gamma,x : \sigma \vdash M : \tau}
        \end{prooftree}
        
        \vspace{0.5cm}
        
        \begin{prooftree}
            \hypo{\Gamma,x : \sigma\vdash M :\tau}
            \infer1{\Gamma\vdash \lambda x^\sigma. M : \sigma \to \tau}
        \end{prooftree}
        \qquad
        \begin{prooftree}
            \hypo{\Gamma\vdash M : \sigma\to\tau}
            \hypo{\Gamma\vdash N : \sigma}
            \infer2{\Gamma\vdash (M\;N) : \tau}
        \end{prooftree}
    \end{center}

    On appelle jugement de typage une instance de cette relation. Si $\Gamma$ est la liste vide, on notera $\vdash M : \tau$ pour $\varnothing \vdash M : \tau$.
\end{defi}

\begin{expl}
    Nous allons montrer que l'on peut typer le terme $\lambda f^{\iota\to\iota}.\lambda x^{\iota}.f\;(f\;x)$ dans l'environnement vide :
    \begin{center}
        \begin{prooftree}
            \hypo{f : \iota\to\iota \vdash f : \iota\to\iota}
            \infer1{f : \iota\to\iota,x : \iota \vdash f : \iota\to\iota}
            \hypo{f : \iota\to\iota \vdash f : \iota\to\iota}
            \infer1{f : \iota\to\iota,x : \iota \vdash f : \iota\to\iota}
            \hypo{f : \iota\to\iota, x : \iota\vdash x^\iota : \iota}
            \infer2{f : \iota\to\iota, x : \iota \vdash f\; x^\iota : \iota}
            \infer2{f : \iota\to\iota, x :\iota \vdash f\;(f\;x^\iota) : \iota}
            \infer1{f : \iota\to\iota\vdash \lambda x^\iota.f\;(f\;x^\iota) : \iota\to\iota}
            \infer1{\vdash \lambda f^{\iota\to\iota}.\lambda x^\iota. f\;(f\;x^\iota) : (\iota\to\iota)\to\iota\to\iota}
        \end{prooftree}
    \end{center}
\end{expl}

Nous allons maintenant montrer des résultats de structure sur la relation de typage. Ceux-ci seront essentiels pour pouvoir montrer des résultats de façon rapide.

\begin{prop}[Unicité du typage]
    Soient $M\in\Lambda_0^{\to}$ et $\Gamma,\Gamma'$ deux environnements tels que $\Gamma\vdash M : \tau$ et $\Gamma'\vdash M : \sigma$, alors $\tau = \sigma$.
\end{prop}

\begin{proof}
    On procède par induction sur $\Gamma\vdash M : \tau$ :
    \begin{itemize}[label=$\bullet$]
        \item Si $\Gamma\vdash x^{\tau'} : \tau$ où $\Gamma = \Delta,x : \tau$, alors on raisonne par induction sur $\Gamma'\vdash x^{\tau'} : \sigma$ :
        \begin{itemize}[label=$\bullet$]
            \item Si $\Gamma'\vdash x^{\tau'} : \sigma$ où $\Gamma' = \Delta', x : \sigma$ alors par construction $\sigma = \tau'$ et de même $\tau = \tau'$ donc au total $\tau = \sigma$.
            \item Si $\Delta'\vdash x^{\tau'} : \sigma'$ où $\Delta = \Delta',y : \tau'',y\notin\varlib M$ et $\sigma' = \tau$ alors $\Delta \vdash x^{\tau'} : \sigma'$ et $\sigma = \sigma'$ par hypothèse d'induction donc $\sigma = \tau = \sigma'$.
            \item Dans les deux autres cas $M$ ne peut pas être mis sous la forme voulue, donc la prémisse est fausse, menant à une conclusion vraie peu importe la conclusion.
        \end{itemize}
        \item Si $\Delta\vdash M : \tau'$ où $\Gamma = \Delta,x : \tau'$ avec $x\notin\varlib M$ et $\tau' = \sigma$ alors $\Gamma\vdash M : \tau$ et $\tau = \tau' = \sigma$.
        \item Si $M = \lambda x^{\tau'}.N$ et que $\Gamma,x : \tau' \vdash N : \tau''$ où $\tau = \tau'\to\tau''$, alors on raisonne par induction sur $\Gamma'\vdash M$ en éliminant les deux cas non pertinents :
        \begin{itemize}[label=$\bullet$]
            \item Si $\Delta'\vdash M :\sigma$ avec $\Delta',y : \sigma' = \Delta, y \notin\varlib N$ et $\sigma = \tau$ alors on en déduit que $\tau = \sigma$ et $\Delta\vdash M : \sigma$.
            \item Si $\Gamma', x : \tau'\vdash N : \sigma'$ alors par hypothèse d'induction $\sigma' = \tau$ donc $\Gamma'\vdash \lambda x^{\tau'}. N :\tau'\to\sigma'$ et $\Gamma\vdash \lambda x^{\tau'}.N : \tau'\to\sigma'$ donc $\tau = \sigma$.
        \end{itemize}
        \item Si $M = P\;Q$ alors on raisonne de façon analogue au cas précédent pour montrer que $\tau = \sigma$.
    \end{itemize}
    Donc $\tau = \sigma$.
\end{proof}

\begin{exo}
    Montrer le lemme de structure suivant, pour $x\neq y$ : si $\Gamma,x : \tau,y : \tau',\Gamma'\vdash M : \sigma$ alors $\Gamma,y : \tau',x : \tau,\Gamma'\vdash M : \sigma$.
\end{exo}

\begin{exo}
    Montrer le lemme de structure suivant : si $x\notin\varlib M$ et que $\Gamma\vdash M : \tau$ alors $\Gamma'\vdash M : \tau$ où $\Gamma'$ est l'environnement $\Gamma$ où l'on a retiré la dernière occurrence de $x$.
\end{exo}

\begin{defi}[Terme typable]
    On dit qu'un terme $M$ est typable s'il existe un environnement $\Gamma$ et type $\tau$ tels que $\Gamma\vdash M : \tau$. On dit que $M$ est typable dans l'environnement $\Gamma$ s'il existe un type $\tau$ tel que $\Gamma\vdash M : \tau$. On dit que $\tau$ est habité s'il existe un lambda-terme $M$ tel que $\vdash M : \tau$.
\end{defi}

\begin{rmk}
    Si $M$ est typable alors le type associé est unique d'après une propriété précédente.
\end{rmk}

Nous voulons alors quotienter les lambda-termes par $\alpha$-équivalence pour définir $\Lambda^{\to} = \quot{\Lambda_0^{\to}}{=_\alpha}$, mais il faut pour cela définir la substitution sur les termes typés, qui prend en compte la cohérence des types.

\begin{prop}[Substitution typée]
    Soient $M,N\in\Lambda_0^{\to}$, $x\in\VV$ et $\Gamma$ un environnement tel que $$\Gamma\vdash M : \tau\qquad \Gamma\vdash N : \sigma\qquad \Gamma\vdash x : \sigma$$ alors $\Gamma\vdash M[N/x] : \tau$ où $M[N/x]$ est défini comme pour une substitution non typée.
\end{prop}

\begin{proof}
    On procède par induction sur $M[N/x]$ :
    \begin{itemize}[label=$\bullet$]
        \item Si $M = x$, alors $\Gamma\vdash x : \tau$ donc par l'exercice précédent $\tau = \sigma$, d'où $\Gamma\vdash N : \tau$.
        \item Si $M = y$, alors $\Gamma\vdash y : \tau$.
        \item Si $M = \lambda y^{\tau'}.M'$ pour $y\notin\varlib{N}$, et par hypothèse d'induction $\Gamma, y : \tau'\vdash M'[N/x] : \tau''$ et $\tau = \tau'\to\tau''$, alors en appliquant la règle de typage correspondante on en déduit que $\Gamma\vdash \lambda y.M' : \tau'\to\tau''$ d'où $\Gamma\vdash M[N/x] : \tau$.
        \item Si $M = (P\;Q)$, $\Gamma\vdash P[N/x] : \tau'\to\tau$ et $\Gamma\vdash Q[N/x] : \tau'$ alors on en déduit que $\Gamma\vdash (P[N/x]\;Q[N/x]) : \tau$ donc par définition de $[N/x]$ cela signifie $\Gamma\vdash M[N/x] : \tau$.
    \end{itemize}
    Donc par induction $\Gamma\vdash M[N/x] : \tau$.
\end{proof}

\begin{exo}
    Montrer que si $\Gamma\vdash (\lambda x.M) : \tau$ alors $\Gamma'\vdash (\lambda y.M[y/x]) : \tau$ où $y\notin\varlib M\cup\Gamma$ et $\Gamma'$ est l'environnement $\Gamma$ dans lequel on a remplacé chaque couple de la forme $(x,\sigma)$ par $(y,\sigma)$. En déduire que $\lambda x.M$ est typable si et seulement si $\lambda y.M[y/x]$ l'est aussi.
\end{exo}

\begin{defi}[$\alpha$-équivalence]
    On définit notre $\alpha$-équivalence $=_\alpha$ comme la plus petite congruence vérifiant la règle suivante :
    \begin{center}
        \begin{prooftree}
            \infer0[$y\notin\varlib M$]{\lambda x.M =_\alpha \lambda y.M[y/x]}
        \end{prooftree}
    \end{center}

    On pose alors $\Lambda^{\to} = \quot{\Lambda_0^{\to}}{=_\alpha}$ l'ensemble des termes typés.
\end{defi}

\begin{exo}
    Montrer que si $M =_\alpha N$ et $\Gamma\vdash M : \tau$ alors $\Gamma\vdash N : \tau$ et donc que le typage est bien défini sur $\Lambda^{\to}$.
\end{exo}

\begin{them}[Préservation du typage]
    En adaptant la réduction $\reduc$ aux lambda-termes typés de façon évidente, si $\Gamma\vdash M : \tau$ et $M\reduc N$ alors $\Gamma\vdash N : \tau$.
\end{them}

\begin{proof}
    On raisonne par induction sur $M\reduc N$ :
    \begin{itemize}[label=$\bullet$]
        \item Si $\Gamma\vdash M : \tau$ et que $M = (\lambda x^\sigma.P)Q$ avec $N = Q[P/x]$ alors on peut prendre $x\notin\varlib Q$ et faire une inversion sur le typage pour trouver que $\Gamma\vdash Q : \sigma$ et $\Gamma,x : \sigma\vdash P : \tau$, mais cela signifie aussi que $\Gamma,x : \sigma\vdash Q : \sigma$ et $\Gamma, x:\sigma\vdash x : \sigma$ donc $\Gamma,x : \sigma \vdash P[Q/x] : \tau$. De plus comme $x\notin\varlib P[Q/x]$, on en déduit que $\Gamma\vdash P[Q/x] : \tau$.
        \item Les autres cas se déroulent directement en appliquant les hypothèses d'induction.
    \end{itemize}
\end{proof}

On en déduit que $\Lambda^\to_\beta = \quot{\Lambda^{\to}}{\beteq}$ est compatible avec la relation de typage.

\begin{exo}
    Vérifier que le théorème de Church-Rosser s'applique encore pour $\Lambda^{\to}$.
\end{exo}

\begin{exo}
    Vérifier que l'ajout de la règle $\eta$ a les mêmes propriétés que pour $\Lambda$.
\end{exo}

\subsection{Type des paires}

On définit ici un nouveau lambda-calcul, nommé $\Lambda^{\to\times 1}$ qui permet de considérer des paires. En effet, contrairement au lambda-calcul non typé, il n'est pas possible en lambda-calcul de coder directement les paires et les projections. L'argument intuitif est que le codage $\langle M,N\rangle := \lambda p. p\;M\;N$ quantifie $p$ sur tous les types possibles et il faudrait donc pouvoir définir une fonction prenant un type quelconque. Nous verrons que cela est possible si on autorise des quantifications de second ordre, mais le typage simple est juste un typage qui n'utilise pas ces quantifications. On ajoute donc en parallèle les types produits, que l'on peut voir comme des analogues des produits cartésiens, le type unit qui est un type contenant un unique élément, et les fonctions permettant à ces types d'être construits et utilisés.

\begin{defi}[Types produit]
    On définit un nouvel ensemble de types, que l'on notera $T_{\to\times 1}$, par la grammaire suivante :
    $$\sigma,\tau ::= \iota\mid \unit\mid \sigma\to\tau\mid \sigma \times \tau$$ où $\iota\in\mathcal B$, $\unit$ est une constante de type, et $\sigma,\tau\in T_{\to\times 1}$.
\end{defi}

\begin{rmk}
    On note $\times$ prioritaire sur $\to$, donc $\sigma\times\tau\to\kappa$ se lit $(\sigma\times\tau)\to\kappa$.
\end{rmk}

\begin{defi}[$\Lambda_0^{\to\times 1}$]
    On définit l'ensemble $\Lambda_0^{\to\times 1}$ par la grammaire enrichie sur celle de $\Lambda^{\to}$ suivante :
    $$M,N ::= \ldots \mid \langle M,N\rangle \mid \pi_1\;M\mid \pi_2\;M\mid \langle\rangle$$ On définit $=_\alpha$ de la même façon que pour $\Lambda_0^{\to}$ et on définit alors $\Lambda^{\to\times 1} = \quot{\Lambda_0^{\to\times 1}}{=_\alpha}$.
\end{defi}

\begin{defi}[Typage dans $\Lambda^{\to\times 1}$]
    On définit la relation de typage $\vdash$ en ajoutant des règles à la relation de typage $\vdash$ dans $\Lambda^{\to}$ :
    \begin{center}
        \begin{prooftree}
            \hypo{\Gamma\vdash M : \tau}
            \hypo{\Gamma\vdash N : \sigma}
            \infer2{\Gamma\vdash \langle M,N\rangle : \tau\times \sigma}
        \end{prooftree}
        \qquad
        \begin{prooftree}
            \hypo{\Gamma\vdash M : \tau\times\sigma}
            \infer1{\Gamma\vdash \pi_1\;M : \tau}
        \end{prooftree}
        \qquad
        \begin{prooftree}
            \hypo{\Gamma\vdash M : \tau\times\sigma}
            \infer1{\Gamma\vdash \pi_2\;M : \sigma}
        \end{prooftree}

        \vspace{0.5cm}

        \begin{prooftree}
            \infer0{\Gamma\vdash \langle\rangle : \unit}
        \end{prooftree}
    \end{center}
\end{defi}

\begin{exo}
    Montrer que la relation de typage est bien définie sur $\Lambda^{\to\times 1}$ i.e. que si $M=_\alpha N$ et $\Gamma\vdash M : \tau$ alors $\Gamma\vdash N : \tau$. \textit{Indication : on généralisera d'abord les propriétés de structure nécessaires de $\Lambda_0^{\to}$.}
\end{exo}

\begin{defi}[$\beta$-réduction]
    On définit aussi la relation $\reduc\subseteq\Lambda^{\to\times 1}\times\Lambda^{\to\times 1}$ comme la plus petite relation compatible contenant les règles suivantes :
    \begin{center}
        \begin{prooftree}
            \infer0{(\lambda x^{\tau}.M)\;N\reduc M[N/x]}
        \end{prooftree}
        \qquad
        \begin{prooftree}
            \infer0{\pi_1\;\langle M,N\rangle\reduc M}
        \end{prooftree}
        \qquad
        \begin{prooftree}
            \infer0{\pi_2\;\langle M,N\rangle\reduc N}
        \end{prooftree}
    \end{center}

    Et $\beteq$ comme la congruence associée.
\end{defi}

\begin{rmk}
    Comme nous avons augmenté le nombre de constructeurs, les relations que vérifie une congruence ou une relation compatible sont aussi plus nombreuses : 
    \begin{center}
        \begin{prooftree}
            \hypo{M\;\RR\;M'}
            \infer1{\langle M,N\rangle\;\RR\;\langle M',N\rangle}
        \end{prooftree}
        \qquad
        \begin{prooftree}
            \hypo{N\;\RR\;N'}
            \infer1{\langle M,N\rangle\;\RR\;\langle M,N'\rangle}
        \end{prooftree}

        \vspace{0.5cm}

        \begin{prooftree}
            \hypo{M\;\RR\;M'}
            \infer1{\pi_1\;M\;\RR\;\pi_1\;M'}
        \end{prooftree}
        \qquad
        \begin{prooftree}
            \hypo{M\;\RR\;M'}
            \infer1{\pi_2\;M\;\RR\;\pi_2\;M'}
        \end{prooftree}
    \end{center}
\end{rmk}

\begin{exo}
    Vérifier que la préservation du typage est encore vraie.
\end{exo}

\begin{exo}
    Vérifier que $(\Lambda^{\to\times 1},\reduc)$ vérifie bien la propriété de Church-Rosser.
\end{exo}

\begin{defi}[$\eta$-réduction]
    On définit la relation $\reduc_{\beta\eta}\subseteq\Lambda^{\to\times 1}\times\Lambda^{\to\times 1}$ comme la plus petite relation compatible contenant $\reduc$ et les règles suivantes :
    \begin{center}
        \begin{prooftree}
            \infer0[$x\notin\varlib M$]{\lambda x^\tau.M\;x\reduc_{\beta\eta} M}
        \end{prooftree}
        \qquad
        \begin{prooftree}
            \infer0{\langle \pi_1\;M,\pi_2\;M\rangle \reduc_{\beta\eta} M}
        \end{prooftree}
        \qquad
        \begin{prooftree}
            \infer0[$M : \unit$]{M\reduc_{\beta\eta} \langle\rangle}
        \end{prooftree}
    \end{center}

    Et $=_{\beta\eta}$ comme la congruence associée.
\end{defi}

\begin{rmk}
    La propriété de Church-Rosser ne tient pas pour $(\Lambda^{\to\times 1},\reduc_{\beta\eta})$ : en effet, si on prend $x : \tau\times\unit$ et qu'on considère $M := \lambda \pi_1\;x,\pi_2\;x\rangle$ alors on peut au choix réduire $M$ en $x$ ou bien en $\langle \pi_1\;x,\langle\rangle\rangle$ qui sont deux formes normales.
\end{rmk}

\begin{exo}
    Définir la relation $\cong$ sur $\Lambda^{\to\times 1}$ et montrer qu'elle est égale à $=_{\beta\eta}$.
\end{exo}

\subsection{Types des unions disjointes}

Nous allons enfin ajouter un constructeur de type analogue aux sommes d'ensembles, c'est-à-dire aux unions disjointes. En théorie des ensembles, $E + F$, que l'on note aussi $E\sqcup F$, est défini par $E + F := \{(x,0)\mid x\in E\}\cup \{(y,1)\mid y\in F\}$, et on possède alors les fonctions $$\fonction{\kappa_1}{E}{E+F}{x}{(x,0)}\qquad\fonction{\kappa_2}{F}{E+F}{y}{(y,1)}$$ qui sont universelle en ce sens que s'il existe deux fonctions $f : E \to G$ et $g : F\to G$ pour un ensemble $G$ quelconque, alors il existe une unique fonction $[f,g] : E+F\to G$ telle que $[f,g](x,0) = f(x)$ et $[f,g](y,1) = g(y)$ pour $x\in E, y\in F$. Cet ensemble de fonctions et cette propriété universelle caractérisent presque l'ensemble $E+F$ (en fait à bijection près) mais cela suffit largement pour décrire le comportement de l'objet $E+F$. D'un point de vue du lambda-calcul, cette définition utilisant principalement des fonctions est privilégiée à l'approche ensembliste de base car elle permet de définir facilement les constructeurs dont nous avons besoin : un constructeur qui se comporte comme $\kappa_1$, un autre comme $\kappa_2$ et un constructeur de la forme $[-,-]$ qui permet de construire des fonctions $E+F\to G$ à partir d'une fonction $E\to G$ et une fonction $F\to G$. On ajoute enfin un type vide, correspondant à l'ensemble $\varnothing$. En théorie des ensembles on possède une fonction $\varnothing \to E$ pour tout ensemble $E$, donc il faut introduire aussi une telle fonction.

\begin{defi}[Types somme]
    On définit $T_{\to\times 1+0}$ par la grammaire suivante : $$\sigma,\tau ::= \iota\mid\unit\mid\voidt\mid \sigma\to\tau\mid\sigma\times\tau\mid\sigma+\tau$$ où $\iota\in\mathcal B$, $\unit$ et $\voidt$ sont des constantes de types et $\sigma,\tau\in T_{\to\times 1+0}$.
\end{defi}

\begin{rmk}
    On donne une priorité à $+$ intermédiaire entre $\times$ et $\to$ : $\tau\times\sigma+\kappa$ se lit $(\tau\times\sigma)+\kappa$ et $\tau+\sigma\to\kappa$ se lit $(\tau+\sigma)\to\kappa$.
\end{rmk}

\begin{defi}[$\Lambda^{\to\times 1+0}$]
    En quotientant par $\alpha$-équivalence, on définit $\Lambda^{\to\times 1+0}$ en enrichissant la grammaire de $\Lambda^{\to\times 1}$ :
    $$M,N,P::=\ldots\Big|\; \kappa_1\;M\;\Big|\kappa_2\;N\;\Big|\;\delta\;(x\mapsto M\mid y\mapsto N)\;P\;\Big|\; \delta_\bot\;M$$
\end{defi}

\begin{rmk}
    La gestion des variables libre doit être actualisée par rapport au fait que $x$ est liée dans $\delta\;(x\mapsto M\mid y\mapsto N)\;P$, mais cette adaptation est un processus administratif. De plus le constructeur peut prendre des variables différentes, par exemple $\delta\;(a\mapsto M\mid b\mapsto N)\;P$ est aussi valide, et l'identification dans $=_\alpha$ a donc une règle en plus.
\end{rmk}

\begin{defi}[Typage dans $\Lambda^{\to\times 1+0}$]
    On définit $\vdash$ sur $\Lambda^{\to\times 1+0}$ en enrichissant les règles précédentes :
    \begin{center}
        \begin{prooftree}
            \hypo{\Gamma\vdash M : \tau}
            \infer1{\Gamma\vdash \kappa_1\;M : \tau+\sigma}
        \end{prooftree}
        \qquad
        \begin{prooftree}
            \hypo{\Gamma\vdash M : \sigma}
            \infer1{\Gamma\vdash \kappa_1\;M : \tau+\sigma}
        \end{prooftree}
        \qquad
        \begin{prooftree}
            \hypo{\Gamma\vdash P : \sigma+\tau}
            \hypo{\Gamma,x : \sigma\vdash M : \kappa}
            \hypo{\Gamma,y : \tau\vdash N : \kappa}
            \infer3{\Gamma\vdash \delta\;(x\mapsto M\mid y\mapsto N)\;P : \kappa}
        \end{prooftree}

        \vspace{0.5cm}

        \begin{prooftree}
            \hypo{\Gamma\vdash M : \voidt}
            \infer1{\Gamma\vdash \delta_\bot\;M : \tau}
        \end{prooftree}
    \end{center}
\end{defi}

\begin{rmk}
    Il n'y a pas de constructeur de type $\voidt$, ce qui est normal pour un type vide. De plus au lieu de considérer une fonction $M : \sigma\to\kappa$ on considère un terme $M : \kappa$ en ajoutant $x : \sigma$ à l'environnement ; s'il est évident que les deux situations sont équivalentes, la deuxième nous permettra une meilleure visualisation de l'isomorphisme de Curry-Howard développé dans un chapitre ultérieur.
\end{rmk}

\begin{rmk}
    La règle de typage pour $\delta_\bot$ fait qu'il n'y a plus unicité du type d'une expression. On peut contourner ce problème en définissant plutôt $\delta_\bot^\tau$ où on indique le type d'arrivée de $\delta_\bot$, et en considérant qu'on ne l'écrira jamais, de la même façon qu'on note $x$ une variable et non $x^\tau$.
\end{rmk}

On ajoute ensuite les réductions.

\begin{defi}[$\beta$-relations]
    On définit la relation $\reduc$ comme la plus petite relation compatible contenant les règles suivantes :
    \begin{center}
        \begin{prooftree}
            \infer0{(\lambda x^{\tau}.M)\;N\reduc M[N/x]}
        \end{prooftree}
        \qquad
        \begin{prooftree}
            \infer0{\pi_1\;\langle M,N\rangle\reduc M}
        \end{prooftree}
        \qquad
        \begin{prooftree}
            \infer0{\pi_2\;\langle M,N\rangle\reduc N}
        \end{prooftree}

        \vspace{0.5cm}
        
        \begin{prooftree}
            \infer0{\delta\;(x\mapsto M\mid y\mapsto N)\;(\kappa_1\;P)\reduc M[P/x]}
        \end{prooftree}
        \qquad
        \begin{prooftree}
            \infer0{\delta\;(x\mapsto M\mid y\mapsto N)\;(\kappa_2\;P)\reduc N[P/y]}
        \end{prooftree}
    \end{center}

    Et $\beteq$ est alors la congruence associée.
\end{defi}

\begin{rmk}
    On ajoute encore des relations pour la compatibilité :
    \begin{center}
        \begin{prooftree}
            \hypo{M\;\RR\;M'}
            \infer1{\kappa_1\;M\;\RR\;\kappa_1\;M'}
        \end{prooftree}
        \qquad
        \begin{prooftree}
            \hypo{M\;\RR\;M'}
            \infer1{\kappa_2\;M\;\RR\;\kappa_2\;M'}
        \end{prooftree}
        \qquad
        \begin{prooftree}
            \hypo{M\;\RR\;M'}
            \infer1{\delta\;(x\mapsto M\mid y\mapsto N)\;P\;\RR\;\delta\;(x\mapsto M'\mid y\mapsto N)\;P}
        \end{prooftree}

        \vspace{0.5cm}
        
        \begin{prooftree}
            \hypo{N\;\RR\;N'}
            \infer1{\delta\;(x\mapsto M\mid y\mapsto N)\;P\;\RR\;\delta\;(x\mapsto M\mid y\mapsto N')\;P}
        \end{prooftree}
        \qquad
        \begin{prooftree}
            \hypo{P\;\RR\;P'}
            \infer1{\delta\;(x\mapsto M\mid y\mapsto N)\;P\;\RR\;\delta\;(x\mapsto M\mid y\mapsto N)\;P'}
        \end{prooftree}

        \vspace{0.5cm}
        
        \begin{prooftree}
            \hypo{M\;\RR\;M'}
            \infer1{\delta_\bot\;M\;\RR\;\delta_\bot\;M'}
        \end{prooftree}
    \end{center}
\end{rmk}

On notera aussi $\reduc_0$ la réduction en surface, qui est la version de $\reduc$ sans la compatibilité, c'est-à-dire n'effectuant une réduction que si celle-ci apparait directement dans le terme entier.

\begin{defi}[$\beta$-$\eta$-relations]
    On définit la relation $\reduc_{\beta\eta}$ comme la plus petite relation compatible contenant $\reduc$ et les règles suivantes :
    \begin{center}
        \begin{prooftree}
            \infer0[$x\notin\varlib M$]{\lambda x^\tau.M\;x\reduc_{\beta\eta} M}
        \end{prooftree}
        \qquad
        \begin{prooftree}
            \infer0{\langle \pi_1\;M,\pi_2\;M\rangle \reduc_{\beta\eta} M}
        \end{prooftree}
        \qquad
        \begin{prooftree}
            \infer0[$M : \unit$]{M\reduc_{\beta\eta} \langle\rangle}
        \end{prooftree}

        \vspace{0.5cm}

        \begin{prooftree}
            \infer0{\delta\;(x\mapsto M\;(\kappa_1\;x)\mid y\mapsto M\;(\kappa_2\;y))\;P\reduc_{\beta\eta} M\;P}
        \end{prooftree}
    \end{center}

    Et $=_{\beta\eta}$ comme la congruence associée.
\end{defi}

\begin{exo}[Booléens]
    On définit le type $\boolt := \unit + \unit$ et $\top_\boolt := \kappa_1 \;\langle\rangle, \bot_\boolt := \kappa_2\;\langle\rangle$. Définir les opérations booléennes usuelles $\lnot,\land,\lor$ et vérifier qu'elles se comportent comme attendu. Soit $\tau \in T_{\to\times1+0}$, définir une fonction $\ifthenelsee{-}{-}{-} : \boolt\to\tau\to\tau\to\tau$ telle que \begin{align*}
        \ifthenelsee{\top_\boolt}{M}{N} &\beteq M\\
        \ifthenelsee{\bot_\boolt}{M}{N} &\beteq N
    \end{align*}
\end{exo}


\section{Normalisation des termes typables}

Nous allons voir la propriété essentielle du lambda-calcul simplement typé : tout lambda-terme bien typé est normalisable, faiblement mais aussi fortement. Nous allons dans un premier temps montrer la propriété de la disjonction pour illustrer la technique de preuve qui nous servira ensuite pour la normalisation faible, puis la normalisation forte. Nous nous plaçons dans $\Lambda^{\to\times 1+0}$ mais les résultats fonctionnent aussi en se restreignant respectivement à $\Lambda^{\to\times 1}$ et $\Lambda^\to$. Pour la normalisation faible, nous allons introduire la notion de réduction de tête, qui est une réduction plus faible que $\reduc$ mais nous détaillons dans le passage à la normalisation forte comment adapter la normalisation faible pour la réduction de tête à celle pour la $\beta$-réduction classique.

\subsection{Préliminaires}

Commençons par donner quelques définitions préliminaires pour faciliter la lecture de nos résultats.

\begin{defi}[Substitution simultanée]
    Soit $\sigma : \VV\to\Lambda^{\to\times1+0}$ une fonction partielle. Pour un terme $M$ on donne la définition de la substitution simultanée $M\;\sigma$ par induction sur $M$, où nous noterons $\varlib{\sigma} = \displaystyle{\bigcup_{x\in\mathrm{dom}(\sigma)}\varlib{\sigma(x)}}$ :
    \begin{itemize}[label=$\bullet$]
        \item Si $M = x \in\mathrm{dom}(\sigma)$ alors $M\;\sigma = \sigma(x)$.
        \item Si $M = x \notin\mathrm{dom}(\sigma)$ alors $M\;\sigma = M$.
        \item Si $M = \lambda x.N$ où $x\notin\mathrm{dom}(\sigma)\cup\varlib\sigma$ alors $M\;\sigma = \lambda x.(N\;\sigma)$.
        \item Si $M = N\;P$ alors $M\;\sigma = (N\;\sigma)\;(P\;\sigma)$.
        \item Si $M = \langle\rangle$ alors $M\;\sigma = \langle\rangle$.
        \item Si $M = \langle N,P\rangle$ alors $M\;\sigma = \langle N\;\sigma,P\;\sigma\rangle$.
        \item Si $M = \pi_i\;N$ où $i\in\{1,2\}$ alors $M\;\sigma = \pi_i\;(N\;\sigma)$.
        \item Si $M = \kappa_i\;N$ où $i\in\{1,2\}$ alors $M\;\sigma = \kappa_i\;(N\;\sigma)$.
        \item Si $M = \delta\;(x\mapsto N\mid y\mapsto P)\;Q$ où $x,y\notin\mathrm{dom}(\sigma)\cup\varlib\sigma$ alors $$M\;\sigma = \delta\;(x\mapsto N\;\sigma\mid y\mapsto P\;\sigma)\;(Q\;\sigma)$$
    \end{itemize}

    On appelle par abus de langage substitution une fonction partielle $\sigma : \VV\to\Lambda^{\to\times1+0}$.
\end{defi}

On notera aussi, par convention, $\sigma[N/x]$ pour désigner la substitution $y\mapsto\sigma(y)$ si $y\neq x$ et $x\mapsto u$, avec $x\notin\mathrm{dom}(\sigma)$. On définit aussi pour des variables $x_1,\ldots,x_n$ et des termes $M,M_1,\ldots,M_n$, la substitution $M[M_1/x_1,\ldots,M_n/x_n]$ associée à $\sigma : x_i\mapsto M_i$.

Enfin on note, pour $\mathcal B$ un ensemble de types de base fixé, $\types$ l'ensemble $T_{\to\times 1+0}$ associé.

\subsubsection{Réduction de tête}

L'idée de la réduction de tête est de ne considérer que le sous-terme le plus à gauche du lambda-terme et de le réduire autant que possible, puis s'arrêter. Supposons par exemple que l'on ait un lambda-terme $M = P\;Q$ où $P$ est une fonction, le but sera principalement de réduire $P$ jusqu'à avoir une abstraction $\lambda x.P'$ pour pouvoir ensuite réduire $(\lambda x.P')\;Q\reduc_0 P'[Q/x]$. Notre réduction de tête va suivre cette idée.

\begin{defi}[Réduction de tête]
    On définit la relation $\whr$ sur $\Lambda^{\to\times 1+0}$ par les règles suivantes :
    \begin{center}
        \begin{prooftree}
            \infer0{(\lambda x^{\tau}.M)\;N\whr M[N/x]}
        \end{prooftree}
        \qquad
        \begin{prooftree}
            \infer0{\pi_1\;\langle M,N\rangle\whr M}
        \end{prooftree}
        \qquad
        \begin{prooftree}
            \infer0{\pi_2\;\langle M,N\rangle\whr N}
        \end{prooftree}

        \vspace{0.5cm}
        
        \begin{prooftree}
            \infer0{\delta\;(x\mapsto M\mid y\mapsto N)\;(\kappa_1\;P)\whr M[P/x]}
        \end{prooftree}
        \qquad
        \begin{prooftree}
            \infer0{\delta\;(x\mapsto M\mid y\mapsto N)\;(\kappa_2\;P)\whr N[P/y]}
        \end{prooftree}

        \vspace{0.5cm}

        \begin{prooftree}
            \hypo{M\whr M'}
            \infer1{M\;N\whr M'\;N}
        \end{prooftree}
        \qquad
        \begin{prooftree}
            \hypo{M\whr M'}
            \infer1[$i\in\{1,2\}$]{\pi_i\;M\whr\pi_i\;M'}
        \end{prooftree}
        \qquad
        \begin{prooftree}
            \hypo{M\whr M'}
            \infer1{\delta_\bot M\whr \delta_\bot M'}
        \end{prooftree}

        \vspace{0.5cm}

        \begin{prooftree}
            \hypo{P\whr P'}
            \infer1{\delta\;(x\mapsto M\mid y\mapsto N)\;P\whr \delta\;(x\mapsto M\mid y\mapsto N)\;P'}
        \end{prooftree}
    \end{center}
\end{defi}

Nous utiliserons cependant une définition équivalente utilisant la notion de contexte d'élimination.

\begin{defi}[Contexte d'élimination]
    On définit l'ensemble $\Elim$ des contextes d'élimination de la façon suivante :
    $$E[\;] ::= [\;]\;\Big|\; E[\;]\;M\;\Big|\; \pi_1\;E[\;]\;\Big|\; \pi_2\;E[\;]\;\Big|\; \delta\;(x\mapsto M\mid y\mapsto N)\;E[\;]\;\Big|\; \delta_\bot\;E[\;]$$ où $[\;]$, appelé un trou, est un caractère spécial. On définit $E[M]$ de façon naturelle en remplaçant dans l'expression de $E[\;]$ le caractère $[\;]$ par $M$.
\end{defi}

\begin{rmk}
    Si $E[\;]\in\Elim$ et $F[\;]\in\Elim$ alors $E[F[\;]]\in\Elim$, ce qui nous donne une composition des contextes.
\end{rmk}

\begin{prop}
    Pour $M,N$ deux lambda-termes, on a $M\whr N$ si et seulement s'il existe un contexte $E[\;]\in\Elim$ tel que $M = E[P],N = E[P']$ et $P\reduc_0 P'$.
\end{prop}

\begin{proof}
    Montrons la première implication par induction sur $M\whr N$ :
    \begin{itemize}[label=$\bullet$]
        \item Si $M\whr N$ est obtenu par l'une des cinq première règles alors $M\reduc_0 N$ donc avec le contexte $E[\;] = [\;]$ le résultat est direct.
        \item Si $M = E[M']\;Q$ et $N = E[N']\;Q$ avec $E[\;]\in\Elim$ et $M'\reduc_0 N'$ alors $M = E'[M']$, $N = E'[N']$ avec $E' = E[\;]\;Q$ d'où le résultat.
        \item Si $M = \pi_i\;E[M']$, $N = \pi_i\;E[N']$ où $i\in\{1,2\}$, $E[\;]\in\Elim$ et $M'\reduc_0 N'$ alors $M = E'[M']$ et $N = E'[N']$ avec $E' = \pi_i\;E[\;]$ d'où le résultat.
        \item Si $M = \delta_\bot E[M']$, $N = \delta_\bot E[N']$ avec $E[\;]\in\Elim$ et $M'\reduc_0 N'$ alors $M = E'[M']$ et $N = E'[N']$ pour $E' = \delta_\bot\;E[\;]$.
        \item Si $M = \delta\;(x\mapsto P\mid y\mapsto Q)\;E[M']$ et $N = \delta\;(x\mapsto P\mid y\mapsto Q)\;E[N']$ avec $M'\reduc_0 N'$ alors en posant $E' = \delta\;(x\mapsto P\mid y\mapsto Q)\;E[\;]$ on a le résultat.
    \end{itemize}

    Réciproquement, on montre par induction sur $E[\;]$ le résultat :
    \begin{itemize}[label=$\bullet$]
        \item Pour $E[\;] = [\;]$ cela signifie que $M\reduc_0 N$ auquel cas $M\whr N$.
        \item Pour $E[\;] = E[\;]\;P$ on utilise l'hypothèse d'induction et la $6^\mathrm e$ règle d'induction de $\whr$.
        \item De même pour chaque règle $i$ de construction de $E[\;]$ de notre grammaire il y a une règle correspondante dans $\whr$ qui permet de passer directement de l'hypothèse d'induction au résultat.
    \end{itemize}
\end{proof}

\begin{prop}
    Soit $M$ une forme normale pour $\whr$, typable et close, alors $M$ est de l'une des formes suivantes :
    $$\lambda x.N\qquad \langle\rangle\qquad \langle N,P\rangle \qquad \kappa_1\;N\qquad\kappa_2\;N$$
\end{prop}

\begin{proof}
    On procède par induction sur $M$ :
    \begin{itemize}[label=$\bullet$]
        \item $M$ ne peut être une variable car c'est une formule close.
        \item Si $M = N\;P$ alors comme $M$ est une forme normale, $N$ doit aussi être une forme normale (ou on pourrait réduire à l'intérieur et donc réduire $M$) mais par hypothèse d'induction, $N$ est de l'une des formes citées. Cependant comme $M$ est typable $N$ est de type $\sigma\to\tau$ donc est forcément de la forme $\lambda x.N'$ mais dans ce cas $M \reduc_0 N'[P/x]$ donc $M$ ne peut être de la forme $N\;P$.
        \item Si $M = \lambda x.N$ alors $M$ est de l'une des formes voulues, donc le résultat est vérifié.
        \item Si $M = \langle N,P\rangle$ alors $M$ est de l'une des formes voulues.
        \item Si $M = \pi_i\;N$ pour $i\in\{1,2\}$ alors $N$ est une forme normale donc par hypothèse d'induction $N = \langle P_1,P_2\rangle$ pour des raisons de typages, mais alors $M\reduc_0 P_i$ donc $M$ ne peut être de la forme $\pi_i\;N$.
        \item Si $M = \langle\rangle$ alors $M$ est de l'une des formes voulues.
        \item Si $M = \kappa_i\;N$ pour $i\in\{1,2\}$ alors $M$ est de l'une des formes voulues.
        \item Si $M = \delta\;(x_1\mapsto N_1\mid x_2\mapsto N_2)\;P$ alors par hypothèse $P$ est une forme normale de type $\sigma+\tau$ donc de la forme $\kappa_i\;P'$ donc $M\reduc_0 N_i[P'/x_i]$ donc $M$ ne peut être de cette forme.
        \item Si $M = \delta_\bot\;M'$ alors $M'$ est une forme normale close de type $\voidt$ ce qui est impossible par hypothèse d'induction.
    \end{itemize}
\end{proof}

\begin{exo}
    Montrer que si $M$ est une forme normale pour $\whr$ typable, alors $M$ est de l'une des formes suivantes : $$E[x]\qquad \lambda x.N\qquad \langle\rangle\qquad \langle N,P\rangle\qquad \kappa_1\;N\quad \kappa_2\;N$$ où $E[\;]\in\Elim$ et $x\in\VV$.
\end{exo}

\subsection{Interprétation adéquate}

Avec ces outils en tête, nous pouvons démontrer le théorème de faible normalisation. Cette preuve n'est pas directe car les outils d'induction sur les lambda-termes ou le typage sont limités. A la place, on va chercher une façon d'associer à un type $\tau\in\types$ un ensemble de lambda-termes $\trad\tau$ de telle sorte que $\trad\tau\subseteq P$ avec $P$ la propriété que l'on veut montrer sur les lambda-termes typés. En prouvant alors que $\vdash M : \tau\implies M\in\trad\tau$ on prouve notre propriété.

\begin{defi}[Interprétation adéquate]
    Soit $\trad - : \types\to\mathcal P(\Lambda^{\to\times1+0})$ une interprétation.
    
    \'Etant donnés une substitution $\sigma$ et un contexte de typage $\Gamma$ on note $\sigma\models_{\trad -} \Gamma$ si $\mathrm{dom}(\Gamma)\subseteq \mathrm{dom}(\sigma)$ et si pour tout $(x : \tau) \in\Gamma$, $\sigma(x)\in\trad\tau$.

    On dit que $\trad -$ est acceptable si pour tout $M$ et $\sigma\models_{\trad -}\Gamma$ tels que $\Gamma \vdash M : \tau$ on a $M\;\sigma \in\trad\tau$.
\end{defi}

L'objectif est donc de construire une interprétation adéquate. Pour cela on raisonnera généralement par induction sur la relation de typage. Une première remarque est que si $\sigma\models_{\trad -}\Gamma$ alors pour $M = x$, si $\Gamma\vdash M : \tau$ cela signifie que $M\;\sigma = \sigma(x)\in\trad\tau$. Nous nous concentrons donc sur la définition de $\trad -$ pour les autres constructeurs de type. On ajoutera en plus les interprétations des constantes de types, càd des $\iota\in\mathcal B$, $\unit$ et $\voidt$.

\begin{defi}
    Soient $A,B\in\mathcal P(\Lambda^{\to\times1+0})$, on définit $$A\oto B := \{ M \mid \forall N\in A, M\;N\in B\}$$ \begin{center} et \end{center} $$A\otimes B := \{ M \mid (\pi_1\;M \in A) \land (\pi_2\;M\in B)\}$$
\end{defi}

L'objectif de cette définition est bien sûr de considérer $\trad{A\to B} = \trad A \oto \trad B$ et $\trad{A\times B} = \trad A \otimes \trad B$, et on peut donc déjà essayer de démontrer que si l'interprétation est adéquate à une étape, elle l'est en considérant $\oto$ et $\otimes$ pour construire l'étape suivante. On peut vérifier que le comportement vis à vis des règles d'élimination est le bon, mais pour la règle d'introduction de $\lambda$, on trouve un problème.

On considère \begin{center}
    \begin{prooftree}
        \hypo{\Gamma,x : \kappa\vdash M : \tau}
        \infer1{\Gamma\vdash \lambda x.M : \kappa\to\tau}
    \end{prooftree}
\end{center}
Alors, pour $\sigma\models_{\trad -}\Gamma$ et $x\notin\varlib\sigma$ (quitte à renommer $x$), comme $(\lambda x.M)\;\sigma = \lambda x.M\;\sigma$, on doit montrer que si $N\in\trad\kappa$ alors $(\lambda x.M\;\sigma)\;N\in\trad\tau$. Par hypothèse d'induction, on sait que $(M\;\sigma)[N/x] = M\;\sigma[N/x]$ pour $\sigma\models_{\trad -}\Gamma$, et on voudrait donc en déduire que $(\lambda x.M\;\sigma)\;N$ : ce passage étant une application de réduction $\reduc_0$, on va demander une propriété de stabilité (remarquons qu'ici la stabilité est dans l'autre sens). Cependant il serait vain de ne demander que la stabilité par $\reduc_0$ car cela marchera pour une étape, mais ne nous permettra pas de fonctionner lors d'une induction sur les constructeurs de type. On va donc demander une stabilité pour $\whr$ :

\begin{defi}[Stabilité par expansion de tête]
    On dit qu'un ensemble $A\subseteq \Lambda^{\to\times1+0}$ est stable par expansion de tête si pour tout $N\in A$ et $M$ tel que $M\whr N$, on a $M\in A$. On va noter $\WHE$ cette propriété.
\end{defi}

\begin{exo}
    Montrer que si $A$ vérifie $\WHE$ alors pour tout $M$ et $N\in A$ tels que $M\whr^* N$, on a $M\in A$.
\end{exo}

\subsubsection{Une première interprétation}

Nous allons définir une première interprétation permettant de donner un théorème important : la propriété de la disjonction, qui dit que si $M : \tau+\kappa$ alors $M\whr\kappa_i\;N$ pour un certain $N$ et un certain $i\in\{1,2\}$.

\begin{defi}[Interprétation positive]
    On définit $\boxplus$, $\boldsymbol 1$ et $\boldsymbol 0$ de la façon suivante :
    $$A\boxplus B := \{ M\mid (\exists N, (M\whr^* \kappa_1\;N)\land (N\in A))\lor (\exists N, (M\whr^* \kappa_2\;N)\land(N\in B))\}$$
    $$\boldsymbol 1 := \{ M\mid M\whr^* \langle\rangle\}$$
    $$\boldsymbol 0 := \varnothing$$
\end{defi}

Prouvons un lemme avant de prouver le théorème important de cette sous-partie :

\begin{lem}
    Soient $A,B,C\in\mathcal P(\Lambda^{\to\times1+0})$ où $C$ vérifie $\WHE$. Soient $M,N_1,N_2$ tels que $M\in A\boxplus B, N_1[Q/x_1]\in C$ et $N_2[Q/x_2]\in C$ pour tout $Q\in A\boxplus B$. Alors $$\delta \;(x_1\mapsto N_1\mid x_2\mapsto N_2)\;M\in C$$
\end{lem}

\begin{proof}
    Par définition de $M\in A\boxplus B$, on trouve $M'$ tel que $M\whr^* \kappa_i\;M'$ avec $i\in\{1,2\}$. Cela signifie de plus que $\delta\;(x_1\mapsto N_1\mid x_2\mapsto N_2)\;M \whr^* \delta\;(x_1\mapsto N_1\mid x_2\mapsto N_2)\;(\kappa_i\;M')$ donc que $$\delta\;(x\mapsto N\mid y\mapsto P)\;M \whr^* N_i[M'/x_i]$$ or $N_i[M'/x_i]\in C$ donc comme $C$ vérifie $\WHE$, $$\delta \;(x_1\mapsto N_1\mid x_2\mapsto N_2)\;M\in C$$
\end{proof}

On peut alors donner le résultat menant à la propriété de la disjonction :

\begin{them}
    Soit $\trad -$ une interprétation. Si :
    \begin{itemize}[label=$\bullet$]
        \item $\trad\iota$ vérifie $\WHE$ pour tout $\iota\in\mathcal B$.
        \item $\trad\voidt = \boldsymbol 0$ et $\trad\unit = \boldsymbol 1$.
        \item $\trad{\sigma\to\kappa}=\trad\sigma\oto\trad\kappa$.
        \item $\trad{\sigma\times\kappa}=\trad\sigma\otimes\trad\kappa$.
        \item $\trad{\sigma+\kappa}=\trad\sigma\boxplus\trad\kappa$.
    \end{itemize}

    Alors :
    \begin{itemize}[label=$\bullet$]
        \item Pour tout type $\tau\in\types$, $\trad\tau$ vérifie $\WHE$.
        \item $\trad-$ est une interprétation adéquate.
    \end{itemize}
\end{them}

\begin{proof}
    On montre par induction sur $\tau\in\types$ que $\trad\tau$ vérifie $\WHE$ :
    \begin{itemize}[label=$\bullet$]
        \item Par hypothèse $\trad\iota$ vérifie $\WHE$ pour $\iota\in\mathcal B$.
        \item On vérifie que $\boldsymbol 0$ et $\boldsymbol 1$ vérifient $\WHE$ : pour $\boldsymbol 0$ on quantifie sur l'ensemble vide, et pour $\boldsymbol 1$, soit $N$ tel que $N\whr^*\langle\rangle$, alors pour $M$ tel que $M\whr^* N$, on a $M\whr^* \langle\rangle$ d'où le résultat.
        \item Si $\trad\tau$ et $\trad\kappa$ vérifient $\WHE$, alors $\trad\tau\oto\trad\kappa$ vérifie aussi $\WHE$. En effet, si $N\in\trad\tau\oto\trad\kappa$ et $M\whr N$ alors pour $P\in \trad\tau$, on a $N\;P\in\trad\kappa$ donc, comme $\trad\kappa$ vérifie $\WHE$ et $M\;P\whr N\;P$, on en déduit que $M\;P\in\trad\kappa$, donc $M\in\trad\tau\oto\trad\kappa$.
        \item Si $\trad{\tau_1}$ et $\trad{\tau_2}$ vérifient $\WHE$, soit $N\in\trad{\tau_1}\otimes\trad{\tau_2}$ et $M\whr N$. Alors $\pi_i\;N\in\trad{\tau_i}$ et comme $\pi_i\;M\whr\pi_i\;N$ on en déduit que $\pi_i\;M\in\trad{\tau_i}$ donc $M\in\trad{\tau_1}\otimes\trad{\tau_2}$.
        \item Si $\trad{\tau_1}$ et $\trad{\tau_2}$ vérifient $\WHE$, soit $N\in\trad{\tau_1}\boxplus\trad{\tau_2}$ et $M\whr N$. On trouve $i\in\{1,2\}$ tel que $N\whr^* \kappa_i\;P$ avec $P\in\trad{\tau_i}$ mais alors $M\whr^* \kappa_i\;P$ donc $M\in\trad{\tau_1}\boxplus\trad{\tau_2}$.
    \end{itemize}
    On en déduit donc que toute interprétation vérifie $\WHE$.

    Montrons maintenant que $\trad -$ est adéquate. On considère $\sigma\models_{\trad -}\Gamma$ et on raisonne par induction sur $\Gamma\vdash M : \tau$ pour montrer que $M\;\sigma\in\trad\tau$ :
    \begin{itemize}[label=$\bullet$]
        \item Comme on l'a dit, par hypothèse, la règle pour les variables est vérifiée.
        \item Si $M\;\sigma\in\trad\tau$ et $y\notin\varlib{\sigma}$ alors la conclusion tient toujours en prenant $\Gamma,y : \kappa\vdash M$.
        \item Si $\Gamma,x : \kappa\vdash M : \tau$ et $M\;\sigma\in\trad\tau$, alors pour tout $N\in\trad\kappa$ on sait que $M\;\sigma[N/x]\in\trad{\tau}$ donc $M\in\trad{\kappa\to\tau}$.
        \item Si $M = N\;P$, avec $\Gamma\vdash N : \kappa\to\tau$ et $\Gamma\vdash P : \kappa$, où $N\;\sigma\in\trad{\kappa\to\tau}$ et $P\;\sigma\in\trad\kappa$, alors par hypothèse $M\;\sigma = (N\;\sigma)\;(P\;\sigma)\in\trad{\tau}$.
        \item Si $M = \langle N_1,N_2\rangle$ avec $\Gamma\vdash N_1 : \kappa_1$, $\Gamma\vdash N_2 : \kappa_2$ et $\tau = \kappa_1\times\kappa_2$, alors $(\pi_i\;M\;)\sigma \whr N_i\;\sigma \in\kappa_i$ donc $M\;\sigma\in\trad{\kappa_1\times\kappa_2}$.
        \item Si $M = \pi_i\;N$ avec $i\in\{1,2\}$ et $\Gamma\vdash N : \langle \kappa_1,\kappa_2\rangle$ alors par définition de $N\in\trad{\kappa_1\times\kappa_2}$ on en déduit que $M \in\trad{\tau}$ et $\tau = \kappa_i$ pour un certain $i$.
        \item Si $M = \langle\rangle$ alors $M\;\sigma = \langle\rangle \in\trad\unit$.
        \item Si $M = \kappa_i\;N$, $i\in\{1,2\}$ avec $\Gamma\vdash N : \kappa$ et $\tau = \kappa+\kappa'$ alors par définition $M\in\trad{\kappa+\kappa'}$.
        \item Si $M = \delta\;(x_1\mapsto N_1\mid x_2\mapsto N_2)\;P$ alors par le lemme précédent cela signifie que $M\in\trad{\tau}$ pour $\tau$ le type de $N_i[P/x_i]$.
        \item Si $M = \delta_\bot N$ alors $N\in\varnothing$ donc tout est vrai, la propriété en particulier.
    \end{itemize}
    Donc $\trad -$ est adéquate.
\end{proof}

\begin{cor}[Propriété de la disjonction]
    On en déduit le corollaire suivant : si $M$ est un terme typable dans l'ensemble vide alors 
    \begin{itemize}[label=$\bullet$]
        \item si $\vdash M : \tau_1+\tau_2$ alors il existe $N$ tel que $M\whr \kappa_i\;N$ et $\vdash N : \tau_i$ pour un certain $i\in\{1,2\}$.
        \item $\vdash M : \voidt$ est impossible.
    \end{itemize}
\end{cor}

\begin{proof}
    On construit l'interprétation comme proposé dans le théorème précédent, en associant à $\iota$ l'ensemble $\trad\iota = \{ M \mid \;\vdash M : \iota\}$. Par préservation du typage, si $N\whr M$ et $M\in\trad\iota$ on en déduit que $N\in\trad\iota$, donc l'interprétation est adéquate. On en déduit directement que $\vdash M : \voidt$ est impossible puisque cela signifie $M\in\varnothing$. Si $\vdash M : \tau+\tau_2$ alors $M\in \trad{\tau_1}\boxplus\trad{\tau_2}$ donc on trouve par définition de $\boxplus$ un $i\in\{1,2\}$ et un $N$ tels que $M\whr^* \kappa_i\;N$ et $N\in\trad{\tau_i}$. De plus $\vdash \kappa_i\;N : \tau_1+\tau_2$ donc $\vdash N : \tau_i$ par inversion sur le typage.
\end{proof}

\begin{rmk}
    On n'a pas utilisé de $\sigma$ dans ce cas car $\sigma = \varnothing\models_{\trad -}\varnothing$ et on considère uniquement le typage dans $\varnothing$. 
\end{rmk}

\subsection{Normalisation faible}

On va considérer la normalisation faible pour $\whr$ dans toute cette partie. On note $\WN$ l'ensemble des termes faiblement normalisables pour $\whr$. Un premier résultat est que si $N\in\WN$ et $M\whr N$ alors $M\in\WN$. Donnons un autre résultat :

\begin{lem}
    Soit $A\subseteq \WN$, l'ensemble $\{M\mid \exists N \in A, M\whr^* N\}$ est le plus petit ensemble contenant $A$ et vérifiant $\WHE$.
\end{lem}

\begin{proof}
    Par définition si $M\in A$ alors $M\whr^* M$ donc $M$ l'ensemble décrit contient $A$. Si pour un certain $N$ il existe $P\in A$ tel que $N\whr^* P$ et que $M\whr N$ alors $M\whr^* P$ donc l'ensemble vérifie $\WHE$. De plus un ensemble contenant $A$ et vérifiant $\WHE$ contient tous les $M$ tels que $M\whr^* N$ pour $N\in A$ d'après un résultat précédent.
\end{proof}

\begin{lem}
    Soient $A,B\subseteq\WN$, alors $A\otimes B\subseteq \WN$.
\end{lem}

\begin{proof}
    On considère $M\in A\otimes B$. Si $M = \langle N,P\rangle$ alors $M$ est une forme normale et donc $M\in\WN$. Sinon, alors $\pi_1\;M\in A\subseteq\WN$ donc $\pi_1\;M$ est sous forme normale (et donc $M$ aussi) ou on trouve $N\in\WN$ tel que $\pi_1\;M\whr N$ mais par inversion sur $\whr$ on en déduit que $N = \pi_1\;N'$ et $M\whr N'$, donc $N'\in\WN$, donc $M\in\WN$.
\end{proof}

Cependant un nouveau problème se présente. Supposons qu'on veuille définir $\trad\voidt = \boldsymbol 0$ comme précédemment, alors $\boldsymbol 0 \oto A = \{M\mid \forall N\in\varnothing, M\;N\in A\}$ qui correspond à l'ensemble $\Lambda^{\to\times 1+0}$ tout entier, et est donc trop large pour considérer que $\trad\voidt \subseteq \WN$. On veut donc imposer que tous les ensembles $\trad\tau$ soient non vides. \'Evidemment, cela signifie qu'il nous faut modifier notre traduction de $\voidt$, mais plus généralement cette condition est compatible avec les traductions négatives (qui se basent sur les éliminateurs comme les projections ou les applications) mais pas avec les traductions positives (qui se basent sur les constructeurs comme les injections). On veut donc une traduction de $\unit$, $\voidt$ et $\tau+\tau'$ qui se basent sur les éliminateurs.

Un dernier obstacle à contourner est que si l'on regarde l'élimination de $\voidt$ par exemple, on a un type dont on ne connaît rien qui est mentionné, et de même pour l'élimination de $+$. On voudrait donc quantifier sur tous les $\trad \tau$ mais comme nous cherchons à définir $\trad\tau$, la définition serait circulaire. De plus, considérer juste l'ensemble des parties $A\subseteq\Lambda^{\to\times1+0}$ est truc large, donc nous allons définir des parties plus restreintes vérifiant nos conditions dans lesquelles on veut travailler, que l'on appelle des ensembles saturés.

\begin{defi}[Ensemble saturé]
    On définit $\SATW\subseteq \mathcal P(\WN)$ l'ensemble des parties dites $\WN$-saturées, qui est tel que $A\in \SATW$ si et seulement \begin{itemize}[label=$\bullet$]\item si $E[x]\in A$ pour tout $E[\;]\in\Elim$ et $x\in\VV$ \item $M\in A$ si $M\whr N$ et $N\in A$\end{itemize}
\end{defi}

\begin{rmk}
    La deuxième condition est exactement que $A$ vérifie $\WHE$.
\end{rmk}

\begin{prop}
    L'ensemble $\SATW$ est un treillis complet pour l'inclusion dont le majorant est $\top := \WN$ et le minorant est $\bot := \{M\mid \exists E[\;]\in\Elim,\exists x\in\VV, M\whr^* E[x]\}$.
\end{prop}

\begin{proof}
    Il est évident que $\WN$ est le majorant pour l'inclusion de $\SATW$. On sait de plus que l'ensemble $\{E[x]\mid E[\;]\in\Elim,x\in\VV\}$ est contenu dans tous les éléments de $\SATW$, donc le plus petit ensemble contenant cet ensemble et vérifiant $\WHE$ est inclus dans tous les éléments de $\SATW$. Cet ensemble est $\{M\mid \exists E[\;]\in\Elim,x\in\VV,M\whr^* E[x]\}$ donc $\bot$ est effectivement le minorant de tous les éléments de $\SATW$ pour l'inclusion. Soit $\mathcal A\subseteq \SATW$ un ensemble de parties $\WN$-saturées, montrons que $\bigcap\mathcal A\in\SATW$. On sait que pour tout $E[\;]\in\Elim,x\in\VV$, $E[x]\in A$ pour $A\in\mathcal A$ et de plus si $N\in\bigcap\mathcal A$ alors pour chaque $A\in\mathcal A$, si $M\whr N$ alors $M\in A$, donc au total $M\in\bigcap\mathcal A$, donc $\bigcap\mathcal A\in\SATW$. Pour $\bigcup\mathcal A$, le fait de contenir les $E[x]$ découle déjà du résultat pour l'intersection, et si $N\in\mathcal A$ et $M\whr N$ alors on trouve $A\in\mathcal A$ tel que $N\in A$ et comme $A$ vérifie $\WHE$, $M\in A$ donc $M\in\bigcup\mathcal A$.
\end{proof}

On définit alors les interprétations négatives sur $\SATW$.

\begin{prop}
    Une autre écriture de $\bot$ est $$\bot = \{M\mid \forall A \in\SATW, \delta_\bot\;M\in A\}$$
\end{prop}

\begin{proof}
    Si $M\in\bot$ alors on trouve $E[x]$ tel que $M\whr^* E[x]$ et $E[x]\in A$ pour tout $A\in\SATW$ donc $M\in A$ et $\delta_\bot\;M\in A$ en considérant $E'[x] = \delta_\bot\;E[x]$. Réciproquement, si $\delta_\bot\;M\in A$ pour tout $A\in\SATW$, alors en particulier $\delta_\bot\;M\in \bot$ donc on trouve $E[x]$ tel que $\delta_\bot\;M\whr^* E[x]$ et en prenant $E'[x] = \delta_\bot\;E[x]$ on trouve donc $E'[x]$ tel que $M\;\whr^* E'[x]$ par inversion sur $\whr$, donc $M\in\bot$.
\end{proof}

\begin{defi}
    Soient $A,B\in\SATW$. On définit pour $C\in\SATW$ l'ensemble intermédiaire $\Lambda_{A,C,x} := \{N\mid \forall P\in A,N[P/x]\in C\}$ puis $$A\oplus_\WN B := \{M\mid \forall C\in\SATW,\forall N_1\in\Lambda_{A,C,x_1},\forall N_2\in\Lambda_{B,C,x_2},\delta\;(x_1\mapsto N_1\mid x_2\mapsto N_2)\;M\in C\}$$
\end{defi}

On vérifie alors que $\SATW$ est bien stable par les relations attendues.

\begin{lem}
    Soient $A,B\in\SATW$, alors les ensembles suivants sont dans $\SATW$ : $$A\oto B\qquad A\otimes B\qquad A\oplus_\WN B$$
\end{lem}

\begin{proof}
    Pour $E[x]$ avec $E[\;]\in\Elim,x\in\VV$, si l'on prend $M\in A$ alors $E[x]\;M = E'[x]$ avec $E'[\;] = E[\;]\;M$ donc comme $B\in\SATW$, $E[x]\;M\in B$, d'où $E[x]\in A\oto B$. Si $N\in A\oto B$ et $M\whr N$ alors pour $P\in A$, $M\;P\whr N\;P$ et $N\;P\in B$ donc par $\WHE$ on en déduit que $M\;P\in B$, donc $M\in A\oto B$, donc $A\oto B$ vérifie $\WHE$.

    Soit $E[\;]\in\Elim,x\in\VV$, on définit $E' := \pi_i\;E[\;]$ et comme $A$ et $B$ contiennent tous les $E'[x]$, cela signifie que $E[x]\in A\otimes B$. Si $N\in A\otimes B$ et $M\whr N$ alors comme $\pi_1\; N\in A$ et $\pi_2\;N\in B$, cela signifie que $\pi_1\;M\in A$ et $\pi_2\;M\in B$ par $\WHE$, donc $M\in A\otimes B$. Donc $A\otimes B$ vérifie $\WHE$.

    Pour $E[\;]\in\Elim,x\in\VV$, $C\in\SATW$ et $N_1\in \Lambda_{A,C,x_1},N_2\in\Lambda_{B,C,x_2}$, en posant comme nouveau contexte $E'[\;] = \delta\;(x_1\mapsto N_1\mid x_2\mapsto N_2)\;E[\;]$ on trouve que $E'[x]\in C$ car $C\in\SATW$. Si $N\in A\oplus_\WN B$ et $M\whr N$ alors soient $C,N_1,N_2$ définis comme précédemment. On sait de plus que $\delta\;(x_1\mapsto N_1\mid x_2\mapsto N_2)\; N\in C$ donc par $\WHE$ on en déduit que $\delta\;(x_1\mapsto N_1\mid x_2\mapsto N_2)\; M\in C$, donc $M\in A\oplus_{\WN} B$. Donc $A\oplus_\WN B$ vérifie $\WHE$.
\end{proof}

On en vient maintenant au théorème analogue à celui de la section précédente.

\begin{them}
    Soit $\trad -$ une interprétation telle que
    \begin{itemize}[label=$\bullet$]
        \item Pour tout $\iota\in\mathcal B$, $\trad\iota \in\SATW$.
        \item $\trad\voidt = \bot$ et $\trad\unit = \top$.
        \item Pour tout type $\tau,\tau'$, $$\trad{\tau\to\tau'} = \trad\tau\oto\trad{\tau'}\qquad \trad{\tau\times\tau'} = \trad\tau\otimes\trad{\tau'}\qquad \trad{\tau +\tau'} = \trad\tau\oplus_\WN \trad{\tau'}$$
    \end{itemize}
    Alors 
    \begin{itemize}[label=$\bullet$]
        \item Pour tout $\tau\in\types$, $\trad\tau\in\SATW$.
        \item $\trad -$ est adéquate.
    \end{itemize}
\end{them}

\begin{proof}
    Le fait que $\trad\tau\in\SATW$ se déduit directement des lemmes précédents en effectuant une induction sur $\types$.

    Soit $\sigma\models_{\trad -} \Gamma$ pour une substitution $\sigma$ et un environnement $\Gamma$, on va montrer par induction sur $\Gamma\vdash M : \tau$ que pour tout $M\in\Lambda^{\to\times 1+0}, \tau\in\types$ on a $\Gamma\vdash M : \tau \implies M \in\trad\tau$ :
    \begin{itemize}[label=$\bullet$]
        \item Les $6$ premières règles d'induction se traitent strictement de la même façon puisque les définitions sont identiques.
        \item Si $M = \langle\rangle$ alors $M\;\sigma = \langle\rangle\in\WN = \trad\unit$.
        \item Si $M = \kappa_i\;N, i\in\{1,2\}$ et $\Gamma\vdash M : \tau_1+\tau_2$ alors pour $C\in\SATW$, $N_1\in\Lambda_{\trad{\tau_1},C,x}$ et $N_2\in\Lambda_{\trad{\tau_2},C,x}$ on a $$\delta\;(x_1\mapsto N_1\mid x_2\mapsto N_2)\;M\whr N_i[N/x_i]\in C$$ donc par $\WHE$ on en déduit que $\delta\;(x_1\mapsto N_1\mid x_2\mapsto N_2)\;M\in C$, d'où $M\in \trad{\tau_1}\oplus_\WN \trad{\tau_2}$.
        \item Si $M = \delta_\bot N$, $\Gamma\vdash M : \tau$ avec $N\in\trad\voidt$ alors par notre définition équivalente de $\bot = \trad\voidt$ on en déduit que $M\in \trad{\tau}$.
    \end{itemize}
    Donc $\trad -$ est adéquate.
\end{proof}

\begin{cor}
    Si on trouve $\Gamma$ tel que $\Gamma\vdash M : \tau$ pour un lambda-terme $M$ et un type $\tau$, alors $M\in \WN$.
\end{cor}

\begin{proof}
    Comme $\trad\tau\subseteq\WN$ et que $\trad -$ est adéquate, cela signifie que $M\;\sigma \in \trad\tau$ pour $\sigma\models_{\trad -}\Gamma$, donc $M\;\sigma$ est normalisable. En prenant la fonction $\sigma$ qui à $x$ associe $x^\tau$ avec $\tau$ tel que $(x : \tau)\in \Gamma$, on en déduit que $M\;\sigma = M$ et donc que $M$ est normalisable.
\end{proof}

\subsection{Normalisation forte}

\subsubsection{Notion de terme fortement normalisant}

Commençons par nous placer dans le cas d'un système de réécriture $(E,\to)$. On va commencer par montrer la validité d'une définition inductive pour traduire ce qu'est un terme fortement normalisable.

\begin{defi}[Ensemble $\SN$]
    Soit $(E,\to)$, on définit $\SN$ par la partie de $E$ constituée des éléments $x\in E$ tels qu'il n'existe pas de suite infinie $(x_i)_{i\in\nat}$ telle que $x_0 = x$ et $x_i\reduc x_{i+1}$ pour tout $i\in\nat$.
\end{defi}

\begin{prop}
    L'ensemble $\SN'$ défini par induction comme la plus petite partie de $E$ stable par la règle suivante :
    \begin{center}
        \begin{prooftree}
            \hypo{\forall y\in E,x\to y\implies y\in \SN'}
            \infer1{x\in \SN'}
        \end{prooftree}
    \end{center}
    Et $\SN'=\SN$.
\end{prop}

\begin{proof}
    On raisonne par induction sur $\SN'$. Soit $x\in E$, on suppose que pour tout $y\in E$ tel que $x\to y$, il n'existe pas de $(y_i)_{i\in\nat}$ avec $y_0 = y$ et $\forall i\in\nat, y_i \to y_{i+1}$. Supposons qu'il existe une suite $(x_i)_{i\in\nat}$ telle que $x_0 = x$ et $\forall i\in\nat, x_i \to x_{i+1}$. En définissant $y_i = x_{i+1}$ on trouve une contradiction, donc une telle suite $(x_i)$ n'existe pas. Donc $\SN'\subseteq \SN$.

    Par contraposée, si $x\notin\SN'$ alors on peut trouver un élément $y\notin\SN'$. Dans ce cas on peut trouver par l'axiome du choix dépendant une suite $(x_i)$ telle que $\forall i\in\nat, x_i\notin\SN'$. Donc $\SN\subseteq\SN'$. D'où le résultat.
\end{proof}

L'ensemble $\SN$ désignera maintenant l'ensemble $\SN$ correspondant au système de réécriture $(\Lambda^{\to\times1+0},\reduc)$.

\subsubsection{Passer de la réduction de tête à la $\beta$-réduction}

Encore une fois, nous allons adapter notre interprétation adéquate en changeant l'invariant qui définira nos ensembles saturés. Pour adapter le fait de contenir $E[x]$, il nous suffit de considérer $E[\;]\in\Elim\cap\SN$. Cependant, pour l'expansion de tête, s'assurer d'avoir $(\lambda x.M\;\sigma)\;N\in A$ si $M\;\sigma[N/x]\in A$, est moins direct. En effet, en imaginant par exemple $(\lambda x.\lambda y.y)\;\Omega$ on voit qu'un terme non normalisable peut disparaître lors de la réduction. Cependant, si $N$ est lui-même dans $\SN$, alors $(\lambda x.M\;\sigma)\;N$ sera bien dans $\SN$.

Pour faciliter les notations, nous allons définir la notion de réduction de tuples.

\begin{defi}[Réduction de tuples]
    Soient $M_1,\ldots,M_n$ des termes. On note $(M_1,\ldots,M_n)\reduc (N_1,\ldots,N_n)$ s'il existe $i\in\{1,\ldots,n\}$ tel que $M_i\reduc N_i$ et pour $j\neq i$ on a $M_j=N_j$.
\end{defi}

Nous allons maintenant montrer un résultat intermédiaire pour montrer ensuite la propriété $\WHE$ dans les cas qui nous intéressent.

\begin{lem}[Standardisation faible]
    Soit $E[\;]\in\Elim$. Alors :
    \begin{itemize}[label=$\bullet$]
        \item Si $E[(\lambda x.M)\;N]\reduc P$, alors soit $P = E[M[N/x]]$, soit $P = E'[(\lambda x.M')\;N']$ où $(E[\;],M,N)\reduc (E'[\;],M',N')$.
        \item Pour $i\in\{1,2\}$, si $E[\pi_i\;\langle M_1,M_2\rangle]\reduc N$, alors soit $N = E[M_i]$ soit $N = E'[\pi_i\;\langle M_1',M_2'\rangle]$ où $(E[\;],M_1,M_2)\reduc (E'[\;],M_1',M_2)$.
        \item Soit $i\in\{1,2\}$, si $E[\delta\;(x_1\mapsto M_1\mid x_2\mapsto M_2)\;(\kappa_i\;N)]\reduc P$ alors on a deux possibilités : $P = E[M_i[N/x_i]]$ ou $P = E'[\delta\;(x_1\mapsto M_1'\mid x_2\mapsto M_2')\;(\kappa_i\;N')]$ avec $(E[\;],M_1,M_2,N)\reduc (E'[\;],M_1',M_2',N')$.
    \end{itemize}
\end{lem}

\begin{proof}
    On montre d'abord le cas sans $E[\;]$ :
    \begin{itemize}[label=$\bullet$]
        \item Si $(\lambda x.M)\;N\reduc P$ alors par inversion soit $P = M[N/x]$, soit $P = (\lambda x.M')\;N'$ avec $(M,N)\reduc (M',N')$.
        \item Si $\pi_i\;\langle M_1,M_2\rangle\reduc N$ alors par inversion soit $N = M_i$ soit $N = \pi_i\;\langle M_1',M_2'\rangle$ avec $(M_1,M_2)\reduc (M_1',M_2')$.
        \item Si $\delta\;(x_1\mapsto M_1\mid x_2\mapsto M_2)\;(\kappa_i\;N)\reduc P$ alors par inversion, soit $P = M_i[P/x_i]$ soit $P = \delta\;(x_1\mapsto M'_1\mid x_2\mapsto M'_2)\;(\kappa_i\;N')$ avec $(M_1,M_2,N)\reduc (M'_1,M'_2,N')$.
    \end{itemize}
    
    On raisonne maintenant par induction sur $E[\;]$ :
    \begin{itemize}[label=$\bullet$]
        \item Si $E = [\;]$ alors le résultat découle de ce qu'on a prouvé juste avant.
        \item Supposons que la propriété soit vraie pour $E[\;]$. On considère alors $E'[\;] = E[\;]\;M$ et un $N$ entrant dans l'un des cas précédemment traités. Dans chaque cas, si $E'[N]\reduc P$, par inversion sur $\reduc$, on trouve au choix que $E[N] \reduc N'$ et auquel cas par hypothèse d'induction on en déduit le résultat, ou que $M\reduc M'$ ce qui nous donne que $(E'[\;],N)\reduc (E''[\;],N)$ avec $E'' = E[\;]\;M'$.
    \end{itemize}
\end{proof}

\begin{exo}
    Traiter les autres cas d'induction.
\end{exo}

On en déduit le résultat de stabilité :

\begin{prop}[Stabilité par réduction de tête]
    Soit $E[\;]\in\Elim$. Alors
    \begin{itemize}[label=$\bullet$]
        \item Si $E[M[N/x]]\in \SN$ et $N\in\SN$ alors $E[(\lambda x.M)\;N]\in\SN$.
        \item Soit $i\in\{1,2\}$. Si $E[M_i]\in\SN$ et $M_{3-i}\in\SN$ alors $E[\pi_i\;\langle M_1,M_2\rangle]\in\SN$.
        \item Soit $i\in\{1,2\}$. Si $E[M_i[N/x_i]]$ et $N,M_{3-i}\in\SN$ alors $E[\delta\;(x_1\mapsto M_1\mid x_2\mapsto M_2)\;(\kappa_i\;N)]\in\SN$.
    \end{itemize}
\end{prop}

\begin{proof}
    On va montrer par induction sur $(E[\;],M,N)\SN$ que si $E[M[N/x]]\in\SN$ et $N\in\SN$ alors $E[(\lambda x.M)\;N]\in\SN$. Supposons que pour tout $(E'[\;],M',N')$ tel que $(E[\;],M,N)\reduc (E'[\;],M',N')$ alors si $E'[M'[N'/x]]\in\SN$ et $N'\in\SN$ alors $E[(\lambda x.M')\;N']\in\SN$. On suppose de plus que $E[M[N/x]]\in\SN$ et $N\in\SN$. Alors, si $E[(\lambda x.M)\;N]\reduc P$ il y a deux cas possibles : soit $P = M[N/x]$ et dans ce cas $P\in\SN$, soit $(E[\;],M,N)\reduc (E'[\;],M',N')$ et $P = E'[(\lambda x.M')\;N']$. Dans ce cas, on sait que $E'[M'[N/x]]\in\SN$ car $E[M[N/x]]\reduc^* E'[M'[N'/x]]$ et $E[M[N/x]]\in\SN$. De plus $N\reduc^* N'$ donc $N'\in\SN$ d'où $P = E[(\lambda x.M')\;N']\in\SN$. Ainsi par induction pour tout $(E[\;],M,N)\in\SN$, la propriété est vraie. Enfin, on remarque que si $E[M[N/x]]\in\SN$ alors en particulier $(E[\;],M,N)\in\SN$ car une réduction infinie depuis $(E[\;],M,N)$ serait infinie depuis $E[\;]$ ou depuis $M$ (car $N\in\SN$) et pourrait donc se simuler par compatibilité de $\reduc$ depuis $E[M[N/x]]$.

    On traite les deux autres cas de façon analogue.
\end{proof}

\begin{exo}
    Rédiger les deux autres cas de la preuve.
\end{exo}

\subsubsection{Interprétation adéquate}

On peut maintenant adapter notre interprétation. Nous allons donner la notion d'ensemble $\SN$-saturé, qui est une adaptation de $\SATW$ comme attendu.

\begin{defi}
    On définit $\SATS$ comme la partie de $\mathcal P(\SN)$ contenant tous et uniquement les ensembles $A$ tels que :
    \begin{itemize}[label=$\bullet$]
        \item Pour tout $x\in\VV$ et $E[\;]\in\Elim\cap\SN$, $E[x]\in A$.
        \item Si $N\in\SN$ et $E[M[N/x]]\in A$ alors $E[(\lambda x.M)\;N]\in A$.
        \item Pour tous $i\in\{1,2\}$, si $E[M_i]\in A$ et $M_{3-i}\in \SN$ alors $E[\pi_i\;\langle M_1,M_2\rangle]\in A$.
        \item Pour tous $i\in\{1,2\}$, si $E[M_i[N/x_i]]\in A$ et $N,M_{3-i}\in\SN$ alors $$E[\delta\;(x_1\mapsto M_1\mid x_2\mapsto M_2)\;(\kappa_i\;N)]\in A$$
    \end{itemize}
\end{defi}

\begin{exo}
    Montrer que $\SATS$ est un treillis complet pour l'inclusion avec pour bornes supérieure et inférieure respectivement $$\top := \SN$$ \begin{center} et \end{center} $$\bot := \{M\mid \exists E[\;]\in\Elim, \exists x\in\VV, M\reduc^* E[x]\}$$
\end{exo}

On adapte enfin $\oplus$.

\begin{defi}[Somme dans $\SATS$]
    On définit l'opération $\oplus_\SN$, pour $A\in\SATS$ et $B\in\SATS$, par : $$A\oplus_\SN B := \{ M \mid \forall C\in\SATS,\forall N_1\in\Lambda_{A,C,x_1},\forall N_2\in\Lambda_{B,C,x_2}, \delta\;(x_1\mapsto M_1\mid x_2\mapsto M_2)\;M\in C\}$$
\end{defi}

\begin{lem}
    Soient $A,B\in \SATS$, alors les ensembles suivants sont dans $\SATS$ : $$A\oto B\qquad A\otimes B\qquad A\oplus_\SN B$$
\end{lem}

\begin{proof}
    On remarque que si $E[\;]\in\Elim\cap\SN$ et $x\in\VV$, alors pour $N\in A$ on a $E[\;]\;N\in B$, et $\pi_1\;E[x]\in A, \pi_2\;E[x]\in B$ et si $C\in \SATS$ et $N_1,N_2$ définis comme dans la définition de $\oplus_\SN$, alors $\delta\;(x_1\mapsto M_1\mid x_2\mapsto M_2)\;E[x]\in C$ car tous les ensembles dans $\SATS$ contiennent $E[x]$ pour $E[\;]\in\Elim\cap\SN$, et nous n'ajoutons que des termes dans $\SN$ donc on obtient bien un $E'[x]$ pour $E'[\;]\in\Elim\cap\SN$.

    Supposons que $N\in\SN$ et $E[M[N/x]]\in A\oto B$, alors soit $P\in A$. D'après ce qu'on a dit, $E[M[N/x]]\;P\in B$, or $B\in\SATS$ et $N\in\SN$ donc $E[(\lambda x.M)\;N]\;P\in B$. De plus par le lemme précédent, comme $N\in\SN$ et $E[M[N/x]]\;P\in \SN$ on en déduit que $E[M[N/x]]\in\SN$. Donc $E[(\lambda x.M)\;N]\in B$.
    
    On traite les autres cas de façon similaire.
\end{proof}

\begin{exo}
    Démontrer les cas restants.
\end{exo}

On peut alors donner le théorème d'adéquation.

\begin{them}
    Soit $\trad -$ une interprétation telle que :
    \begin{itemize}[label=$\bullet$]
        \item Pour tout $\iota\in\mathcal B$, $\trad\iota\in\SATS$.
        \item $\trad\unit = \top$ et $\trad\voidt = \bot$.
        \item Pour tout $\tau,\tau'\in\types$, $\trad{\tau\to\tau'}=\trad\tau\oto\trad{\tau'}$.
        \item Pour tout $\tau,\tau'\in\types$, $\trad{\tau\times\tau'}=\trad\tau\otimes\trad{\tau'}$.
        \item Pour tout $\tau,\tau'\in\types$, $\trad{\tau + \tau'}=\trad\tau\oplus_\SN\trad{\tau'}$.
    \end{itemize}

    Alors $\trad -$ est adéquat.
\end{them}

\begin{exo}
    Montrer le théorème précédent.
\end{exo}

\begin{cor}[Forte normalisation]
    Soit $M\in\Lambda^{\to\times1+0}$ tel qu'il existe $\Gamma,\tau$ tels que $\Gamma\vdash M : \tau$, alors $M\in\SN$.
\end{cor}

\begin{proof}
    On construit l'interprétation adéquate respectant les prémisses du théorème précédent. Pour cela, pour $\iota\in\mathcal B$, on associe $\trad\iota = \{M\mid \exists \Gamma, \Gamma\vdash M : \iota\}$. On vérifie que cet ensemble est bien dans $\SATS$ :
    \begin{itemize}[label=$\bullet$]
        \item Pour un contexte $E[\;]\in\Elim\cap\SN$ typé dans $\Gamma$, on prend $E[x]$ avec $x\notin \varlib{E[\;]}$ et $(\Gamma,x : \iota)$, donc $E[x]\in\trad\iota$.
        \item Si $N\in\SN$ et que $\Gamma\vdash E[M[N/x]] : \iota$ alors comme $E[(\lambda x.M)\;N]\reduc E[M[N/x]]$, par préservation du typage, on déduit que $\Gamma\vdash E[(\lambda x.M)\;N] : \iota$.
        \item De même pour les autres cas, par préservation du typage.
    \end{itemize}

    On en déduit que si $\sigma\models_{\trad -} \Gamma$ et $\Gamma\vdash M : \tau$ alors $M\;\sigma\in\trad\tau$ donc en particulier pour $\sigma$ la substitution triviale, on a $M\in\trad\tau$ donc $M\in\SN$.
\end{proof}