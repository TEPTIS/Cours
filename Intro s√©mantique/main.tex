\documentclass[11pt,french]{article}

\title{Intro Sémantique}
\author{Cassis}
\date{March 2022}

\usepackage{prelude}

\begin{document}

\setlength{\unitlength}{1cm}

\date{}

\thispagestyle{empty}

\vspace{0.5cm}

\begin{center}

	\vspace{1.5cm}

	\rule[11pt]{9.5cm}{0.5pt}

	\textbf{\huge Introduction à la \\ sémantique dénotationnelle}

	\vspace{0.2cm}

	\text{}

	\vspace{0.2cm}

	\text{Cassis}

	\text{Juillet 2022}

	\rule{9.5cm}{0.5pt}

	\vspace{8cm}

\end{center}

\section*{Introduction}

Ce document cherche à introduire la sémantique dénotationnelle. Il s'adresse en priorité à des néophytes sur le sujet qui ont malgré tout de l'expérience en mathématiques. Nous utiliserons comme modèle le langage OCaml. Les parties seront réparties d'abord en une présentation de ce que nous allons utiliser dans OCaml (nous utiliserons seulement une version réduite de ce langage), puis nous parlerons de domaines et d'ordre partiel complet, en utilisant comme exemple notre modèle de OCaml.

\vspace{\baselineskip}

\vfill

\date{}

\newpage
\thispagestyle{empty}
\tableofcontents
\vspace{5cm}

\vfill \hfill
\newpage \setcounter{page}{1}

\part{Théorie des domaines}

La théorie des domaines est une théorie visant à étudier certains types d'ensembles permettant de modéliser le lambda-calcul. Nous étudierons ici les domaines de Scott à partir des ordres partiels complets, puis nous étudierons la topologie liée à ces domaines.

\section{Ordre et domaine}

Nous supposons ici connues les notions basiques de théorie des ordres (relation d'ordre, borne supérieure, majorant, etc).

\subsection{Définitions de base}

\begin{defi}[Ordre partiel, dirigé]
    Soit $(D,\leq)$ un ensemble ordonné. On dit qu'une partie $\Delta\subseteq D$ est filtrante si $$\forall x,y\in\Delta,\exists z\in\Delta, x\leq z\land y\leq z$$ Un exemple classique de partie filtrante est celui des chaînes, que l'on peut voir comme des suites croissantes d'éléments. On notera $\Delta\subsetdir D$ pour dire que $\Delta$ est une partie filtrante de $D$.
    
    On dit que $D$ est un ordre partiel dirigé complet si toute partie filtrante $\Delta\subseteq D$ admet une borne supérieure, qu'on notera $\bigvee\Delta$. Si de plus, $D$ possède un minorant, que l'on notera $\bot$, alors on dit que $D$ est un ordre partiel complet. On abrégera désormais \og ordre partiel dirigé complet\fg{} en dcpo (\textit{directed complete partial order}) et \og ordre partiel complet\fg{} en cpo (\textit{complete partial order}).
\end{defi}

Définissons maintenant la notion adaptée de morphisme dans ce contexte.

\begin{defi}[Fonction croissante, continue]
    Soient deux dcpo $D$ et $D'$ et $f : D\to D'$. On dit que $f$ est croissante lorsque $$\forall x,y\in D, x\leq y\implies f(x)\leq f(y)$$ et continue lorsque $f$ est croissante et que $$\forall \Delta\subsetdir D, f(\bigvee\Delta)=\bigvee f(\Delta)$$
\end{defi}

\begin{exo}
    Pour s'assurer que la définition de fonction continue est valide, vérifier que l'image d'une partie filtrante par une fonction croissante est une partie filtrante de l'ensemble d'arrivée.
\end{exo}

\begin{exo}
    Montrer que si $D$ est un dcpo, alors l'identité est continue sur $D$. Montrer que la composée de deux fonctions continues est continue.
\end{exo}

De l'exercice précédent, on déduit que les dcpo forment une catégorie, notée $\dcpo$ (cf. Définition \ref{categorie}). De plus, $\cpo$ est la sous-catégorie pleine dont les objets sont les cpo.

Donnons des exemples classiques et utiles de cpo. La vérification que ceux-ci sont des cpo est laissée en exercice.

\begin{expl}
    \ 
    \begin{itemize}[label=$\bullet$]
        \item Si $X$ est un ensemble, alors l'ensemble $X_\bot=X\cup\{\bot\}$ (où $\bot\notin X$) est un cpo en le munissant de l'ordre défini par $$x\leq y\iff x=\bot \lor x=y$$ On appelle cela le domaine plat de $X$. Par exemple, le domaine $\mathbb N_\bot$ peut être représenté par la figure~\ref{Nbot}, indiquant l'ordre dans l'ensemble par une flèche.
        \begin{figure}[t]
        \centering
        \rule{17cm}{0.5pt}\\
        \vspace{1cm}
        \begin{tikzcd}
            0 & 1 & 2 & 3 & \cdots\\
            & &\bot\ar[ull]\ar[ul]\ar[u]\ar[ur]\ar[urr]& & \\
        \end{tikzcd}\\
        \rule{17cm}{0.5pt}
        \caption{Domaine $\mathbb N_\bot$}
        \label{Nbot}
        \end{figure}
        
        \item Si $X$ et $Y$ sont deux ensembles, alors l'ensemble des fonction partielles de $X$ dans $Y$, noté $X\pto Y$, forme un cpo. L'ordre est donné par l'inclusion des graphes de fonction, c'est-à-dire que $f\leq g$ si $g$ est égale à $f$ là où $f$ est définie. Le minorant de cet ensemble est la fonction définie nulle part.
    \end{itemize}
\end{expl}

Nous pouvons déjà motiver notre étude des cpo par un premier résultat : une fonction continue d'un cpo dans lui-même possède un point fixe (ce qui fait des cpo de bons candidats pour modéliser le combinateur $Y$).

\begin{prop}
    Soit $(D,\leq)$ un cpo et $f : D\to D$ continue. Alors $\displaystyle{\bigvee_{n\in\mathbb N}f^n(\bot)}$ est le plus petit pré-point fixe de $f$ (c'est-à-dire point $x$ tel que $f(x)\leq x$).
\end{prop}
\begin{proof}
    Tout d'abord, cet élément existe car $(f^n(\bot))$ est une suite croissante d'éléments. En effet, comme $\bot\leq f(\bot)$, en appliquant $n$ fois $f$ qui est croissante, on en déduit que $f^n(\bot)\leq f^{n+1}(\bot)$. C'est un point fixe car $$f\left(\bigvee_{n\in\mathbb N}f^n(\bot)\right)=\bigvee_{n\in\mathbb N} f^{n+1}(\bot)=\bigvee_{n\in\mathbb N}f^n(\bot)$$
    
    De plus, si $f(x)\leq x$, alors on peut montrer par récurrence que $f^n(\bot)\leq x$. Le car $n=0$ est évident par minoration de $\bot$, et si $f^n(\bot)\leq x$ alors $f^{n+1}(\bot)\leq f(x)$ donc $f^{n+1}(\bot)\leq x$ par transitivité. On en déduit par inégalité sur les bornes supérieures le résultat.
\end{proof}

\begin{exo}
    Soit $D$ un treillis complet, c'est-à-dire que $D$ est un ensemble ordonné tel que toute partie de $D$ a une borne supérieure et une borne inférieure, et soit $f : D \to D$ croissante. Montrer qu'elle a un plus petit point fixe, qui est $\displaystyle{\bigwedge\compre{x}{f(x)\leq x}}$, et que l'ensemble des points fixes de $f$ forme un treillis complet.
\end{exo}

\begin{exo}
    Soit $D$ un cpo, $f : D \to D$ croissante. On définit par induction transfinie $f^0=\bot$, $f^{\lambda+1}=f(f^\lambda)$ et pour $\lambda$ ordinal limite, $f^\lambda=\displaystyle{\bigvee_{\alpha< \lambda} f^\alpha}$. Montrer qu'il existe un ordinal $\mu$ tel que $f^\mu$ est le plus petit pré-point fixe de $f$. \textit{Indication : un cpo étant un ensemble, il est plus petit que la classe des ordinaux.}
\end{exo}

\subsection{\'Elements finis}

Intéressons-nous désormais aux éléments comportant de l'information finie. En effet, nous allons vouloir écrire nos éléments comme des limites d'éléments finis, par exemple pour définir une fonction comme la limite d'un graphe se remplissant au fur et à mesure. Pour cela, nous avons besoin de la notion d'élément compact

\begin{defi}[\'Element compact]
    Soit $D$ un dcpo. On appelle élément compact (ou fini) de $D$ un élément $d\in D$ tel que $$\forall \Delta\subsetdir D, \left[d\leq \bigvee \Delta \implies \exists \delta\in\Delta, d\leq \delta\right]$$
    
    On note l'ensemble des éléments compacts de $D$ par $\compact D$.
\end{defi}

On souhaite alors déterminer des dcpo où les éléments sont des limites de suites de compacts, d'où la notion d'algébricité.

\begin{defi}[Algébricité]
    On dit d'un dcpo $D$ qu'il est algébrique si pour tout $x\in D$, l'ensemble $\Delta=\compre{d}{d\in\compact D, d\leq x}$ est une partie filtrante telle que $\bigvee\Delta = x$. On appelle les éléments de $\Delta$ les approximations de $x$, et on dit que $\compact D$ est la base de $D$.
    
    La sous-catégorie pleine de $\dcpo$ contenant les dcpo algébriques est notée $\adcpo$, et de même $\acpo$ est la sous-catégorie pleine de $\cpo$ contenant les cpo algébriques.
\end{defi}

\begin{expl}
    Nous avons vu que $X\pto Y$ est un cpo, c'est en fait un acpo où les éléments compacts sont les fonctions de domaine de définition fini.
\end{expl}

\begin{exo}
    Prouver que l'exemple précédent est bien un acpo.
\end{exo}

Définissons maintenant une caractérisation de la continuité, qui sera plus naturelle en considérant les éléments compacts comme finis (en effet, on souhaite donc particulièrement regarder des limites dans $\compact D$).

\begin{prop}[$\epsilon\delta$-continuité]
    Soient $D$ et $D'$ deux adcpo. Alors
    \begin{itemize}[label=$\bullet$]
        \item Une fonction $f : D \to D'$ est continue si et seulement si elle est croissante et pour tout $d'\in\compact{D'}$ et $x\in D$ tel que $d'\leq f(x)$, il existe $d\in\compact D$ tel que $d'\leq f(d)$.
        \item L'ensemble $\compre{(d,d')\in\compact{D}\times\compact{D'}}{d'\leq f(d)}$ détermine le graphe de la fonction $f$.
    \end{itemize}
\end{prop}

\begin{proof}
    Prouvons le premier point.
    
    Supposons $f$ continue. Soit $d'$ et $x$ pris comme dans l'énoncé, tels que $d'\leq f(x)$. Alors on trouve une partie filtrante $\Delta$ de $\compact D$ dont $x$ est la borne supérieure, et comme $f(\Delta)$ est une partie filtrante de borne supérieure plus grande que $d'$, par compacité de $d'$ on trouve $f(d)$ tel que $d'\leq f(d)$, donc il existe bien un tel $d\in\compact D$. Réciproquement, montrons que $f$ est continue. Soit $\Delta\subsetdir D$. Par croissance, on a $\bigvee f(\Delta)\leq f(\bigvee\Delta)$. Pour montrer l'inégalité inverse, il suffit de montrer que pour tout élément compact $d'$ tel que $d'\leq \bigvee f(\Delta)$ (par algébricité de $D'$), il existe $d\in\Delta$ tel que $d'\leq f(d)$. On construit d'abord $\delta\in\compact D$ tel que $d'\leq f(\delta)$ et $\delta \leq \bigvee f(\Delta)$, et on construit alors $d$ à partir de la compacité de $\delta$.
    
    On remarque que pour $x\in D$, on a 
    \begin{align*}
        f(x) &= \bigvee \compre{d'}{d'\in\compact{D'}, d'\leq f(x)}\\
        &= \bigvee \compre{d'}{d'\in\compact{D'}, \exists d\in\compact D, d\leq x\land d'\leq f(d)}
    \end{align*}
    
    D'où le résultat.
\end{proof}

\subsection{Construire des dcpo}

Nous avons déjà une intuition de ce que nous voulons pour interpréter notre langage OCaml : nous allons utiliser des cpo, et une fonction sera une fonction continue sur ces cpo. Pour pouvoir utiliser ce modèle, il nous faut cependant construire des ensembles adaptés aux ajouts que nous avons faits. Il faut donc pouvoir construire une notion qui permette d'avoir à la fois nos domaines et aussi des domaines pour le produit et l'espace des fonctions continues. Ces domaines sont les domaines de Scott, que nous verrons en fin de partie.

\begin{defi}[Produit de dcpo]
    Soient $D$ et $D'$ deux dcpo. On définit leur produit $D\times D'$ en donnant l'ordre $$(d,d')\leq (e,e')\iff d\leq e\land d'\leq e'$$ Si $D$ et $D'$ sont des cpo, alors $D\times D'$ est aussi un cpo.
\end{defi}

\begin{proof}
    Soit $\Delta\subsetdir D\times D'$. On définit $\Delta_D=\compre{d}{\exists d'\in D', (d,d')\in\Delta}$ et de même $\Delta_{D'}$. On peut alors définir $(\delta,\delta')=(\bigvee \Delta_D,\bigvee \Delta_{D'})$. On vérifie directement que $(\delta,\delta')=\bigvee \Delta$. Enfin, le minorant de deux cpo est $(\bot,\bot)$.
\end{proof}

Donnons un résultat primordial pour justifier la continuité des fonctions que nous utiliserons. En effet, si la continuité n'est en général pas simplement la continuité pour chaque argument, il se trouve que c'est le cas ici.

\begin{prop}
    Soient $D,D',E$ des dcpo, et $f : D\times D'\to E$. Alors $f$ est continue si et seulement si les fonctions $f_x : y \mapsto f(x,y)$ et les fonctions $f_y : x\mapsto f(x,y)$ sont continues pour tout $x$ et tout $y$.
\end{prop}

\begin{proof}
    Le sens direct se fait de façon en remarquant que si $\Delta\subsetdir D'$ alors pour tout $x\in D$, $(x,\Delta)=\compre{(x,\delta)}{\delta\in\Delta}$ est une partie filtrante de $D\times D'$. Alors $$\bigvee f_x(\Delta)= \bigvee f(x,\Delta)=f\bigvee(x,\Delta)=f_x(\bigvee \Delta)$$
    
    Réciproquement, soit $f$ continue en chaque argument. Alors soit $\Delta\subsetdir D\times D'$. On réutilise la définition précédente de $\Delta_D$ et $\Delta_{D'}$. Alors :
    \begin{center}
        \begin{align*}
            f(\bigvee \Delta) &= f(\bigvee \Delta_D,\bigvee \Delta_{D'})=\bigvee f(\Delta_D,\bigvee\Delta_{D'})\\
            &= \bigvee \compre{\delta,\bigvee\Delta}{\delta\in \Delta_D} = \bigvee f(\Delta_D,\Delta_{D'})
        \end{align*}
    \end{center}
    
    Il convient alors de prouver que $\bigvee f(\Delta_D,\Delta_{D'})=\bigvee f(\Delta)$. $\Delta\subseteq \Delta_D\times\Delta_{D'}$ d'où l'une des inégalités. Réciproquement, comme $\Delta$ est filtrant, pour chaque élément de $\Delta_D\times \Delta_{D'}$ il existe un majorant dans $\Delta$.
\end{proof}

Pour le produit, l'algébricité se passe correctement.

\begin{prop}
    Soient $D,D'$ deux adcpo, alors $D\times D'$ est un adcpo et $\compact{D\times D'}=\compact D\times \compact{D'}$.
\end{prop}

\begin{proof}
    Prouvons d'abord l'égalité sur les éléments compacts.
    
    Soit $(d,d')\in\compact{D\times D'}$. Soit $\Delta\subsetdir D$ tel que $d'\leq \bigvee \Delta$. Alors $(\Delta,d')\subsetdir D\times D'$ et $(d,d')\leq \bigvee (\Delta,d')$ donc par algébricité on trouve un élément $\delta$ de $\Delta$ tel que $d\leq \delta$, donc $d$ est algébrique. De même, on prouve que $d'$ est algébrique. Donc $\compact{D\times D'}\subseteq \compact{D}\times\compact{D'}$.
    
    Soit $(d,d')\in\compact D\times \compact{D'}$. Soit $\Delta\subsetdir D\times D'$ tel que $(d,d')\leq \bigvee \Delta$. On sait que $\bigvee\Delta=\bigvee(\Delta_D,\Delta_{D'})$, donc on trouve par algébricité de $d$ et de $d'$, respectivement $\delta\in\Delta_D$ et $\delta'\in\Delta_{D'}$ tels que $d\leq \delta$ et $d'\leq \delta'$. Or par définition de $\Delta_D$ et $\Delta_{D'}$, on trouve $\alpha$ et $\alpha'$ tels que $(\delta,\alpha)\in\Delta$ et $(\alpha',\delta')\in\Delta$. Puisque $\Delta$ est filtrante, on trouve $(\beta,\beta')$ supérieur à ces deux valeur, dans $\Delta$. Donc on a trouvé $(d,d')\leq (\beta,\beta')\in\Delta$ : $(d,d')$ est algébrique dans $D\times D'$.
    
    Montrons alors que $D\times D'$ est algébrique. Soit $(x,y)\in D\times D'$. La partie de $\compact{D\times D'}$ donnée par $\compre{(d,d')}{(d,d')\in\compact{D\times D'},(d,d')\leq (x,y)}$ est filtrante car filtrante sur chaque coordonnée. De plus, En étudiant coordonnée par coordonnée, on en déduit bien que $(x,y)$ est la borne supérieure de cet ensemble.
\end{proof}

Intéressons-nous maintenant à l'espace fonctionnel d'un dcpo.

\begin{defi}[Espace des fonctions d'un dcpo]
    L'ensemble $D\to D'$, constitué des fonctions continues de $D$ dans $D'$, pour $D$ et $D'$ deux dcpo, est aussi un dcpo, avec l'ordre suivant :
    $$f\leq f'\iff \forall x\in D, f(x)\leq f'(x)$$
    
    De plus, si $D'$ est un cpo, alors $D\to D'$ est un cpo.
\end{defi}

\begin{proof}
    Soit $\Delta\subsetdir D\to D'$. On pose $f : x \mapsto \bigvee \Delta(x)$. Montrons que cette fonction est bien continue. Soit $\Delta'\subsetdir D$. Alors 
    \begin{center}
        \begin{align*}
            f(\bigvee\Delta') &= \bigvee\Delta(\bigvee \Delta') = \bigvee\compre{\bigvee g(\Delta')}{g\in\Delta}=\bigvee\Delta(\Delta')\\
            &= \bigvee\compre{\bigvee \Delta(\delta)}{\delta\in\Delta'}=\bigvee f(\Delta')
        \end{align*}
    \end{center}
    
    De plus, $x\mapsto \bot$ est un minorant de $D\to D'$ s'il existe.
\end{proof}

Les exercices suivants serviront pour l'interprétation catégorique du lambda-calcul.

\begin{exo}
    Soit $D$ un dcpo. Montrer que la fonction $\fonction{Y}{(D\to D)}{D}{f}{\displaystyle{\bigvee_{n\in\nat}f^n(\bot)}}$ est continue.
\end{exo}

\begin{exo}
    Soient $D$ et $D'$ deux dcpo. Montrer que les fonctions $\pi_1 : D\times D' \to D$ et $\pi_2 : D\times D' \to D'$, les deux projections, sont continues. Montrer aussi que la fonction $\langle -,-\rangle : D\to D'\to D\times D'$, $x\mapsto (y\mapsto \langle x,y\rangle)$ est continue.
\end{exo}

\begin{exo}
    Soient $D$, $D'$ et $E$ trois dcpo. Montrer que la fonction $\ev : (D\to D')\times D \to D'$ donnée par $\ev (f,x)=f(x)$ est continue. Soit la fonction $\Lambda : ((D\times D')\to E)\to (D\to D'\to E)$ qui à $f$ associe $\Lambda(f) : x \mapsto (y\mapsto f(x,y))$, montrer que $\Lambda$ est continue.
\end{exo}

Un souci arrive alors : si $D$ et $D'$ sont algébriques, $D\to D'$ ne l'est pas forcément. Pour autant, on peut dire plusieurs choses.

\begin{prop}[Fonction en escalier]
    Soient $D$ et $D'$ deux cpo, et $(d,d')\in\compact{D}\times\compact{D'}$ alors :
    \begin{itemize}[label=$\bullet$]
        \item La fonction $d\to d'$ est compact, avec la définition suivante :
        $$(d\to d')(x)=\left\{\begin{array}{cl}
            d' & \mathrm{si}\; d\leq x \\
            \bot & \mathrm{sinon}
        \end{array}\right.$$
        \item Si $D$ et $D'$ sont algébrique, alors $$f=\bigvee\compre{d\to d'}{(d\to d')\leq f}$$
    \end{itemize}
\end{prop}

\begin{proof}
    Remarquons d'abord que la compacité de $d$ permet de déduire que $d\to d'$ est continue. De plus, pour tout $f : D \to D'$, $d\to d' \leq f$ si et seulement si $d'\leq f(d)$.
    
    \begin{itemize}[label=$\bullet$]
        \item Si $d\to d'\leq \bigvee\Delta$, alors $d'=(d\to d')(d)\leq\bigvee\compre{f(d)}{d\in\Delta}$ d'où la conclusion par compacté de $d'$.
        \item On remarque que $\compre{d\to d'}{(d\to d')\leq f}\leq g$ si et seulement si pour tout $d,d'$, $(d'\leq f(d)\implies d'\leq g(d))$ et si et seulement si $f\leq g$.
    \end{itemize}
\end{proof}

Le souci maintenant est que cet ensemble n'est pas filtrant. En effet, nous voulons donc pouvoir définir la borne supérieure d'un nombre fini de ces fonctions en escalier. Pour cela, nous ajoutons une condition.

\begin{defi}[Domaine de Scott]
    On dit qu'un dcpo est borné complet si pour toute paire $(x,y)$ telle qu'il existe un majorant de $\{x,y\}$, il existe une borne supérieure à $\{x,y\}$.
    
    On appelle un domaine de Scott un cpo algébrique borné complet et on note $\scott$ la sous-catégorie pleine de $\acpo$ contenant comme objet les domaines de Scott.
\end{defi}

Nous admettrons alors un dernier résultat.

\begin{them}
    Si $D$ et $D'$ sont des domaines de Scott, alors $D\to D'$ est aussi un domaine de Scott, et les éléments algébriques de ce domaines sont exactement de la forme $(d_1\to d'_1)\wedge (d_2\to d'_2)\wedge\ldots\wedge (d_n\to d'_n)$ pour un $n$ fini.
\end{them}


\newpage

\section[Interprétation catégorique]{Interprétation dans une catégorie cartésienne fermée du lambda-calcul}

Cette section réutilisera les résultats de la section précédente pour développer un cadre général d'interprétation catégorique. Nous allons donc faire correspondre à nos lambda-termes des fonctions dans une catégorie. Le choix d'une catégorie cartésienne fermée est naturel puisque c'est la notion la plus simple de catégorie possédant des exponentiations, c'est-à-dire des objets de la forme $a\to b$. Nous verrons ainsi, tout d'abord, la définition d'une catégorie cartésienne fermée, pour pouvoir ensuite donner l'interprétation d'un lambda-terme du lambda-calcul simplement typé (sans extension car celles-ci seront traitées dans la partie sur l'interprétation de OCaml).

\subsection{Définitions}

Rappelons la définition d'une catégorie.

\begin{defi}[Catégorie]\label{categorie}
    Une catégorie $\cat$ est une classe composée :
    \begin{itemize}[label=$\bullet$]
        \item d'objets, dont on notera la classe $\cat_0$.
        \item de flèches, dont on notera la classe $\cat_1$. Chaque flèche $f$ possède un domaine, noté $\mathrm{dom}(f)$ et un codomaine, noté $\mathrm{codom}(f)$. On notera plus simplement $f : a\to b$ pour dire que $\mathrm{dom}(f)=a$ et $\mathrm{codom}(f)=b$.
        \item d'une opération de composition, associative, notée $\circ$, qui associe à deux flèches $f : a\to b$ et $g : b\to c$ une flèche $g\circ f : a \to c$.
        \item pour chaque objet $c$, d'une flèche $\id_c$ appelée identité de $c$ telle que pour toute flèche $f : a \to c, \id_c\circ f = f$ et pour toute flèche $g : c \to b, g\circ \id_c = g$.
    \end{itemize}
\end{defi}

Nous donnerons des exemples de catégories classiques et utiles dans notre étude.

\begin{expl}
    \ 
    \begin{itemize}[label=$\bullet$]
        \item La catégorie \textbf{Set} des ensembles avec comme flèches les applications entre les ensembles.
        \item Les catégories $\dcpo$ et $\cpo$ avec comme flèches les application continues.
        \item Les catégories $\adcpo$ et $\acpo$ avec les mêmes flèches.
        \item La catégorie $\scott$ des domaines de Scott avec toujours les applications continues.
    \end{itemize}
\end{expl}

\begin{rmk}
    Le fait que les catégories précédentes en sont bien a été vérifié au long des exercices de la partie précédente, lorsqu'on justifiait la continuité des différentes fonctions.
\end{rmk}

Nous avons, de plus, besoin de trois éléments pour définir une catégorie cartésienne fermée : un objet terminal, un produit et une exponentiation. Nous allons donc définir ces termes dans un premier temps.

\begin{defi}[Objet terminal]
    On appelle objet terminal d'une catégorie $\cat$ un objet, noté $1$, tel que pour tout objet $a\in\cat_0$, il existe une unique flèche $!_a : a \to 1$.
\end{defi}

\begin{defi}[Produit]
    Soit une catégorie $\cat$, deux objets $a$ et $b$. On appelle produit de $a$ et $b$, et on note $a\times b$, l'unique objet à isomorphisme près tel que pour tout objet $x$ et toute paire de flèche $f : x \to a, g : x\to b$ il existe une unique fonction, notée $\langle f,g\rangle$ qui fasse commuter le diagramme de la figure \ref{produit}.
    
    \begin{figure}[t]
        \centering
        \rule{17cm}{0.5pt}
        \begin{tikzcd}
        \\
            & x \ar[dl,"f"]\ar[dr,"g"]\ar[d,dashed,"\langle f\comma g\rangle"] \\
            a & a\times b \ar[l,"\pi_1"]\ar[r,"\pi_2"]& b\\
        \end{tikzcd}
        \rule{17cm}{0.5pt}
        \vspace{-0.5cm}
        \caption{Diagramme commutatif du produit}
        \label{produit}
    \end{figure}
\end{defi}

\begin{defi}[Exponentiation]
    Soient $a$ et $b$ deux objets de $\cat$. On appelle exponentielle de $a$ par $b$ l'objet $a^b$, représentant les fonctions $b\to a$. On se munit d'une fonction $\ev : a^b\times b \to a$ et on a la propriété universelle que pour toute fonction $f : c\times b \to a$, alors il existe une unique fonction $\Lambda(f)$ (appelée curryfication de $f$) telle que le diagramme de la figure \ref{expo} commute.
    \begin{figure}[t]
        \centering
        \rule{17cm}{0.5pt}
        \begin{tikzcd}
            \\
            a^b\times b \ar[r,"\ev"] & a\\
            c\times b \ar[u,"\Lambda(f)\times \id_b"]\ar[ur,"f"]\\
        \end{tikzcd}
        \rule{17cm}{0.5pt}
        \caption{Diagramme commutatif de l'exponentielle}
        \label{expo}
    \end{figure}
\end{defi}

Une catégorie cartésienne fermée est donc une catégorie dans laquelle on a un élément terminale, tous les produits binaires (et donc tous les produits finis, par récurrence évidente) et toutes les exponentiations. Notre cadre d'étude, que sont les domaines de Scott, est une catégorie cartésienne fermée $\scott$, grâce aux exercices précédents.

\subsection{Interprétation catégorique}

Nous pouvons désormais définir l'interprétation catégorique du lambda-calcul simplement typé dans une CCC. Pour cela, on se fixe tout d'abord les objets de notre catégorie. Ceux-ci seront des interprétations de nos types. On notera $\llbracket -\rrbracket$ la fonction d'interprétation, associant à un objet syntaxique (type, terme...) une interprétation sémantique dans notre catégorique. Nous supposons donc que pour chaque type $\tau$ il existe un objet $\llbracket\tau\rrbracket$ associé dans lequel sera interprété notre type. Les morphismes seront les interprétations des jugements de typage.

Nous allons d'abord définir l'interprétation d'un contexte. Soit un contexte $\Gamma$ donné sous la forme $\Gamma = x_1 : \tau_1,\ldots , x_n : \tau_n$, on définit $\llbracket\Gamma\rrbracket=\llbracket\tau_1\rrbracket\times\ldots\times\llbracket\tau_n\rrbracket$. Si $\Gamma = \varnothing$, alors $\llbracket\Gamma\rrbracket=1$. En effet, une flèche $1\to a$ est exactement un élément de $a$, donc un jugement de la forme $\vdash M : \tau$ sera exactement un élément de $\llbracket\tau\rrbracket$. Nous pouvons maintenant, par induction sur la structure d'un lambda-terme simplement typé, définir l'interprétation d'un lambda-terme.

\begin{defi}[Interprétation]
    On définit par induction l'interprétation d'un lambda-terme :
    \begin{itemize}[label=$\bullet$]
        \item Si $\Gamma\vdash x : \tau$, on note $i$ l'indice d'occurrence de $x$ dans $\Gamma$, alors $$\llbracket\Gamma\vdash x : \tau\rrbracket = \pi_i$$ l'interprétation d'une variable est donc une projection du contexte.
        \item Si $\Gamma\vdash \lambda x.M : \sigma\to\tau$, on note $\llbracket\Gamma,x : \sigma\rrbracket=\llbracket\Gamma\rrbracket\times \llbracket\sigma\rrbracket$, ce qui nous permet e curryfier notre fonction :
        $$\llbracket\Gamma\vdash\lambda x.M : \sigma\to\tau\rrbracket=\Lambda\llbracket\Gamma,x : \sigma\vdash M : \tau\rrbracket$$
        \item Si $\Gamma\vdash M\;N : \tau$, nous avons juste à appliquer l'évaluation à $M$ et $N$ :
        $$\llbracket\Gamma\vdash M\;N : \tau\rrbracket = \ev\circ\langle\llbracket\Gamma\vdash M : \sigma\to\tau\rrbracket,\llbracket\Gamma\vdash N : \sigma\rrbracket\rangle$$
    \end{itemize}
\end{defi}

Ainsi, une catégorie cartésienne close, et en particulier $\scott$, permet d'interpréter les éléments de base du lambda-calcul simplement typé.

\newpage

\part{Théorie des domaines}

La théorie des domaines est une théorie visant à étudier certains types d'ensembles permettant de modéliser le lambda-calcul. Nous étudierons ici les domaines de Scott à partir des ordres partiels complets, puis nous étudierons la topologie liée à ces domaines.

\section{Ordre et domaine}

Nous supposons ici connues les notions basiques de théorie des ordres (relation d'ordre, borne supérieure, majorant, etc).

\subsection{Définitions de base}

\begin{defi}[Ordre partiel, dirigé]
    Soit $(D,\leq)$ un ensemble ordonné. On dit qu'une partie $\Delta\subseteq D$ est filtrante si $$\forall x,y\in\Delta,\exists z\in\Delta, x\leq z\land y\leq z$$ Un exemple classique de partie filtrante est celui des chaînes, que l'on peut voir comme des suites croissantes d'éléments. On notera $\Delta\subsetdir D$ pour dire que $\Delta$ est une partie filtrante de $D$.
    
    On dit que $D$ est un ordre partiel dirigé complet si toute partie filtrante $\Delta\subseteq D$ admet une borne supérieure, qu'on notera $\bigvee\Delta$. Si de plus, $D$ possède un minorant, que l'on notera $\bot$, alors on dit que $D$ est un ordre partiel complet. On abrégera désormais \og ordre partiel dirigé complet\fg{} en dcpo (\textit{directed complete partial order}) et \og ordre partiel complet\fg{} en cpo (\textit{complete partial order}).
\end{defi}

Définissons maintenant la notion adaptée de morphisme dans ce contexte.

\begin{defi}[Fonction croissante, continue]
    Soient deux dcpo $D$ et $D'$ et $f : D\to D'$. On dit que $f$ est croissante lorsque $$\forall x,y\in D, x\leq y\implies f(x)\leq f(y)$$ et continue lorsque $f$ est croissante et que $$\forall \Delta\subsetdir D, f(\bigvee\Delta)=\bigvee f(\Delta)$$
\end{defi}

\begin{exo}
    Pour s'assurer que la définition de fonction continue est valide, vérifier que l'image d'une partie filtrante par une fonction croissante est une partie filtrante de l'ensemble d'arrivée.
\end{exo}

\begin{exo}
    Montrer que si $D$ est un dcpo, alors l'identité est continue sur $D$. Montrer que la composée de deux fonctions continues est continue.
\end{exo}

De l'exercice précédent, on déduit que les dcpo forment une catégorie, notée $\dcpo$ (cf. Définition \ref{categorie}). De plus, $\cpo$ est la sous-catégorie pleine dont les objets sont les cpo.

Donnons des exemples classiques et utiles de cpo. La vérification que ceux-ci sont des cpo est laissée en exercice.

\begin{expl}
    \ 
    \begin{itemize}[label=$\bullet$]
        \item Si $X$ est un ensemble, alors l'ensemble $X_\bot=X\cup\{\bot\}$ (où $\bot\notin X$) est un cpo en le munissant de l'ordre défini par $$x\leq y\iff x=\bot \lor x=y$$ On appelle cela le domaine plat de $X$. Par exemple, le domaine $\mathbb N_\bot$ peut être représenté par la figure~\ref{Nbot}, indiquant l'ordre dans l'ensemble par une flèche.
        \begin{figure}[t]
        \centering
        \rule{17cm}{0.5pt}\\
        \vspace{1cm}
        \begin{tikzcd}
            0 & 1 & 2 & 3 & \cdots\\
            & &\bot\ar[ull]\ar[ul]\ar[u]\ar[ur]\ar[urr]& & \\
        \end{tikzcd}\\
        \rule{17cm}{0.5pt}
        \caption{Domaine $\mathbb N_\bot$}
        \label{Nbot}
        \end{figure}
        
        \item Si $X$ et $Y$ sont deux ensembles, alors l'ensemble des fonction partielles de $X$ dans $Y$, noté $X\pto Y$, forme un cpo. L'ordre est donné par l'inclusion des graphes de fonction, c'est-à-dire que $f\leq g$ si $g$ est égale à $f$ là où $f$ est définie. Le minorant de cet ensemble est la fonction définie nulle part.
    \end{itemize}
\end{expl}

Nous pouvons déjà motiver notre étude des cpo par un premier résultat : une fonction continue d'un cpo dans lui-même possède un point fixe (ce qui fait des cpo de bons candidats pour modéliser le combinateur $Y$).

\begin{prop}
    Soit $(D,\leq)$ un cpo et $f : D\to D$ continue. Alors $\displaystyle{\bigvee_{n\in\mathbb N}f^n(\bot)}$ est le plus petit pré-point fixe de $f$ (c'est-à-dire point $x$ tel que $f(x)\leq x$).
\end{prop}
\begin{proof}
    Tout d'abord, cet élément existe car $(f^n(\bot))$ est une suite croissante d'éléments. En effet, comme $\bot\leq f(\bot)$, en appliquant $n$ fois $f$ qui est croissante, on en déduit que $f^n(\bot)\leq f^{n+1}(\bot)$. C'est un point fixe car $$f\left(\bigvee_{n\in\mathbb N}f^n(\bot)\right)=\bigvee_{n\in\mathbb N} f^{n+1}(\bot)=\bigvee_{n\in\mathbb N}f^n(\bot)$$
    
    De plus, si $f(x)\leq x$, alors on peut montrer par récurrence que $f^n(\bot)\leq x$. Le car $n=0$ est évident par minoration de $\bot$, et si $f^n(\bot)\leq x$ alors $f^{n+1}(\bot)\leq f(x)$ donc $f^{n+1}(\bot)\leq x$ par transitivité. On en déduit par inégalité sur les bornes supérieures le résultat.
\end{proof}

\begin{exo}
    Soit $D$ un treillis complet, c'est-à-dire que $D$ est un ensemble ordonné tel que toute partie de $D$ a une borne supérieure et une borne inférieure, et soit $f : D \to D$ croissante. Montrer qu'elle a un plus petit point fixe, qui est $\displaystyle{\bigwedge\compre{x}{f(x)\leq x}}$, et que l'ensemble des points fixes de $f$ forme un treillis complet.
\end{exo}

\begin{exo}
    Soit $D$ un cpo, $f : D \to D$ croissante. On définit par induction transfinie $f^0=\bot$, $f^{\lambda+1}=f(f^\lambda)$ et pour $\lambda$ ordinal limite, $f^\lambda=\displaystyle{\bigvee_{\alpha< \lambda} f^\alpha}$. Montrer qu'il existe un ordinal $\mu$ tel que $f^\mu$ est le plus petit pré-point fixe de $f$. \textit{Indication : un cpo étant un ensemble, il est plus petit que la classe des ordinaux.}
\end{exo}

\subsection{\'Elements finis}

Intéressons-nous désormais aux éléments comportant de l'information finie. En effet, nous allons vouloir écrire nos éléments comme des limites d'éléments finis, par exemple pour définir une fonction comme la limite d'un graphe se remplissant au fur et à mesure. Pour cela, nous avons besoin de la notion d'élément compact

\begin{defi}[\'Element compact]
    Soit $D$ un dcpo. On appelle élément compact (ou fini) de $D$ un élément $d\in D$ tel que $$\forall \Delta\subsetdir D, \left[d\leq \bigvee \Delta \implies \exists \delta\in\Delta, d\leq \delta\right]$$
    
    On note l'ensemble des éléments compacts de $D$ par $\compact D$.
\end{defi}

On souhaite alors déterminer des dcpo où les éléments sont des limites de suites de compacts, d'où la notion d'algébricité.

\begin{defi}[Algébricité]
    On dit d'un dcpo $D$ qu'il est algébrique si pour tout $x\in D$, l'ensemble $\Delta=\compre{d}{d\in\compact D, d\leq x}$ est une partie filtrante telle que $\bigvee\Delta = x$. On appelle les éléments de $\Delta$ les approximations de $x$, et on dit que $\compact D$ est la base de $D$.
    
    La sous-catégorie pleine de $\dcpo$ contenant les dcpo algébriques est notée $\adcpo$, et de même $\acpo$ est la sous-catégorie pleine de $\cpo$ contenant les cpo algébriques.
\end{defi}

\begin{expl}
    Nous avons vu que $X\pto Y$ est un cpo, c'est en fait un acpo où les éléments compacts sont les fonctions de domaine de définition fini.
\end{expl}

\begin{exo}
    Prouver que l'exemple précédent est bien un acpo.
\end{exo}

Définissons maintenant une caractérisation de la continuité, qui sera plus naturelle en considérant les éléments compacts comme finis (en effet, on souhaite donc particulièrement regarder des limites dans $\compact D$).

\begin{prop}[$\epsilon\delta$-continuité]
    Soient $D$ et $D'$ deux adcpo. Alors
    \begin{itemize}[label=$\bullet$]
        \item Une fonction $f : D \to D'$ est continue si et seulement si elle est croissante et pour tout $d'\in\compact{D'}$ et $x\in D$ tel que $d'\leq f(x)$, il existe $d\in\compact D$ tel que $d'\leq f(d)$.
        \item L'ensemble $\compre{(d,d')\in\compact{D}\times\compact{D'}}{d'\leq f(d)}$ détermine le graphe de la fonction $f$.
    \end{itemize}
\end{prop}

\begin{proof}
    Prouvons le premier point.
    
    Supposons $f$ continue. Soit $d'$ et $x$ pris comme dans l'énoncé, tels que $d'\leq f(x)$. Alors on trouve une partie filtrante $\Delta$ de $\compact D$ dont $x$ est la borne supérieure, et comme $f(\Delta)$ est une partie filtrante de borne supérieure plus grande que $d'$, par compacité de $d'$ on trouve $f(d)$ tel que $d'\leq f(d)$, donc il existe bien un tel $d\in\compact D$. Réciproquement, montrons que $f$ est continue. Soit $\Delta\subsetdir D$. Par croissance, on a $\bigvee f(\Delta)\leq f(\bigvee\Delta)$. Pour montrer l'inégalité inverse, il suffit de montrer que pour tout élément compact $d'$ tel que $d'\leq \bigvee f(\Delta)$ (par algébricité de $D'$), il existe $d\in\Delta$ tel que $d'\leq f(d)$. On construit d'abord $\delta\in\compact D$ tel que $d'\leq f(\delta)$ et $\delta \leq \bigvee f(\Delta)$, et on construit alors $d$ à partir de la compacité de $\delta$.
    
    On remarque que pour $x\in D$, on a 
    \begin{align*}
        f(x) &= \bigvee \compre{d'}{d'\in\compact{D'}, d'\leq f(x)}\\
        &= \bigvee \compre{d'}{d'\in\compact{D'}, \exists d\in\compact D, d\leq x\land d'\leq f(d)}
    \end{align*}
    
    D'où le résultat.
\end{proof}

\subsection{Construire des dcpo}

Nous avons déjà une intuition de ce que nous voulons pour interpréter notre langage OCaml : nous allons utiliser des cpo, et une fonction sera une fonction continue sur ces cpo. Pour pouvoir utiliser ce modèle, il nous faut cependant construire des ensembles adaptés aux ajouts que nous avons faits. Il faut donc pouvoir construire une notion qui permette d'avoir à la fois nos domaines et aussi des domaines pour le produit et l'espace des fonctions continues. Ces domaines sont les domaines de Scott, que nous verrons en fin de partie.

\begin{defi}[Produit de dcpo]
    Soient $D$ et $D'$ deux dcpo. On définit leur produit $D\times D'$ en donnant l'ordre $$(d,d')\leq (e,e')\iff d\leq e\land d'\leq e'$$ Si $D$ et $D'$ sont des cpo, alors $D\times D'$ est aussi un cpo.
\end{defi}

\begin{proof}
    Soit $\Delta\subsetdir D\times D'$. On définit $\Delta_D=\compre{d}{\exists d'\in D', (d,d')\in\Delta}$ et de même $\Delta_{D'}$. On peut alors définir $(\delta,\delta')=(\bigvee \Delta_D,\bigvee \Delta_{D'})$. On vérifie directement que $(\delta,\delta')=\bigvee \Delta$. Enfin, le minorant de deux cpo est $(\bot,\bot)$.
\end{proof}

Donnons un résultat primordial pour justifier la continuité des fonctions que nous utiliserons. En effet, si la continuité n'est en général pas simplement la continuité pour chaque argument, il se trouve que c'est le cas ici.

\begin{prop}
    Soient $D,D',E$ des dcpo, et $f : D\times D'\to E$. Alors $f$ est continue si et seulement si les fonctions $f_x : y \mapsto f(x,y)$ et les fonctions $f_y : x\mapsto f(x,y)$ sont continues pour tout $x$ et tout $y$.
\end{prop}

\begin{proof}
    Le sens direct se fait de façon en remarquant que si $\Delta\subsetdir D'$ alors pour tout $x\in D$, $(x,\Delta)=\compre{(x,\delta)}{\delta\in\Delta}$ est une partie filtrante de $D\times D'$. Alors $$\bigvee f_x(\Delta)= \bigvee f(x,\Delta)=f\bigvee(x,\Delta)=f_x(\bigvee \Delta)$$
    
    Réciproquement, soit $f$ continue en chaque argument. Alors soit $\Delta\subsetdir D\times D'$. On réutilise la définition précédente de $\Delta_D$ et $\Delta_{D'}$. Alors :
    \begin{center}
        \begin{align*}
            f(\bigvee \Delta) &= f(\bigvee \Delta_D,\bigvee \Delta_{D'})=\bigvee f(\Delta_D,\bigvee\Delta_{D'})\\
            &= \bigvee \compre{\delta,\bigvee\Delta}{\delta\in \Delta_D} = \bigvee f(\Delta_D,\Delta_{D'})
        \end{align*}
    \end{center}
    
    Il convient alors de prouver que $\bigvee f(\Delta_D,\Delta_{D'})=\bigvee f(\Delta)$. $\Delta\subseteq \Delta_D\times\Delta_{D'}$ d'où l'une des inégalités. Réciproquement, comme $\Delta$ est filtrant, pour chaque élément de $\Delta_D\times \Delta_{D'}$ il existe un majorant dans $\Delta$.
\end{proof}

Pour le produit, l'algébricité se passe correctement.

\begin{prop}
    Soient $D,D'$ deux adcpo, alors $D\times D'$ est un adcpo et $\compact{D\times D'}=\compact D\times \compact{D'}$.
\end{prop}

\begin{proof}
    Prouvons d'abord l'égalité sur les éléments compacts.
    
    Soit $(d,d')\in\compact{D\times D'}$. Soit $\Delta\subsetdir D$ tel que $d'\leq \bigvee \Delta$. Alors $(\Delta,d')\subsetdir D\times D'$ et $(d,d')\leq \bigvee (\Delta,d')$ donc par algébricité on trouve un élément $\delta$ de $\Delta$ tel que $d\leq \delta$, donc $d$ est algébrique. De même, on prouve que $d'$ est algébrique. Donc $\compact{D\times D'}\subseteq \compact{D}\times\compact{D'}$.
    
    Soit $(d,d')\in\compact D\times \compact{D'}$. Soit $\Delta\subsetdir D\times D'$ tel que $(d,d')\leq \bigvee \Delta$. On sait que $\bigvee\Delta=\bigvee(\Delta_D,\Delta_{D'})$, donc on trouve par algébricité de $d$ et de $d'$, respectivement $\delta\in\Delta_D$ et $\delta'\in\Delta_{D'}$ tels que $d\leq \delta$ et $d'\leq \delta'$. Or par définition de $\Delta_D$ et $\Delta_{D'}$, on trouve $\alpha$ et $\alpha'$ tels que $(\delta,\alpha)\in\Delta$ et $(\alpha',\delta')\in\Delta$. Puisque $\Delta$ est filtrante, on trouve $(\beta,\beta')$ supérieur à ces deux valeur, dans $\Delta$. Donc on a trouvé $(d,d')\leq (\beta,\beta')\in\Delta$ : $(d,d')$ est algébrique dans $D\times D'$.
    
    Montrons alors que $D\times D'$ est algébrique. Soit $(x,y)\in D\times D'$. La partie de $\compact{D\times D'}$ donnée par $\compre{(d,d')}{(d,d')\in\compact{D\times D'},(d,d')\leq (x,y)}$ est filtrante car filtrante sur chaque coordonnée. De plus, En étudiant coordonnée par coordonnée, on en déduit bien que $(x,y)$ est la borne supérieure de cet ensemble.
\end{proof}

Intéressons-nous maintenant à l'espace fonctionnel d'un dcpo.

\begin{defi}[Espace des fonctions d'un dcpo]
    L'ensemble $D\to D'$, constitué des fonctions continues de $D$ dans $D'$, pour $D$ et $D'$ deux dcpo, est aussi un dcpo, avec l'ordre suivant :
    $$f\leq f'\iff \forall x\in D, f(x)\leq f'(x)$$
    
    De plus, si $D'$ est un cpo, alors $D\to D'$ est un cpo.
\end{defi}

\begin{proof}
    Soit $\Delta\subsetdir D\to D'$. On pose $f : x \mapsto \bigvee \Delta(x)$. Montrons que cette fonction est bien continue. Soit $\Delta'\subsetdir D$. Alors 
    \begin{center}
        \begin{align*}
            f(\bigvee\Delta') &= \bigvee\Delta(\bigvee \Delta') = \bigvee\compre{\bigvee g(\Delta')}{g\in\Delta}=\bigvee\Delta(\Delta')\\
            &= \bigvee\compre{\bigvee \Delta(\delta)}{\delta\in\Delta'}=\bigvee f(\Delta')
        \end{align*}
    \end{center}
    
    De plus, $x\mapsto \bot$ est un minorant de $D\to D'$ s'il existe.
\end{proof}

Les exercices suivants serviront pour l'interprétation catégorique du lambda-calcul.

\begin{exo}
    Soit $D$ un dcpo. Montrer que la fonction $\fonction{Y}{(D\to D)}{D}{f}{\displaystyle{\bigvee_{n\in\nat}f^n(\bot)}}$ est continue.
\end{exo}

\begin{exo}
    Soient $D$ et $D'$ deux dcpo. Montrer que les fonctions $\pi_1 : D\times D' \to D$ et $\pi_2 : D\times D' \to D'$, les deux projections, sont continues. Montrer aussi que la fonction $\langle -,-\rangle : D\to D'\to D\times D'$, $x\mapsto (y\mapsto \langle x,y\rangle)$ est continue.
\end{exo}

\begin{exo}
    Soient $D$, $D'$ et $E$ trois dcpo. Montrer que la fonction $\ev : (D\to D')\times D \to D'$ donnée par $\ev (f,x)=f(x)$ est continue. Soit la fonction $\Lambda : ((D\times D')\to E)\to (D\to D'\to E)$ qui à $f$ associe $\Lambda(f) : x \mapsto (y\mapsto f(x,y))$, montrer que $\Lambda$ est continue.
\end{exo}

Un souci arrive alors : si $D$ et $D'$ sont algébriques, $D\to D'$ ne l'est pas forcément. Pour autant, on peut dire plusieurs choses.

\begin{prop}[Fonction en escalier]
    Soient $D$ et $D'$ deux cpo, et $(d,d')\in\compact{D}\times\compact{D'}$ alors :
    \begin{itemize}[label=$\bullet$]
        \item La fonction $d\to d'$ est compact, avec la définition suivante :
        $$(d\to d')(x)=\left\{\begin{array}{cl}
            d' & \mathrm{si}\; d\leq x \\
            \bot & \mathrm{sinon}
        \end{array}\right.$$
        \item Si $D$ et $D'$ sont algébrique, alors $$f=\bigvee\compre{d\to d'}{(d\to d')\leq f}$$
    \end{itemize}
\end{prop}

\begin{proof}
    Remarquons d'abord que la compacité de $d$ permet de déduire que $d\to d'$ est continue. De plus, pour tout $f : D \to D'$, $d\to d' \leq f$ si et seulement si $d'\leq f(d)$.
    
    \begin{itemize}[label=$\bullet$]
        \item Si $d\to d'\leq \bigvee\Delta$, alors $d'=(d\to d')(d)\leq\bigvee\compre{f(d)}{d\in\Delta}$ d'où la conclusion par compacté de $d'$.
        \item On remarque que $\compre{d\to d'}{(d\to d')\leq f}\leq g$ si et seulement si pour tout $d,d'$, $(d'\leq f(d)\implies d'\leq g(d))$ et si et seulement si $f\leq g$.
    \end{itemize}
\end{proof}

Le souci maintenant est que cet ensemble n'est pas filtrant. En effet, nous voulons donc pouvoir définir la borne supérieure d'un nombre fini de ces fonctions en escalier. Pour cela, nous ajoutons une condition.

\begin{defi}[Domaine de Scott]
    On dit qu'un dcpo est borné complet si pour toute paire $(x,y)$ telle qu'il existe un majorant de $\{x,y\}$, il existe une borne supérieure à $\{x,y\}$.
    
    On appelle un domaine de Scott un cpo algébrique borné complet et on note $\scott$ la sous-catégorie pleine de $\acpo$ contenant comme objet les domaines de Scott.
\end{defi}

Nous admettrons alors un dernier résultat.

\begin{them}
    Si $D$ et $D'$ sont des domaines de Scott, alors $D\to D'$ est aussi un domaine de Scott, et les éléments algébriques de ce domaines sont exactement de la forme $(d_1\to d'_1)\wedge (d_2\to d'_2)\wedge\ldots\wedge (d_n\to d'_n)$ pour un $n$ fini.
\end{them}


\newpage

\section[Interprétation catégorique]{Interprétation dans une catégorie cartésienne fermée du lambda-calcul}

Cette section réutilisera les résultats de la section précédente pour développer un cadre général d'interprétation catégorique. Nous allons donc faire correspondre à nos lambda-termes des fonctions dans une catégorie. Le choix d'une catégorie cartésienne fermée est naturel puisque c'est la notion la plus simple de catégorie possédant des exponentiations, c'est-à-dire des objets de la forme $a\to b$. Nous verrons ainsi, tout d'abord, la définition d'une catégorie cartésienne fermée, pour pouvoir ensuite donner l'interprétation d'un lambda-terme du lambda-calcul simplement typé (sans extension car celles-ci seront traitées dans la partie sur l'interprétation de OCaml).

\subsection{Définitions}

Rappelons la définition d'une catégorie.

\begin{defi}[Catégorie]\label{categorie}
    Une catégorie $\cat$ est une classe composée :
    \begin{itemize}[label=$\bullet$]
        \item d'objets, dont on notera la classe $\cat_0$.
        \item de flèches, dont on notera la classe $\cat_1$. Chaque flèche $f$ possède un domaine, noté $\mathrm{dom}(f)$ et un codomaine, noté $\mathrm{codom}(f)$. On notera plus simplement $f : a\to b$ pour dire que $\mathrm{dom}(f)=a$ et $\mathrm{codom}(f)=b$.
        \item d'une opération de composition, associative, notée $\circ$, qui associe à deux flèches $f : a\to b$ et $g : b\to c$ une flèche $g\circ f : a \to c$.
        \item pour chaque objet $c$, d'une flèche $\id_c$ appelée identité de $c$ telle que pour toute flèche $f : a \to c, \id_c\circ f = f$ et pour toute flèche $g : c \to b, g\circ \id_c = g$.
    \end{itemize}
\end{defi}

Nous donnerons des exemples de catégories classiques et utiles dans notre étude.

\begin{expl}
    \ 
    \begin{itemize}[label=$\bullet$]
        \item La catégorie \textbf{Set} des ensembles avec comme flèches les applications entre les ensembles.
        \item Les catégories $\dcpo$ et $\cpo$ avec comme flèches les application continues.
        \item Les catégories $\adcpo$ et $\acpo$ avec les mêmes flèches.
        \item La catégorie $\scott$ des domaines de Scott avec toujours les applications continues.
    \end{itemize}
\end{expl}

\begin{rmk}
    Le fait que les catégories précédentes en sont bien a été vérifié au long des exercices de la partie précédente, lorsqu'on justifiait la continuité des différentes fonctions.
\end{rmk}

Nous avons, de plus, besoin de trois éléments pour définir une catégorie cartésienne fermée : un objet terminal, un produit et une exponentiation. Nous allons donc définir ces termes dans un premier temps.

\begin{defi}[Objet terminal]
    On appelle objet terminal d'une catégorie $\cat$ un objet, noté $1$, tel que pour tout objet $a\in\cat_0$, il existe une unique flèche $!_a : a \to 1$.
\end{defi}

\begin{defi}[Produit]
    Soit une catégorie $\cat$, deux objets $a$ et $b$. On appelle produit de $a$ et $b$, et on note $a\times b$, l'unique objet à isomorphisme près tel que pour tout objet $x$ et toute paire de flèche $f : x \to a, g : x\to b$ il existe une unique fonction, notée $\langle f,g\rangle$ qui fasse commuter le diagramme de la figure \ref{produit}.
    
    \begin{figure}[t]
        \centering
        \rule{17cm}{0.5pt}
        \begin{tikzcd}
        \\
            & x \ar[dl,"f"]\ar[dr,"g"]\ar[d,dashed,"\langle f\comma g\rangle"] \\
            a & a\times b \ar[l,"\pi_1"]\ar[r,"\pi_2"]& b\\
        \end{tikzcd}
        \rule{17cm}{0.5pt}
        \vspace{-0.5cm}
        \caption{Diagramme commutatif du produit}
        \label{produit}
    \end{figure}
\end{defi}

\begin{defi}[Exponentiation]
    Soient $a$ et $b$ deux objets de $\cat$. On appelle exponentielle de $a$ par $b$ l'objet $a^b$, représentant les fonctions $b\to a$. On se munit d'une fonction $\ev : a^b\times b \to a$ et on a la propriété universelle que pour toute fonction $f : c\times b \to a$, alors il existe une unique fonction $\Lambda(f)$ (appelée curryfication de $f$) telle que le diagramme de la figure \ref{expo} commute.
    \begin{figure}[t]
        \centering
        \rule{17cm}{0.5pt}
        \begin{tikzcd}
            \\
            a^b\times b \ar[r,"\ev"] & a\\
            c\times b \ar[u,"\Lambda(f)\times \id_b"]\ar[ur,"f"]\\
        \end{tikzcd}
        \rule{17cm}{0.5pt}
        \caption{Diagramme commutatif de l'exponentielle}
        \label{expo}
    \end{figure}
\end{defi}

Une catégorie cartésienne fermée est donc une catégorie dans laquelle on a un élément terminale, tous les produits binaires (et donc tous les produits finis, par récurrence évidente) et toutes les exponentiations. Notre cadre d'étude, que sont les domaines de Scott, est une catégorie cartésienne fermée $\scott$, grâce aux exercices précédents.

\subsection{Interprétation catégorique}

Nous pouvons désormais définir l'interprétation catégorique du lambda-calcul simplement typé dans une CCC. Pour cela, on se fixe tout d'abord les objets de notre catégorie. Ceux-ci seront des interprétations de nos types. On notera $\llbracket -\rrbracket$ la fonction d'interprétation, associant à un objet syntaxique (type, terme...) une interprétation sémantique dans notre catégorique. Nous supposons donc que pour chaque type $\tau$ il existe un objet $\llbracket\tau\rrbracket$ associé dans lequel sera interprété notre type. Les morphismes seront les interprétations des jugements de typage.

Nous allons d'abord définir l'interprétation d'un contexte. Soit un contexte $\Gamma$ donné sous la forme $\Gamma = x_1 : \tau_1,\ldots , x_n : \tau_n$, on définit $\llbracket\Gamma\rrbracket=\llbracket\tau_1\rrbracket\times\ldots\times\llbracket\tau_n\rrbracket$. Si $\Gamma = \varnothing$, alors $\llbracket\Gamma\rrbracket=1$. En effet, une flèche $1\to a$ est exactement un élément de $a$, donc un jugement de la forme $\vdash M : \tau$ sera exactement un élément de $\llbracket\tau\rrbracket$. Nous pouvons maintenant, par induction sur la structure d'un lambda-terme simplement typé, définir l'interprétation d'un lambda-terme.

\begin{defi}[Interprétation]
    On définit par induction l'interprétation d'un lambda-terme :
    \begin{itemize}[label=$\bullet$]
        \item Si $\Gamma\vdash x : \tau$, on note $i$ l'indice d'occurrence de $x$ dans $\Gamma$, alors $$\llbracket\Gamma\vdash x : \tau\rrbracket = \pi_i$$ l'interprétation d'une variable est donc une projection du contexte.
        \item Si $\Gamma\vdash \lambda x.M : \sigma\to\tau$, on note $\llbracket\Gamma,x : \sigma\rrbracket=\llbracket\Gamma\rrbracket\times \llbracket\sigma\rrbracket$, ce qui nous permet e curryfier notre fonction :
        $$\llbracket\Gamma\vdash\lambda x.M : \sigma\to\tau\rrbracket=\Lambda\llbracket\Gamma,x : \sigma\vdash M : \tau\rrbracket$$
        \item Si $\Gamma\vdash M\;N : \tau$, nous avons juste à appliquer l'évaluation à $M$ et $N$ :
        $$\llbracket\Gamma\vdash M\;N : \tau\rrbracket = \ev\circ\langle\llbracket\Gamma\vdash M : \sigma\to\tau\rrbracket,\llbracket\Gamma\vdash N : \sigma\rrbracket\rangle$$
    \end{itemize}
\end{defi}

Ainsi, une catégorie cartésienne close, et en particulier $\scott$, permet d'interpréter les éléments de base du lambda-calcul simplement typé.

\newpage

\part{Conclusion}

Nous allons donc pouvoir construire une interprétation du langage OCaml dans la catégorie $\scott$ des domaines de Scott. Plus précisément, nous allons construire la sous-catégorie $\ocaml$ dont les objets sont :
$$\sigma,\tau ::= \mathbb N_\bot \mid \mathbb B_\bot \mid \mathbb U_\bot \mid \sigma\to\tau\mid \sigma\times \tau$$ et dont les morphismes sont les interprétations dans la CCC ainsi construite des termes définis par la syntaxe de $\ocaml$. Nous devons donc définir une fonction d'interprétation $\llbracket - \rrbracket$ qui à un terme syntaxique associe le morphisme de $\ocaml$ qui lui correspond. On confondra ici un objet de la forme $1\to A$ et un élément du domaine de Scott $A$. Par la structure de CCC, nous avons déjà l'interprétation de $\fun\;x\to M$, de $x$ et de $M\; N$, de $\langle M,N\rangle$ et des projections. $\fun\;x\to M$ nous permet de définir $\letin{x}{e}{e'}$. Nous allons détailler l'interprétation des différentes autres constructions.

L'interprétation des constantes booléennes est celle des élément de $\mathbb B$ dans $\mathbb B_\bot$. De même, $k\in\mathbb N$ est interprété par lui-même dans $\mathbb N_\bot$ et $\pare$ est interprété par le seul élément de $\mathbb U$ dans $\mathbb U_\bot$. On notera désormais $\mathbb B_\bot = \{\true,\false,\bot\}$ pour ne pas confondre $\false$ et $\bot$.

\begin{prop}
    On appelle fonction stricte une fonction $f : D\to D$ telle que $f(\bot)=\bot$.
    Une fonction stricte sur un domaine plat est continue.
\end{prop}
\begin{proof}
    La croissance est respectée puisque si $x\leq y$ alors soit $x=\bot$ auquel cas $f(x)\leq f(y)$ soit $x=y$ auquel cas $f(x)\leq f(y)$. Une partie filtrante $\Delta$ contiendra au plus un élément différent de $\bot$. Si elle ne contient que $\bot$, alors l'image sera $\bot$ et la continuité est bien respectée, et s'il existe $a\neq \bot$, alors $\bigvee\Delta = a$ donc $\bigvee f(\Delta)=f(a)$.
\end{proof}

On en déduit donc que l'interprétation de $\mathrm{not}$ par la négation booléenne stricte est continue, on peut donc définir $\llbracket\mathrm{not}\rrbracket : \mathbb B_\bot \to \mathbb B_\bot$.

Nous allons maintenant définir les opérations binaires. Celles-ci seront définie uniquement lorsque leurs deux arguments sont définis. Par exemple, $+$ sera défini comme $\bot$ si l'un des arguments est $\bot$ et comme la somme de ses arguments sinon.

Ainsi on définit $+,-,\times,\andt,\|,\leq$ comme des fonctions continues car elles sont continues en chaque argument (en fixant un argument, on a une fonction stricte sur un domaine plat).

L'interprétation du $\letrec{x}{e}{e'}$ se fait à partir de celle de $Y$, qui est le combinateur de point fixe $Y(f) =\displaystyle{\bigvee_{n\in\mathbb N} f^n(\bot)}$.

Il nous reste enfin à définir l'interprétation $\llbracket\ifthenelse{}{}{}\rrbracket$. Cette fonction est de la forme $\mathbb B_\bot \times D\times D \to D$ où $D$ est un domaine de Scott et renvoie $\bot$ si l'un des arguments est $\bot$ et fait sinon l'association suivante : $(\true,e,e')\mapsto e$ et $(\false,e,e')\mapsto e'$. Il suffit alors d'utiliser la continuité en chaque argument. En effet, pour $\mathbb B_\bot$ on a une fonction stricte sur un domaine plat et pour chaque $D$ la fonction est une projection, qui est donc continue.

Nous pouvons donc donner une sémantique de nos lambda-termes, cf. figure \ref{semantique}. Nous écrirons $\gamma$ pour un élément d'un contexte donné, $\gamma\in\llbracket\Gamma\rrbracket$.

\begin{expl}
    Donnons la sémantique du terme $$\letrec{\fact}{\fun\;x\to\ifthenelse{x=0}{1}{x\times \fact\;(x-1)}}{\fact}$$
    \begin{multline*}
        \llbracket\letrec{\fact}{\fun\;x\to\ifthenelse{x=0}{1}{x\times \fact\;(x-1)}}{\fact}\rrbracket = \bigvee_{n\in\nat}\bigg[ g\mapsto\\ x\mapsto \llbracket\ifthenelse{\!\!\!}{\!\!\!\!}{\!\!\!}\rrbracket \circ \langle \llbracket=\rrbracket\circ \langle 0,x\rangle, 1, \llbracket\times\rrbracket \circ\langle x,\ev\circ\langle g,\llbracket -\rrbracket\circ\langle x,1\rangle\rangle\rangle\rangle\bigg]^n(\bot)
    \end{multline*}
\end{expl}

\begin{figure}[p]
    \centering
    \rule{17cm}{0.5pt}\\
    $$
    \begin{array}{rl}
        \llbracket\Gamma\vdash x : \tau\rrbracket &= \pi_i\\ \\
        \llbracket\Gamma\vdash k : \intt\rrbracket &= \gamma \mapsto k\\ \\
        \llbracket\Gamma\vdash \true : \boolt\rrbracket &= \gamma \mapsto \true\\ \\
        \llbracket\Gamma\vdash \false : \boolt\rrbracket &= \gamma \mapsto \true\\ \\
        \llbracket\Gamma\vdash \fun\; x \to M : \sigma\to\tau\rrbracket &= \Lambda\llbracket\Gamma, x : \sigma\vdash M : \tau\rrbracket\\ \\
        \llbracket\Gamma\vdash M\; N : \tau\rrbracket &= \ev \circ \langle \llbracket\Gamma\vdash M : \sigma \to\tau\rrbracket,\llbracket\Gamma\vdash N : \sigma\rrbracket\rangle\\ \\
        \llbracket\Gamma\vdash \ifthenelse{M}{N}{P} : \tau\rrbracket &= \llbracket\ifthenelse{\!\!\!\!}{\!\!\!\!}{\!\!\!}\rrbracket\circ \langle \llbracket\Gamma\vdash M : \boolt\rrbracket,\llbracket\Gamma\vdash N : \tau\rrbracket,\llbracket\Gamma\vdash P : \tau\rrbracket\rangle\\ \\
        \llbracket\Gamma\vdash \letin{x}{M}{N} : \tau\rrbracket &= \ev\circ  \langle    \Lambda\llbracket\Gamma,x : \sigma\vdash N : \tau \rrbracket,\llbracket\Gamma\vdash M : \sigma\rrbracket\rangle \\ \\
        \llbracket\Gamma\vdash \letrec{x}{M}{N} : \tau\rrbracket &= \ev\circ  \langle    \Lambda\llbracket\Gamma,x : \sigma\vdash N : \tau \rrbracket,\displaystyle{\bigvee_{n\in\nat}} \Lambda\llbracket\Gamma,x : \sigma\vdash M : \sigma\to\sigma\rrbracket^n(\bot_{\llbracket\sigma\rrbracket})\rangle \\ \\
        \llbracket\Gamma\vdash M + N : \intt\rrbracket &= \llbracket+\rrbracket \circ \langle\llbracket\Gamma\vdash M : \intt\rrbracket,\llbracket\Gamma\vdash N : \intt\rrbracket\rangle\\ \\
        \llbracket\Gamma\vdash M - N : \intt\rrbracket &= \llbracket-\rrbracket \circ \langle\llbracket\Gamma\vdash M : \intt\rrbracket,\llbracket\Gamma\vdash N : \intt\rrbracket\rangle\\ \\
        \llbracket\Gamma\vdash M \times N : \intt\rrbracket &= \llbracket\times\rrbracket \circ \langle\llbracket\Gamma\vdash M : \intt\rrbracket,\llbracket\Gamma\vdash N : \intt\rrbracket\rangle\\ \\
        \llbracket\Gamma\vdash M \| N : \boolt\rrbracket &= \llbracket\|\rrbracket \circ \langle\llbracket\Gamma\vdash M : \boolt\rrbracket,\llbracket\Gamma\vdash N : \boolt\rrbracket\rangle\\ \\
        \llbracket\Gamma\vdash M \andt N : \boolt\rrbracket &= \llbracket\andt\rrbracket \circ \langle\llbracket\Gamma\vdash M : \boolt\rrbracket,\llbracket\Gamma\vdash N : \boolt\rrbracket\rangle\\ \\
        \llbracket\Gamma\vdash \nott (M) : \boolt\rrbracket &= \llbracket\nott\rrbracket \circ \llbracket\Gamma\vdash M : \boolt\rrbracket \\ \\
        \llbracket\Gamma\vdash M \leq N : \boolt\rrbracket &= \llbracket\leq\rrbracket \circ \langle\llbracket\Gamma\vdash M : \intt\rrbracket,\llbracket\Gamma\vdash N : \intt\rrbracket\rangle\\ \\
        \llbracket\Gamma\vdash \langle M,N\rangle : \sigma\times \tau\rrbracket &= \langle \llbracket\Gamma\vdash M : \sigma\rrbracket,\llbracket\Gamma\vdash N : \tau\rrbracket\rangle\\ \\
        \llbracket\Gamma\vdash \pi_1M : \tau\rrbracket &= \ev \circ \langle \pi_1,\llbracket\Gamma\vdash M :\tau\times\sigma\rrbracket\rangle\\ \\
        \llbracket\Gamma\vdash \pi_2M : \tau\rrbracket &= \ev \circ \langle \pi_2,\llbracket\Gamma\vdash M :\tau\times\sigma\rrbracket\rangle\\ \\
        
    \end{array}
    $$
    \rule{17cm}{0.5pt}
    \caption{Sémantique de OCaml}
    \label{semantique}
\end{figure}

\end{document}
