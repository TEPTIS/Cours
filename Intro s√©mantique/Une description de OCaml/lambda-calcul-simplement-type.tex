\section[Lambda-calcul typé]{Le lambda-calcul simplement typé}

Comme nous l'avons vu, le lambda-calcul possède un problème majeur : nous n'avons aucune assurance que calculer un terme convergera. Pour contrer ce fait, on utilise des types, qui sont des annotations de nos lambda-termes. Nous verrons dans cette partie ce qui constitue le lambda-calcul simplement typé, puis des résultats concernant ce lambda-calcul (principalement la forte normalisation). Enfin, nous explorerons des extensions classiques telles que l'ajout de types produits.

\subsection{Définition des types}

Nous travaillerons dans un premier temps dans un lambda-calcul ne possédant qu'un type de base, noté $\iota$. Un type précise dans quel univers est contenu un terme. Pour dire qu'un terme $M$ est annoté du type $\tau$, on note $M :\tau$.

\begin{defi}[Type]
    On définit la grammaire des types :
    $$\tau,\sigma ::= \iota\mid\tau\to\sigma$$
    
    Nous lisons $\to$ avec une associativité à droite, ce qui signifie que $\tau_1\to\tau_2\to\tau_3 = \tau_1\to(\tau_2\to\tau_3)$.
\end{defi}

De plus, nous noterons désormais $\lambda x^\tau.M$ plutôt que $\lambda x.M$ pour préciser le type de $x$.

Pour pouvoir définir la relation de typage, il convient de définir plusieurs éléments. Tout d'abord, intéressons-nous à ce qu'est un environnement de typage. C'est un ensemble (que l'on note comme une liste d'éléments séparés par des virgules) de couples $(x,\tau)$ où $x$ est une variable et $\tau$ un type. On notera $x : \tau$ les éléments de cet ensemble. Un environnement de typage pourra par exemple être $\Gamma = x : \iota, f : \iota\to\iota, g : \iota\to\iota\to\iota$. De plus, on note $\Gamma,x : \tau$ pour noter l'ensemble $\Gamma\cup \{(x,\tau)\}$. Cela nous permet de définir un jugement de typage, qui se fait dans un environnement.

\begin{defi}[Jugement de typage]
    On définit $\Gamma\vdash M : \tau$ un jugement de typage par induction sur la structure de $M$ :
    $$(x,\tau)\in\Gamma\;\dfrac{}{\Gamma\vdash x : \tau}\;\mathrm{Ax}\qquad \dfrac{\Gamma,x : \tau\vdash M : \sigma}{\Gamma\vdash\lambda x^\tau.M : \tau\to\sigma}\;\mathrm{Abs}\qquad\dfrac{\Gamma\vdash M : \sigma \to \tau \quad \Gamma\vdash N : \sigma}{\Gamma\vdash M N : \tau}\;\mathrm{App}$$
    
    Lorsque $\Gamma=\varnothing$, on note $\vdash M : \tau$ directement, et on dit alors que $M$ est typable, ou de type $\tau$.
\end{defi}

\begin{expl}
    \ 
    \begin{itemize}[label=$\bullet$]
        \item On montre que $\vdash \lambda x^\iota.x : \iota \to\iota$ :
        \begin{center}
            \begin{prooftree}
                \infer{0}[Ax]{x :\iota\vdash x : \iota}
                \infer{1}[Abs]{\vdash\lambda x^\iota.x : \iota\to\iota}
            \end{prooftree}
        \end{center}
        \item On montre aussi que $\vdash\lambda f^{\iota\to\iota}.\lambda x^\iota. f\;x : (\iota\to\iota)\to\iota\to\iota$ :
        \begin{center}
            \begin{prooftree}
                \infer{0}[Ax]{f : \iota\to\iota,x : \iota\vdash f :\iota\to\iota}
                \infer{0}[Ax]{f : \iota\to\iota,x : \iota\vdash x : \iota}
                \infer{2}[App]{f : \iota\to\iota,x : \iota\vdash f\;x : \iota}
                \infer{1}[Abs]{f : \iota\to\iota\vdash.\lambda x^\iota. f\;x : (\iota\to\iota)\to\iota\to\iota}
                \infer{1}[Abs]{\vdash \lambda f^{\iota\to\iota}.\lambda x^\iota. f\;x : (\iota\to\iota)\to\iota\to\iota}
            \end{prooftree}
        \end{center}
    \end{itemize}
\end{expl}

\begin{exo}
    Typer le lambda-terme $\lambda f^{\iota\to\iota}.\lambda g^{\iota\to\iota}.\lambda x^\iota.f\;(g\;x)$ puis généraliser ce typage à un lambda-terme identique pour lequel on remplace $\iota$ par un type indéterminé $\tau$.
\end{exo}

\subsection{Propriétés du lambda-calcul simplement typé}

Tout d'abord, le lambda-calcul simplement typé, comme le non typé, est confluent et possède la propriété de Church-Rosser. Nous allons maintenant énumérer les autres propriétés de ce lambda-calcul.

\begin{them}[Préservation du typage]
    Soit un lambda-terme typé dans un contexte $\Gamma\vdash M : \tau$ et $N$ tel que $M\betareq N$. Alors $\Gamma\vdash N :\tau$.
\end{them}

De plus, le lambda-calcul simplement typé est fortement normalisant.

\begin{defi}
    Si $M$ est un terme bien typé, tel que $\vdash M :\tau$, alors toute suite de réduction depuis $M$ est finie (et converge donc vers une unique forme normale).
\end{defi}

\subsection{Extensions du langage}

Le lambda-calcul simplement typé, dans son état actuel, possède un souci majeur : il ne permet pas d'exprimer l'arithmétique. Nous allons donc explorer une première extension du langage dans laquelle l'arithmétique est présente. Nous en profiterons pour ajouter la logique booléenne.

\subsubsection{Ajout de l'arithmétique et des booléens}

Cette partie traitera de l'ajout du type \intt et du type \boolt.Si nous avons codé les entiers naturels et les booléens dans le lambda-calcul non typé, le lambda-calcul typé, de par sa restriction forte sur la nature de ses objets, impose de définir les opérations arithmétiques et logiques de base dans la syntaxe du langage.

\begin{defi}
    Nous définissons la grammaire des types et du langage :
    $$\sigma,\tau ::= \intt\mid\boolt\mid \sigma\to\tau$$
        \begin{multline*}
        M,N,P ::= x\mid \lambda x^\tau.M\mid M N\mid k\mid \top\mid\bot\mid M+N\mid M\times N\mid M-N\mid M\lor N\mid M \land N\mid \lnot M\\\mid M\leq N\mid\ifthenelse{M}{N}{P}
        \end{multline*}
    où $k\in\mathbb N$.
\end{defi}

Nous devons alors redéfinir nos règles de typage en ajoutant celles nécessaires à l'ajout des opérations que nous avons.

\begin{defi}[Typage]
    Les nouvelles règles de typages sont les suivantes :
    $$\dfrac{}{\Gamma\vdash k : \intt}\qquad\dfrac{}{\Gamma\vdash \top : \boolt}\qquad\dfrac{}{\Gamma\vdash\bot : \boolt}$$
    $$\dfrac{\Gamma\vdash M : \intt\quad \Gamma\vdash N :\intt}{\Gamma\vdash M+N : \intt}\qquad \dfrac{\Gamma\vdash M : \intt\quad\Gamma\vdash N:\intt}{\Gamma\vdash M\times N : \intt}\qquad \dfrac{\Gamma\vdash M : \intt\quad\Gamma\vdash N:\intt}{\Gamma\vdash M- N : \intt}$$
    $$\dfrac{\Gamma\vdash M :\boolt \quad\Gamma\vdash N : \boolt}{\Gamma\vdash M \lor N : \boolt}\qquad \dfrac{\Gamma\vdash M :\boolt \quad\Gamma\vdash N : \boolt}{\Gamma\vdash M \land N : \boolt}\qquad \dfrac{\Gamma\vdash M :\boolt}{\Gamma\vdash \lnot M : \boolt}\qquad$$
    $$\dfrac{\Gamma\vdash M : \intt\quad \Gamma\vdash N :\intt}{\Gamma\vdash M\leq N : \boolt}\qquad\dfrac{\Gamma\vdash M : \boolt\quad \Gamma\vdash N : \tau\quad \Gamma\vdash P : \tau}{\Gamma\vdash\ifthenelse{M}{N}{P} : \tau}$$
\end{defi}

Enfin, il convient aussi de redéfinir la réduction de termes. Nous allons noter $b$ un booléen et $k$ un entier, sous-entendu dans cette notation que les expressions associées sont réduites au maximum. Les ajouts à gauche d'une règles sont des conditions à vérifier pour appliquer la règle qui ne sont pas d'ordre syntaxique. Par exemple dans la première règle qui sera présentée, $k+k'=m$ sert simplement à préciser la valeur de $m$, mais cette condition n'est pas vérifiable par le système syntaxique. \'Evidemment, nous gardons pour autant la règle de la beta-réduction, mais nous allons y ajouter d'autres règles. Nous noterons simplement $\to$ la réduction ainsi faite.

\begin{defi}[Réduction]
    Voici les règles de réduction supplémentaires :
    $$\dfrac{M\to M'}{M+N \to M' + N}\qquad\dfrac{N \to N'}{M+N\to M+N'}\qquad k+k'=n\;\dfrac{}{k+k'\to n}$$
    $$\dfrac{M\to M'}{M\times N \to M' \times N}\qquad\dfrac{N \to N'}{M\times N\to M\times N'}\qquad k\times k'=m\;\dfrac{}{k\times k'\to n}$$
    $$ \dfrac{M\to M'}{M-N \to M' - N}\qquad\dfrac{N \to N'}{M-N\to M-N'}\qquad k-k'=n\;\dfrac{}{k-k'\to n}$$
    $$\dfrac{M\to M'}{M\lor N \to M' \lor N}\qquad\dfrac{N \to N'}{M\lor N\to M\lor N'}\qquad b\lor b'=c\;\dfrac{}{b\lor b'\to c}$$
    $$\dfrac{M\to M'}{M\land  N \to M' \land  N}\qquad\dfrac{N \to N'}{M\land  N\to M\land  N'}\qquad b\land  b'=c\;\dfrac{}{b\land  b'\to c}$$
    $$\dfrac{M\to M'}{\lnot M\to\lnot M'}\qquad\lnot b = b'\;\dfrac{}{\lnot b\to b'}$$
    $$k\leq k'\;\dfrac{}{k\leq k'\to \top}\qquad k>k'\;\dfrac{}{k\leq k' \to\bot} $$
    $$\dfrac{}{\ifthenelse{\top}{M}{N}\to M}\qquad\dfrac{}{\ifthenelse{\bot}{M}{N}\to N}$$
\end{defi}

Les propriétés du lambda-calcul simplement typé sont encore conservées en utilisant ces ajouts.

\begin{exo}
    \ 
    \begin{itemize}[label=$\bullet$]
        \item Effectuer la réduction du lambda-terme $\ifthenelse{0=27\times (0+0)}{(\lambda x.x+1) 3}{5}$
        \item Typer le lambda-terme $\lambda x. \ifthenelse{(x-5 \leq 0)\lor (6\leq x)}{1}{0}$
    \end{itemize}
\end{exo}

\subsubsection{Ajout des couples}

Pour ajouter ce que nous avions utilisé dans le lambda-calcul non typé, il nous manque la récursion et les couples de valeurs. La récursion sera le problème d'une partie suivante, les couples seront étudiés ici.

Pour éviter de réécrire toute la grammaire que nous avions écrite précédemment, nous écrirons simplement $G$ pour l'ensemble des règles de construction définies.

\begin{defi}
    La grammaire des types est la suivante :
    $$\sigma,\tau ::= \intt\mid\boolt\mid\sigma\to\tau\mid\sigma\times\tau$$
    
    Celles des termes vaut :
    $$M,N,P ::= G \mid \langle M,N\rangle\mid \pi_1 M\mid\pi_2 M$$
\end{defi}

\begin{defi}[Typage]
    Les règles de typage ajoutées sont :
    $$\dfrac{\Gamma\vdash M : \sigma\quad\Gamma\vdash N : \tau}{\Gamma\vdash \langle M,N\rangle : \sigma\times \tau}\qquad\dfrac{\Gamma\vdash M : \sigma\times\tau}{\Gamma\vdash \pi_1 M : \sigma}\qquad\dfrac{\Gamma\vdash M : \sigma\times \tau}{\Gamma\vdash \pi_2 M : \tau}$$
\end{defi}

\begin{defi}[Réduction]
    Les règles de réduction ajoutées sont :
    $$\dfrac{M\to M'}{\langle M,N\rangle\to\langle M',N\rangle}\qquad\dfrac{N\to N'}{\langle M,N\rangle\to\langle M,N'\rangle}$$
    $$\dfrac{}{\pi_1\langle M,N\rangle \to M}\qquad \dfrac{}{\pi_2\langle M,N\rangle \to N}$$
\end{defi}

\begin{exo}
    Typer la fonction $\lambda x.\lambda y.\langle x,y\rangle$.
\end{exo}

\newpage